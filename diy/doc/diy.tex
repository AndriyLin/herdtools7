\documentclass{article}
\usepackage{a4}
\usepackage{fullpage}
\usepackage[T1]{fontenc}
\usepackage{url}
\urldef{\urlfamilies}{\url}{http://www.cl.cam.ac.uk/~pes20/ppc-supplemental/test6.pdf}
\urldef{\urlgraphviz}{\url}{http://graphviz.org/}
\urldef{\urlgv}{\url}{http://www.gnu.org/software/gv/}
\usepackage{hevea}
\usepackage{moreverb}
\usepackage{graphicx}
\usepackage{color}
\usepackage{macros}
\usepackage{amsmath}
\usepackage{amssymb}
\usepackage{array}
\usepackage{syntaxdef}
\title{A \prog{don't (diy)} tutorial}
\author{Version \version}
\date{\today}

\begin{document}
\maketitle

\diy{} is a tool suite for testing shared memory models.  We
provide several tools, \prog{litmus} (Part~\ref{part:litmus}) for running tests,
\diy{} generators (Part~\ref{part:diy}) for producing tests from concise
specifications, and
\herd{} (Part~\ref{part:herd}) for simulating memory models.
%\jingle{} (Part~\ref{part:jingle}) for converting tests.
In Part~\ref{part:examples} we describe
a few concrete experiments, illustrating frequent usage patterns of
\diy{} generators and of \litmus.
Finally (Part~\ref{part:auto}), we briefly
describe our experimental \dont{} tool
for either checking the
conformance of a machine to an architecture or exploring the memory model of a
given machine automatically.

The software is written in
\footahref{http://caml.inria.fr/ocaml/}{Objective Caml}, and released as
sources.  The web site of~\diy{} is
\ahrefurl{\url{http://diy.inria.fr/}}, authors can be contacted at
\texttt{diy-devel@inria.fr}.
This software is released under the terms of the Lesser GNU Public License.

\begin{htmlonly}
\begin{quote}
This document in \ahref{diy.ps}{\sc Postscript} and \ahref{diy.pdf}{{\sc Pdf}}.
\end{quote}
\end{htmlonly}

\cutdef{section}
\tableofcontents
\cutend

\clearpage
\part{Running\label{part:litmus} tests with \prog{litmus}}
\cutname{litmus.html}
Traditionally, a \emph{litmus test} is a small parallel program designed
to exercise the memory model of a parallel, shared-memory, computer.
Given a litmus test in assembler (X86, Power or ARM) \litmus{}
runs the test.

Using \litmus{} thus requires a parallel machine,
which must additionally feature \prog{gcc} and the \prog{pthreads} library.
Our tool \litmus{} has some limitations especially
as regards recognised instructions.
Nevertheless, \litmus{} should accept all tests
produced by the companion test generators (see Part~\ref{part:diy})
and has been successfully used on Linux, MacOS, AIX and~Android.


\section{A tour of~\litmus{}}

\subsection{A \label{litmus:simple} simple run}
\aname{SB}{Consider} the following (rather classical, store buffering)
\afile{SB.litmus} litmus test for X86:
\verbatiminput{SB.litmus}
A litmus test source has three main sections:
\begin{enumerate}
\item The initial state defines the initial values of registers
and memory locations. Initialisation to zero may be omitted.
\item The code section defines the code to be run concurrently
--- above there are two threads.
Yes we know, our X86 assembler syntax is a mistake.
\item The final condition applies to the final values
of registers and memory locations.
\end{enumerate}

\label{x86:classic}Run the test by\ifhevea{}
(complete \ahref{SB.log}{run log})\fi:
\begin{verbatim}
% litmus7 SB.litmus
%%%%%%%%%%%%%%%%%%%%%%%%%
% Results for SB.litmus %
%%%%%%%%%%%%%%%%%%%%%%%%%
X86 SB
"Fre PodWR Fre PodWR"

{x=0; y=0;}

 P0          | P1          ;
 MOV [x],$1  | MOV [y],$1  ;
 MOV EAX,[y] | MOV EAX,[x] ;

exists (0:EAX=0 /\ 1:EAX=0)
Generated assembler
#START _litmus_P1
	movl $1,(%r10)
	movl (%r9),%eax
#START _litmus_P0
	movl $1,(%r9)
	movl (%r10),%eax

Test SB Allowed
Histogram (4 states)
40    *>0:EAX=0; 1:EAX=0;
499923:>0:EAX=1; 1:EAX=0;
500009:>0:EAX=0; 1:EAX=1;
28    :>0:EAX=1; 1:EAX=1;
Ok

Witnesses
Positive: 40, Negative: 999960
Condition exists (0:EAX=0 /\ 1:EAX=0) is validated
Hash=7dbd6b8e6dd4abc2ef3d48b0376fb2e3
Observation SB Sometimes 40 999960
Time SB 0.44
...
\end{verbatim}
The  litmus test is first reminded, followed by actual assembler
--- the machine is a 64~bits one, in-line address references disappeared,
registers may change, and assembler syntax is now more familiar.
The test has run one million times, producing one million final states,
or \emph{outcomes} for the registers \verb+EAX+ of threads \P{0} and~\P{1}.
The test run validates the condition, with $40$~positive witnesses.



\subsection{Cross\label{cross} compilation}

With option \opt{-o <name.tar>}, \litmus{} does not run the test.
Instead, it produces a tar archive that contains the C~sources for
the test.

Consider \afile{SB-PPC.litmus}, a Power version
of the previous test:
\verbatiminput{SB-PPC.litmus}


Our target machine (ppc) runs MacOS, which we specify with the \opt{-os}~option:
\begin{verbatim}
% litmus7 -o /tmp/a.tar -os mac SB-PPC.litmus
% scp /tmp/a.tar ppc:/tmp
\end{verbatim}
Then, on the remote machine ppc:
\begin{verbatim}
ppc% mkdir SB && cd SB
ppc% tar xf /tmp/a.tar
ppc% ls
comp.sh  Makefile  outs.c  outs.h  README.txt  run.sh  SB-PPC.c  show.awk  utils.c  utils.h
\end{verbatim}
Test is compiled by the shell script \file{comp.sh} (or
by (Gnu) \texttt{make}, at user's choice) and run by the shell script
\file{run.sh}:
\begin{verbatim}
ppc% sh comp.sh
ppc% sh run.sh
...
Test SB-PPC Allowed
Histogram (3 states)
1784  *>0:r3=0; 1:r3=0;
498564:>0:r3=1; 1:r3=0;
499652:>0:r3=0; 1:r3=1;
Ok

Witnesses
Positive: 1784, Negative: 998216
Condition exists (0:r3=0 /\ 1:r3=0) is validated
Hash=4edecf6abc507611612efaecc1c4a9bc
Observation SB-PPC Sometimes 1784 998216
Time SB-PPC 0.55
...
\end{verbatim}
%$
\ifhevea(Complete \ahref{SB-PPC.log}{run log}.) \fi
As we see, the condition validates also on Power.
Notice that compilation produces an executable file, \file{SB-PPC.exe},
which can be run directly, for a less verbose output.


\subsection{Running several tests at once}

\label{stfw}Consider the additional test~\afile{STFW-PPC.litmus}:
\verbatiminput{STFW-PPC.litmus}

To compile the two tests together,
we can give two file names as arguments to litmus:
\begin{verbatim}
$ litmus7 -o /tmp/a.tar -os mac SB-PPC.litmus STFW-PPC.litmus 
\end{verbatim}
Or, more conveniently, list the litmus sources in a file whose
name starts with \file{@}:
\begin{verbatim}
$ cat @ppc
SB-PPC.litmus
STFW-PPC.litmus
$ litmus7 -o /tmp/a.tar -os mac @ppc
\end{verbatim}
%$
To run the test on the remote ppc machine, the same sequence
of commands as in the one test case applies:
\begin{verbatim}
ppc% tar xf /tmp/a.tar && make && sh run.sh
...
Test SB-PPC Allowed
Histogram (3 states)
1765  *>0:r3=0; 1:r3=0;
498741:>0:r3=1; 1:r3=0;
499494:>0:r3=0; 1:r3=1;
Ok

Witnesses
Positive: 1765, Negative: 998235
Condition exists (0:r3=0 /\ 1:r3=0) is validated
Hash=4edecf6abc507611612efaecc1c4a9bc
Observation SB-PPC Sometimes 1765 998235
Time SB-PPC 0.57
...
Test STFW-PPC Allowed
Histogram (4 states)
480   *>0:r3=1; 0:r4=0; 1:r3=1; 1:r4=0;
499560:>0:r3=1; 0:r4=1; 1:r3=1; 1:r4=0;
499827:>0:r3=1; 0:r4=0; 1:r3=1; 1:r4=1;
133   :>0:r3=1; 0:r4=1; 1:r3=1; 1:r4=1;
Ok

Witnesses
Positive: 480, Negative: 999520
Condition exists (0:r3=1 /\ 0:r4=0 /\ 1:r3=1 /\ 1:r4=0) is validated
Hash=92b2c3f6332309325000656d0632131e
Observation STFW-PPC Sometimes 480 999520
Time STFW-PPC 0.56
...
\end{verbatim}
\ifhevea(Complete \ahref{SB-PPC2.log}{run log}.) \fi
Now, the output of \file{run.sh} shows the result of two tests.

\section{Controlling\label{litmus:control} test parameters}
Users can control some of testing conditions.
Those impact efficiency and outcome variability.

Sometimes one looks for a particular outcome
--- for instance, one may seek to get the
outcome \verb+0:r3=1; 1:r3=1;+ that is missing
in the previous experiment for test~\ltest{SB-PPC}.
To that aim, varying test conditions may help.


\subsection{Architecture\label{sec:arch} of tests}

Consider a test \file{a.litmus}
designed to run on $t$ threads \P{0},\ldots, \P{t-1}.
The structure of the executable \file{a.exe} that performs
the experiment is as follows:
\begin{itemize}
\item \label{defn}\label{defa}So as to benefit from parallelism, 
we run $n = \max(1,a/t)$ (integer division)
tests concurrently on a machine where $a$~logical processors are available.
\item \label{defr}Each of these (identical)
tests consists in repeating $r$ times
the following sequence:
\begin{itemize}
\item Fork ~$t$ (POSIX) threads $T_0,\ldots T_{t-1}$
for executing \P{0},\ldots, \P{t-1}.
Which thread executes which code is either fixed, or changing,
controlled by the \emph{launch mode}.
In our experience, the launch mode has marginal impact.

In \emph{cache mode} the $T_k$ threads are re-used.
As a consequence, $t$~threads only are forked.

\item \label{defs}Each thread $T_k$ executes a loop of size~$s$.
Loop iteration number~$i$ executes the code of \P{k} (in fixed mode)
and saves
the final contents of its observed registers in some arrays indexed by~$i$.
Furthermore, still for iteration~$i$, memory location~$x$ is in fact
an array cell.

\label{defmemorymode}\label{defstride}How this array cell is accessed depends
upon the \emph{memory mode}.
In \emph{direct mode} the array cell is accessed directly as~$x[i]$;
as a result, cells are accessed sequentially and false sharing effects
are likely.
In \emph{indirect mode} the array cell is accessed by the means of a
shuffled array of pointers;
as a result we observed a much greater variability of outcomes.
Additionally, the increment of the main loop (of size~$s$)
can be set to a value or \emph{stride} different from the default of~one.
Running a test several times with changing the stride value also
proved quite effective in favouring outcome variability.



\label{defpreload}If the \emph{random preload mode} is enabled,
a preliminary loop of size~$s$ reads
a random subset of the memory locations accessed by~\P{k}.
Preload have a noticeable effect and teh random preload mode is
enabled by default.
Starting from version~5.0, we provide a more precise control
over preloading memory locations --- See Sec.~\ref{preload:custom}.


\label{defsynchronisation}The iterations performed
by the different threads~$T_k$ may be unsynchronised,
exactly synchronised by a pthread based barrier, or approximately synchronised
by specific code.
Absence of synchronisation may be interesting when $t$ exceeds~$a$.
As a matter of fact, in this situation,
any kind of synchronisation leads to prohibitive running times.
However, for a large value of parameter~$s$ and small $t$ we have observed
spontaneous concurrent execution of some iterations amongst many.
Pthread based barriers are exact but they are slow
and in fact offers poor synchronisation for short code sequences.
The approximate synchronisation is thus the preferred technique.

Starting from version 5.0, we provide a slightly altered
user synchronisation mode: \emph{userfence}, which alters
user mode by executing memory fences to speedup write propagation.
The new mode features overall better synchronisation, yielding dramatic
improvements on some examples. However,
outcome variability may suffer from this more accurate synchronisation,
hence user mode remains the default.

\label{timebase:intro}More importantly,
we provide an additional exact, \emph{timebase}
synchronisation
technique: test threads will first synchronise using polling synchronisation
barrier code,
agree on a target timebase\footnote{Power and x86-based systems
provide a user accessible timebase counter that should provide
consistent times to all cores and processors.}  value and then loop
reading the timebase until it exceeds the target value.
This technique yields very good synchronisation and allows
fine synchronisation tuning by assigning different starting delays to
different threads --- see Sec.~\ref{timebase}.
As ARM does not provide timebase counters,
notice that ``timebase'' synchronisation for ARM silently degrades
to synchronisation by the means of the polling synchronisation barrier.


\item Wait for the $t$ threads to terminate and collect outcomes
in some histogram like structure.
\end{itemize}
\item Wait for the $n$~tests to terminate and sum their histograms.
\end{itemize}

Hence, running \file{a.exe} produces $n \times r \times s$ outcomes.
Parameters $n$, $a$, $r$ and~$s$ can first be set directly while
invoking \file{a.exe}, using the appropriate command line options.
For instance, assuming $t=2$,
\verb+./a.exe -a 201 -r 10k -s 1+ and \verb+./a.exe -n 1 -r 1 -s 1M+
will both produce one million outcomes, but the latter is probably
more efficient.
If our machine has $8$~cores,
\verb+./a.exe -a 8 -r 1 -s 1M+ will yield $4$~millions outcomes,
in a time that we hope not to exceed too much the one experienced
with~\verb+./a.exe -n 1+.
Also observe that the memory allocated is roughly proportional
to $n \times s$, while the number of $T_k$~threads created will be
$t \times n \times r$ ($t \times n$ in cache mode).
The \file{run.sh} shell script transmits its command line to all
the executable (\file{.exe}) files
it invokes, thereby providing a convenient means
to control testing condition for several tests.
Satisfactory test parameters are found by experimenting and
the control of executable files by command line options is designed for
that purpose.

Once satisfactory parameters are found, it is a nuisance to repeat them
for every experiment.
Thus, parameters $a$, $r$ and~$s$ can also be set while invoking litmus,
with the same command line options. In fact those settings command
he default values of \file{.exe}~files controls.
Additionally, the synchronisation technique for iterations,
the memory mode, and several others compile time parameters
can be selected by appropriate \litmus{} command line options.
Finally, users can record frequently used parameters in configuration files.

\subsection{Affinity\label{sec:affinity}}

We view affinity as
a scheduler property that binds a (software, POSIX) thread to
a given (hardware) \emph{logical processor}.
In the most simple situation a logical processor is a core.
However in the presence of hyper-threading (x86) or simultaneous multi threading
(SMT, Power) a given core can host several logical processors.

\subsubsection{Introduction to affinity}
In our experience,
binding the threads of test programs to selected logical processors
yields significant speedups and, more importantly, greater outcome variety.
We illustrate the issue by the means of an example.

We consider the test~\afile{ppc-iriw-lwsync.litmus}:
\verbatiminput{ppc-iriw-lwsync.litmus}
The test consists of four threads.
There are  two writers (P0 and P2) that write the value
one into two different locations (\texttt{x} and \texttt{y}),
and two readers that read the contents of \texttt{x} and \texttt{y}
in different orders --- P1 reads \texttt{x} first, while P3 reads
\texttt{y} first.
The load instructions \texttt{lwz} in reader threads are separated
by a lightweight barrier instruction~\texttt{lwsync}.
The final condition \verb+exists (1:r1=1 /\ 1:r3=0 /\ 3:r1=1 /\ 3:r3=0)+
characterises the situation where the reader threads see the writes
by P0 and P2 in opposite order.
The corresponding outcome  \verb+1:r1=1; 1:r3=0; 3:r1=1; 3:r3=0;+
is the only non-sequential consistent (non-SC, see Part~\ref{part:diy}) possible outcome.
By any reasonable memory model for Power,
one expects the condition to validate,
\emph{i.e.} the non-SC outcome to show up.

The tested machine
\ahref{http://www.idris.fr/su/Scalaire/vargas/hw-vargas.html}{\texttt{vargas}}
is a Power~6 featuring 32~cores (\emph{i.e.}
64 logical processors, since SMT is enabled) and running AIX in 64 bits mode.
So as not to disturb other users, we run only one instance of the test,
thus specifying four available processors.
The \litmus{} tool is absent on \texttt{vargas}.
All these conditions command the following invocation of \litmus{},
performed on our local machine:
\begin{verbatim}
$ litmus7 -r 1000 -s 1000 -a 4 -os aix -ws w64 ppc-iriw-lwsync.litmus -o ppc.tar
$ scp ppc.tar vargas:/var/tmp
\end{verbatim}
On \texttt{vargas} we unpack the archive and compile the test:
\begin{verbatim}
vargas% tar xf /var/tmp/ppc.tar && sh comp.sh
\end{verbatim}
Then we run the test:
\begin{verbatim}
vargas% ./ppc-iriw-lwsync.exe
Test ppc-iriw-lwsync Allowed
Histogram (15 states)
163674:>1:r1=0; 1:r3=0; 3:r1=0; 3:r3=0;
34045 :>1:r1=1; 1:r3=0; 3:r1=0; 3:r3=0;
40283 :>1:r1=0; 1:r3=1; 3:r1=0; 3:r3=0;
95079 :>1:r1=1; 1:r3=1; 3:r1=0; 3:r3=0;
33848 :>1:r1=0; 1:r3=0; 3:r1=1; 3:r3=0;
72201 :>1:r1=0; 1:r3=1; 3:r1=1; 3:r3=0;
32452 :>1:r1=1; 1:r3=1; 3:r1=1; 3:r3=0;
43031 :>1:r1=0; 1:r3=0; 3:r1=0; 3:r3=1;
73052 :>1:r1=1; 1:r3=0; 3:r1=0; 3:r3=1;
1     :>1:r1=0; 1:r3=1; 3:r1=0; 3:r3=1;
42482 :>1:r1=1; 1:r3=1; 3:r1=0; 3:r3=1;
90470 :>1:r1=0; 1:r3=0; 3:r1=1; 3:r3=1;
30306 :>1:r1=1; 1:r3=0; 3:r1=1; 3:r3=1;
43239 :>1:r1=0; 1:r3=1; 3:r1=1; 3:r3=1;
205837:>1:r1=1; 1:r3=1; 3:r1=1; 3:r3=1;
No

Witnesses
Positive: 0, Negative: 1000000
Condition exists (1:r1=1 /\ 1:r3=0 /\ 3:r1=1 /\ 3:r3=0) is NOT validated
Hash=4fbfaafa51f6784d699e9bdaf5ba047d
Observation ppc-iriw-lwsync Never 0 1000000
Time ppc-iriw-lwsync 1.32
\end{verbatim}
%$
The non-SC outcome does not show up.

Altering parameters may yield this outcome.
In particular, we may try using all the available logical processors
with option \texttt{-a 64}.
Affinity control offers an alternative, which is enabled at compilation time
with \litmus{} option \texttt{-affinity}:
\begin{verbatim}
$ litmus7 ... -affinity incr1 ppc-iriw-lwsync.litmus -o ppc.tar
$ scp ppc.tar vargas:/var/tmp
\end{verbatim}
Option \texttt{-affinity} takes one argument (\texttt{incr1} above)
that specifies the increment used while allocating
logical processors to test threads.
Here, the  (POSIX) threads created by the test
(named $T_0$, $T_1$, $T_2$ and $T_3$ in Sec.~\ref{sec:arch})
will get bound to logical processors
$0$, $1$, $2$, and~$3$, respectively.

\label{defi}Namely, by default, the logical processors are
ordered as the sequence $0, 1, \ldots, A-1$ ---
where $A$ is the number of available logical processors, which is
inferred by the test executable\footnote{Parameter $A$ is not to be confused with~$a$ of section~\ref{sec:arch}. The former  serves to compute logical threads while the latter governs the number of tests that run simultaneously. However
parameters~$a$ will be set to~$A$ when affinity control is enabled and when
$a$~value is~$0$.}.
Furthermore, logical processors are allocated to threads by
applying the affinity increment while scanning the logical processor sequence.
Observe that since the launch mode is changing (the default) threads
$T_k$ correspond to different test threads~$P_i$ at each run.
The unpack compile and run sequence on \texttt{vargas} now yields
the non-SC outcome, better outcome variety and a lower running time:
\begin{verbatim}
vargas% tar xf /var/tmp/ppc.tar && make
vargas% ./ppc-iriw-lwsync.exe
Test ppc-iriw-lwsync Allowed
Histogram (16 states)
180600:>1:r1=0; 1:r3=0; 3:r1=0; 3:r3=0;
3656  :>1:r1=1; 1:r3=0; 3:r1=0; 3:r3=0;
18812 :>1:r1=0; 1:r3=1; 3:r1=0; 3:r3=0;
77692 :>1:r1=1; 1:r3=1; 3:r1=0; 3:r3=0;
2973  :>1:r1=0; 1:r3=0; 3:r1=1; 3:r3=0;
9     *>1:r1=1; 1:r3=0; 3:r1=1; 3:r3=0;
28881 :>1:r1=0; 1:r3=1; 3:r1=1; 3:r3=0;
75126 :>1:r1=1; 1:r3=1; 3:r1=1; 3:r3=0;
20939 :>1:r1=0; 1:r3=0; 3:r1=0; 3:r3=1;
30498 :>1:r1=1; 1:r3=0; 3:r1=0; 3:r3=1;
1234  :>1:r1=0; 1:r3=1; 3:r1=0; 3:r3=1;
89993 :>1:r1=1; 1:r3=1; 3:r1=0; 3:r3=1;
75769 :>1:r1=0; 1:r3=0; 3:r1=1; 3:r3=1;
76361 :>1:r1=1; 1:r3=0; 3:r1=1; 3:r3=1;
87864 :>1:r1=0; 1:r3=1; 3:r1=1; 3:r3=1;
229593:>1:r1=1; 1:r3=1; 3:r1=1; 3:r3=1;
Ok

Witnesses
Positive: 9, Negative: 999991
Condition exists (1:r1=1 /\ 1:r3=0 /\ 3:r1=1 /\ 3:r3=0) is validated
Hash=4fbfaafa51f6784d699e9bdaf5ba047d
Observation ppc-iriw-lwsync Sometimes 9 999991
Time ppc-iriw-lwsync 0.68
\end{verbatim}


One may change the affinity increment with the command line option
\texttt{-i} of executable files. For instance, one binds the test threads
to logical processors $0$, $2$, $4$ and~$6$ as follows:
\begin{verbatim}
vargas% ./ppc-iriw-lwsync.exe -i 2 
Test ppc-iriw-lwsync Allowed
Histogram (15 states)
160629:>1:r1=0; 1:r3=0; 3:r1=0; 3:r3=0;
33389 :>1:r1=1; 1:r3=0; 3:r1=0; 3:r3=0;
43725 :>1:r1=0; 1:r3=1; 3:r1=0; 3:r3=0;
93114 :>1:r1=1; 1:r3=1; 3:r1=0; 3:r3=0;
33556 :>1:r1=0; 1:r3=0; 3:r1=1; 3:r3=0;
64875 :>1:r1=0; 1:r3=1; 3:r1=1; 3:r3=0;
34908 :>1:r1=1; 1:r3=1; 3:r1=1; 3:r3=0;
43770 :>1:r1=0; 1:r3=0; 3:r1=0; 3:r3=1;
64544 :>1:r1=1; 1:r3=0; 3:r1=0; 3:r3=1;
4     :>1:r1=0; 1:r3=1; 3:r1=0; 3:r3=1;
54633 :>1:r1=1; 1:r3=1; 3:r1=0; 3:r3=1;
92617 :>1:r1=0; 1:r3=0; 3:r1=1; 3:r3=1;
34754 :>1:r1=1; 1:r3=0; 3:r1=1; 3:r3=1;
54027 :>1:r1=0; 1:r3=1; 3:r1=1; 3:r3=1;
191455:>1:r1=1; 1:r3=1; 3:r1=1; 3:r3=1;
No

Witnesses
Positive: 0, Negative: 1000000
Condition exists (1:r1=1 /\ 1:r3=0 /\ 3:r1=1 /\ 3:r3=0) is NOT validated
Hash=4fbfaafa51f6784d699e9bdaf5ba047d
Observation ppc-iriw-lwsync Never 0 1000000
Time ppc-iriw-lwsync 0.92
\end{verbatim}
One observes that the non-SC outcome does not show up
with the new affinity setting.

One may also bind test thread to logical processors randomly with
executable option \texttt{+ra}.
\begin{verbatim}
vargas% ./ppc-iriw-lwsync.exe +ra
Test ppc-iriw-lwsync Allowed
Histogram (15 states)
...
No

Witnesses
Positive: 0, Negative: 1000000
Condition exists (1:r1=1 /\ 1:r3=0 /\ 3:r1=1 /\ 3:r3=0) is NOT validated
Hash=4fbfaafa51f6784d699e9bdaf5ba047d
Observation ppc-iriw-lwsync Never 0 1000000
Time ppc-iriw-lwsync 1.85
\end{verbatim}

As we see, the condition does not validate either with random affinity.
As a matter of fact, logical processors are taken at random in the
sequence $0$, $1$, \ldots, $63$;
while the successful run with \texttt{-i 1} took
them in the sequence $0$, $1$, $2$, $3$.
One can limit the sequence of  logical processor with option \texttt{-p},
which takes a sequence of logical processors numbers as argument:
\begin{verbatim}
vargas% ./ppc-iriw-lwsync.exe +ra -p 0,1,2,3
Test ppc-iriw-lwsync Allowed
Histogram (16 states)
...
8     *>1:r1=1; 1:r3=0; 3:r1=1; 3:r3=0;
...
Ok

Witnesses
Positive: 8, Negative: 999992
Condition exists (1:r1=1 /\ 1:r3=0 /\ 3:r1=1 /\ 3:r3=0) is validated
Hash=4fbfaafa51f6784d699e9bdaf5ba047d
Observation ppc-iriw-lwsync Sometimes 8 999992
Time ppc-iriw-lwsync 0.70
\end{verbatim}
The condition now validates.

\subsubsection{Study of affinity}
As illustrated by the previous example, both the running time and the outcomes
of a test are sensitive to affinity settings.
We measured running time for increasing values of the affinity increment
from $0$ (which disables affinity control)
to~$20$, producing the following figure:
\begin{center}\image[height=75ex]{m1l-time}\end{center}
As regards outcome variety,
we get all of the $16$ possible
outcomes only for an affinity increment of~$1$.

The differences in running times can be explained by reference to the mapping
of logical processors to hardware.
The machine~\texttt{vargas} consists in four MCM's (Multi-Chip-Module), each MCM
consists in four ``chips'', each chip consists in two cores, and
each core may support two logical processors.
As far as we know, by querying \texttt{vargas}
with the AIX commands
\texttt{lsattr}, \texttt{bindprocessor}
and \texttt{llstat},
the MCM's hold the logical processors
$0$--$15$, $16$--$31$, $32$--$47$ and~$48$--$63$,
each chip holds the logical processors $4k, 4k+1, 4k+2, 4k+3$
and each core holds the logical processors $2k, 2k+1$.

The measure of running times for varying increments
reveals two noticeable slowdowns:
from an increment of~$1$ to an increment of~$2$ and from $5$ to~$6$.
The gap between~$1$ and~$2$ reveals the benefits of
SMT for our testing application.
An increment of~$1$ yields both the greatest outcome
variety and the minimal running time.
The other gap may perhaps be explained by reference to MCM's:
for a value of~$5$ the tests runs on the logical processors
$0, 5, 10, 15$, all belonging to the same
MCM; while the next affinity increment of~$6$ results in
running the test on two different MCM ($0, 6, 12$ on the one hand
and $18$ on the other).


As a conclusion, affinity control provides users with a certain level
of control over thread placement, which is likely to yield faster tests when
threads are constrained to run on logical processors that are ``close'' one
to another.
The best results are obtained when SMT is effectively enforced.
However, affinity control is no panacea, and the memory system may
be stressed by other means, such as, for instance, allocating important
chunks of memory (option~\texttt{-s}).

\subsubsection{Advanced\label{affinity:advanced} control}
For specific experiments, the technique of
allocating logical processors sequentially by following a fixed increment
may be two rigid. \litmus{} offers a finer control on affinity by allowing
users to supply the logical processors sequence.
Notice that most users will probably not need this advanced feature.

Anyhow, so as to confirm that testing \afile{ppc-iriw-lwsync}
benefits from not crossing chip boundaries, one may wish to confine
its four threads to logical processors $16$ to~$19$,
that is to the first chip of the second MCM.
This can be done by overriding  the default logical processors sequence
by an user supplied one given as an argument to command-line
option~\texttt{-p}:
\begin{verbatim}
vargas% ./ppc-iriw-lwsync.exe -p 16,17,18,19 -i 1
Test ppc-iriw-lwsync Allowed
Histogram (16 states)
169420:>1:r1=0; 1:r3=0; 3:r1=0; 3:r3=0;
1287  :>1:r1=1; 1:r3=0; 3:r1=0; 3:r3=0;
17344 :>1:r1=0; 1:r3=1; 3:r1=0; 3:r3=0;
85329 :>1:r1=1; 1:r3=1; 3:r1=0; 3:r3=0;
1548  :>1:r1=0; 1:r3=0; 3:r1=1; 3:r3=0;
3     *>1:r1=1; 1:r3=0; 3:r1=1; 3:r3=0;
27014 :>1:r1=0; 1:r3=1; 3:r1=1; 3:r3=0;
75160 :>1:r1=1; 1:r3=1; 3:r1=1; 3:r3=0;
19828 :>1:r1=0; 1:r3=0; 3:r1=0; 3:r3=1;
29521 :>1:r1=1; 1:r3=0; 3:r1=0; 3:r3=1;
441   :>1:r1=0; 1:r3=1; 3:r1=0; 3:r3=1;
93878 :>1:r1=1; 1:r3=1; 3:r1=0; 3:r3=1;
81081 :>1:r1=0; 1:r3=0; 3:r1=1; 3:r3=1;
76701 :>1:r1=1; 1:r3=0; 3:r1=1; 3:r3=1;
93623 :>1:r1=0; 1:r3=1; 3:r1=1; 3:r3=1;
227822:>1:r1=1; 1:r3=1; 3:r1=1; 3:r3=1;
Ok

Witnesses
Positive: 3, Negative: 999997
Condition exists (1:r1=1 /\ 1:r3=0 /\ 3:r1=1 /\ 3:r3=0) is validated
Hash=4fbfaafa51f6784d699e9bdaf5ba047d
Observation ppc-iriw-lwsync Sometimes 3 999997
Time ppc-iriw-lwsync 0.63
\end{verbatim}
%$
Thus we get results similar to the previous experiment on logical processors
$0$ to~$3$ (option \texttt{-i 1} alone).

We may also run four simultaneous instances (\texttt{-n 4}, parameter~$n$ of
section~\ref{sec:arch}) of the test on
the four available MCM's:
\begin{verbatim}
vargas% ./ppc-iriw-lwsync.exe -p 0,1,2,3,16,17,18,19,32,33,34,35,48,49,50,51 -n 4 -i 1
Test ppc-iriw-lwsync Allowed
Histogram (16 states)
...
57    *>1:r1=1; 1:r3=0; 3:r1=1; 3:r3=0;
...
Ok

Witnesses
Positive: 57, Negative: 3999943
Condition exists (1:r1=1 /\ 1:r3=0 /\ 3:r1=1 /\ 3:r3=0) is validated
Hash=4fbfaafa51f6784d699e9bdaf5ba047d
Observation ppc-iriw-lwsync Sometimes 57 3999943
Time ppc-iriw-lwsync 0.75
\end{verbatim}
%$
Observe that, for a negligible penalty in running time, the number
of non-SC outcomes increases significantly.

By contrast, binding threads of a given instance of the test
to different MCM's results in poor running time and no non-SC outcome.
\begin{verbatim}
vargas% ./ppc-iriw-lwsync.exe -p 0,1,2,3,16,17,18,19,32,33,34,35,48,49,50,51 -n 4 -i 4
Test ppc-iriw-lwsync Allowed
Histogram (15 states)
...
Witnesses
Positive: 0, Negative: 4000000
Condition exists (0:r1=1 /\ 0:r3=0 /\ 2:r1=1 /\ 2:r3=0) is NOT validated
Time ppc-iriw-lwsync 1.48
\end{verbatim}
%$
In the experiment above, the increment is~$4$, hence the logical processors
allocated to the first
instance of the test are $0, 16, 32, 48$,
of which indices in the logical processors sequence are $0, 4, 8, 12$,
respectively.
The next allocated index in the sequence is~$12+4 = 16$.
However, the sequence has $16$ items.
Wrapping around yields index~$0$ which happens to be
the same as the starting index.
Then, so as to allocate fresh processors, the starting index is incremented
by one, resulting in allocating processors $1, 17, 33, 49$
(indices $1, 5, 9, 13$) to the second instance --- see section~\ref{incr:full}
for the full story.
Similarly, the third and fourth instances will get processors
$2, 18, 34, 50$ and $3, 19, 35, 51$, respectively.
Attentive readers may have noticed that the same experiment can
be performed with option \opt{-i 16} and no \opt{-p} option.

Finally, users should probably be aware that at least some versions of Linux
for x86 feature a less obvious mapping of logical processors to hardware.
On a bi-processor, dual-core, 2-ways hyper-threading, Linux,  AMD64 machine,
we have checked that logical processors residing on the same core
are $k$ and $k+4$, where $k$ is an arbitrary core number ranging from
$0$ to~$3$.
As a result, a proper choice for favouring effective hyper-threading
on such a machine is \opt{-i 4} (or \opt{-p 0,4,1,5,2,6,3,7 -i 1}).
More worthwhile noticing, perhaps,
the straightforward choice \opt{-i 1} disfavours effective hyper-threading\ldots

\subsubsection{Custom\label{affinity:custom} control}
Most tests run by \litmus{} are produced by the litmus test generators
described in Part~\ref{part:diy}.
Those tests include meta-information that may direct affinity control.
For instance we generate one test with the
\diyone{} tool, see Sec.~\ref{diyone:intro}.
More specifically we generate
\ahref{IRIW+lwsyncs.litmus}{\ltest{IRIW+lwsyncs}} for Power
(\ahref{ppc-iriw-lwsync.litmus}{\ltest{ppc-iriw-lwsync}} in the previous
section) as follows:
\begin{verbatim}
% diyone7 -arch PPC -name IRIW+lwsyncs Rfe LwSyncdRR Fre Rfe LwSyncdRR Fre
\end{verbatim}
We get the new source file~\afile{IRIW+lwsyncs.litmus}:
\verbatiminput{IRIW+lwsyncs.litmus}
The relevant meta-information  is the ``\texttt{Com}'' line that describes
how test threads are related --- for instance, thread~0 stores a value
to memory that is read by thread~1, written ``\texttt{Rf}'' (see Part~\ref{part:diy} for more details).
Custom affinity control will tend to run threads related by ``\texttt{Rf}''
on ``close'' logical processors, where we can for instance consider
that close logical processors belong to the same physical core (SMT for Power).
This minimal logical processor topology is described by two \litmus{}
command-line option: \texttt{-smt <n>} that specifies $n$-way SMT;
and \texttt{-smt\_mode (seq|end)} that specifies how logical processors
from the same core are numbered.
For a 8-cores 4-ways SMT power7 machine we invoke \litmus{} as follows:
\begin{verbatim}
% litmus7 -mem direct -smt 4 -smt_mode seq -affinity custom -o a.tar IRIW+lwsyncs.litmus
\end{verbatim}
Notice that memory mode is direct and that the number of available
logical processors is unspecified, resulting in running one instance of
the test. More importantly, notice that affinity control is enabled
\texttt{-affinity custom}, additionally specifying custom affinity mode.

We then upload the archive \texttt{a.tar} to our Power7 machine,
unpack, compile and run the test:
\begin{verbatim}
power7% tar xmf a.tar
power7% make
...
power7% ./IRIW+lwsyncs.exe -v
./IRIW+lwsyncs.exe  -v
IRIW+lwsyncs: n=1, r=1000, s=1000, +rm, +ca, p='0,1,2,3,4,5,6,7,8,9,10,11,12,13,14,15,16,17,18,19,20,21,22,23,24,25,26,27,28,29,30,31'
thread allocation: 
[23,22,3,2] {5,5,0,0}
\end{verbatim}
Option \texttt{-v} instructs the executable to show settings of the test harness: we see that one instance of the test is run (\texttt{n=1}),
size parameters are reminded (\texttt{r=1000, s=1000}) and
shuffling of indirect memory mode is performed (\texttt{+rm}).
Affinity settings are also given: mode is custom (\texttt{+ca}) and
the logical processor sequence inferred is given (\texttt{-p 0,1,\ldots,31}).
Additionally, the allocation  of test threads to logical processors is
given, as \texttt{[\ldots]}, as well as the allocation  of test threads
to physical cores, as \texttt{\{\ldots\}}.

Here is the run output proper:
\begin{verbatim}
Test IRIW+lwsyncs Allowed
Histogram (15 states)
2700  :>1:r1=0; 1:r3=0; 3:r1=0; 3:r3=0;
142   :>1:r1=1; 1:r3=0; 3:r1=0; 3:r3=0;
37110 :>1:r1=0; 1:r3=1; 3:r1=0; 3:r3=0;
181257:>1:r1=1; 1:r3=1; 3:r1=0; 3:r3=0;
78    :>1:r1=0; 1:r3=0; 3:r1=1; 3:r3=0;
15    *>1:r1=1; 1:r3=0; 3:r1=1; 3:r3=0;
103459:>1:r1=0; 1:r3=1; 3:r1=1; 3:r3=0;
149486:>1:r1=1; 1:r3=1; 3:r1=1; 3:r3=0;
30820 :>1:r1=0; 1:r3=0; 3:r1=0; 3:r3=1;
9837  :>1:r1=1; 1:r3=0; 3:r1=0; 3:r3=1;
2399  :>1:r1=1; 1:r3=1; 3:r1=0; 3:r3=1;
204629:>1:r1=0; 1:r3=0; 3:r1=1; 3:r3=1;
214700:>1:r1=1; 1:r3=0; 3:r1=1; 3:r3=1;
5186  :>1:r1=0; 1:r3=1; 3:r1=1; 3:r3=1;
58182 :>1:r1=1; 1:r3=1; 3:r1=1; 3:r3=1;
Ok

Witnesses
Positive: 15, Negative: 999985
Condition exists (1:r1=1 /\ 1:r3=0 /\ 3:r1=1 /\ 3:r3=0) is validated
Hash=836eb3085132d3cb06973469a08098df
Com=Rf Fr Rf Fr
Orig=Rfe LwSyncdRR Fre Rfe LwSyncdRR Fre
Affinity=[2, 3] [0, 1] ; (1,2) (3,0)
Observation IRIW+lwsyncs Sometimes 15 999985
Time IRIW+lwsyncs 0.70
\end{verbatim}
As we see, the test validates. Namely we observe the non-SC behaviour of
\ltest{IRIW} in spite of the presence of two \texttt{lwsync} barriers.
We may also notice, in the executable output some meta-information related
to affinity: it reads that threads 2 and~3 on the one hand and threads 0
and~1 on the other are considered ``close'' (\emph{i.e.} will run on the
same physical core); while threads 1 and~2 on the one hand and threads 3 and~0
on the other are considered ``far'' (\emph{i.e.} will run on different cores).


Custom affinity can be disabled by enabling another affinity mode.
For instance with \texttt{-i 0} we specify an affinity increment of zero.
That is, affinity control is disabled altogether:
\begin{verbatim}
power7% ./IRIW+lwsyncs.exe -i 0 -v
./IRIW+lwsyncs.exe -i 0  -v
IRIW+lwsyncs: n=1, r=1000, s=1000, +rm, i=0, p='0,1,2,3,4,5,6,7,8,9,10,11,12,13,14,15,16,17,18,19,20,21,22,23,24,25,26,27,28,29,30,31'
Test IRIW+lwsyncs Allowed
Histogram (15 states)
...
No

Witnesses
Positive: 0, Negative: 1000000
Condition exists (1:r1=1 /\ 1:r3=0 /\ 3:r1=1 /\ 3:r3=0) is NOT validated
Hash=836eb3085132d3cb06973469a08098df
Com=Rf Fr Rf Fr
Orig=Rfe LwSyncdRR Fre Rfe LwSyncdRR Fre
Observation IRIW+lwsyncs Never 0 1000000
Time IRIW+lwsyncs 0.90
\end{verbatim}
As we see, the test does not validate under those conditions.

Notice that section~\ref{affinity:experiment} describes a complete experiment
on affinity control.

\subsection{Controlling\label{exec:control} executable files}

\paragraph{Test conditions}
Any executable file produced by \litmus{} accepts the following
command line options.
\begin{description}
\item[{\tt -v}] Be verbose, can be repeated to increase verbosity.
Specifying \opt{-v} is a convenient way to look at the default of options.
\item[{\tt -q}] Be quiet.
\item[{\tt -a <n>}] Run maximal number of tests concurrently for $n$~available
logical processors --- parameter~$a$ in Sec.~\ref{defa}.
Notice that if affinity control is enabled (see below), \opt{-a 0} will
set parameter~$a$ to the number of logical processors effectively available. 
\item[{\tt -n <n>}] Run $n$~tests concurrently --- parameter~$n$
in Sec.~\ref{defn}. 
\item[{\tt -r <n>}] Perform $n$ runs --- parameter~$r$ in Sec.~\ref{defr}.
\item[{\tt -fr <f>}] Multiply~$r$ by~$f$ ($f$~is a floating point number).
\item[{\tt -s <n>}] Size of a run --- parameter~$s$ in Sec.~\ref{defs}.
\item[{\tt -fs <f>}] Multiply~$s$ by~$f$.
\item[{\tt -f <f>}] Multiply~$s$ by~$f$ and divide $r$ by~$f$.
\end{description}
\label{generalized:integer}Notice that options \opt{-s} and~\opt{-r}
accept a generalised
syntax for their integer argument: when suffixed by~\opt{k} (resp.~\opt{M})
the integer gets multiplied by~$10^3$ (resp.~$10^6$).

\label{rm}The following options are accepted only for tests compiled
in indirect memory mode (see Sec.~\ref{defmemorymode}):
\begin{description}
\item[{\tt -rm}] Do not shuffle pointer arrays, resulting a behaviour
similar do direct mode, without recompilation.
\item[{\tt +rm}] Shuffle pointer arrays, provided for regularity.
\end{description}

\label{st}The following option is accepted only for tests compiled
with a specified stride value (see Sec.~\ref{defstride}).
\begin{description}
\item[{\tt -st <n>}] Change stride to~\opt{<n>}.
The default stride is specified at compile time by \litmus{}
option \opt{-stride}.
\end{description}


The following option is accepted when enabled at compile time:
\begin{description}
\item[{\tt -l <n>}]
Insert the assembly code of each thread in a loop of size \opt{<n>}.
\end{description}

\paragraph{\aname{affinity:runopt}{Affinity}}
If affinity control has been enabled at compilation time
(for instance, by supplying option \texttt{-affinity incr1}
to \litmus),
the  executable file produced by \litmus{} accepts the following
command line options.
\begin{description}
\item[{\tt -p <ns>}] Logical processors sequence.
The sequence \texttt{<ns>} is a comma separated list of integers,
The default sequence is inferred by the executable as $0,1,\ldots,A-1$,
where $A$ is the number of logical processors featured by the tested machine;
or is a sequence specified at compile time
with \litmus{} option~\texttt{-p}.
\item[{\tt -i <n>}] Increment for allocating logical processors to threads.
Default is specified at compile time by \litmus{} option~\texttt{-affinity
incr<\textit{n}>}.
Notice that \texttt{-i 0} disable affinity
and that \texttt{.exe} files reject the \texttt{-i} option when affinity
control has not been enabled at compile~time.
\item[{\tt +ra}] Perform random allocation of affinity at each test round.
\item[{\tt +ca}] Perform custom affinity.
\end{description}
Notice that when custom affinity is not available, would it be that
the test source lacked meta-information or that logical processor
topology was not specified at compile-time, then \texttt{+ca}
behaves as \texttt{+ra}.

\label{incr:full}Logical processors are allocated
test instance by test instance
(parameter~$n$ of Sec.~\ref{sec:arch}) and then
thread by thread, scanning the logical processor sequence
left-to-right by steps of the given increment.
More precisely, assume a logical processor sequence
$P = p_0, p_1, \ldots, p_{A-1}$ and an increment~$i$.
The first processor allocated is $p_0$, then $p_i$, then $p_{2i}$ etc,
Indices in the sequence~$P$ are reduced modulo $A$ so as to wrap around.
The starting index of the allocation sequence (initially $0$) is recorded,
and coincidence with the index of the next processor to be 
allocated is checked.
When coincidence occurs, a new index is computed, as the previous
starting index plus one, which also becomes the new starting index.
Allocation then proceeds from this new starting index.
That way, all the processors in the sequence
will get allocated to different threads naturally, provided of
course that less than $A$~threads are scheduled to run.
See section~\ref{affinity:advanced} for an example with $A=16$ and~$i=4$.

\section{Advanced\label{advanced:control} control of test parameters}

\subsection{Timebase \label{timebase}synchronisation mode}
Timebase synchronisation of the testing loop iterations
(see Sec.~\ref{timebase:intro}) is selected by \litmus{} command line option
\opt{-barrier timebase}.
In that mode,
test threads will first synchronise using polling synchronisation
barrier code, agree on a target timebase value and then loop
reading the timebase until it exceeds the target value.
Some tests demonstrate that timebase synchronisation
is more precise than user synchronisation (\opt{-barrier user} and default).

For instance, consider the x86 test \atest{6.SB},
a 6-thread analog of the \ahrefloc{SB}{\ltest{SB}}~test:
\verbatiminput{6.SB.litmus}
As for \atest{SB}, the final condition of
\atest{6.SB} identifies executions where each thread loads the initial
value~$0$ of a location that is writtent into by another thread.
\begin{center}\img{6.SB}\end{center}

We first compile the test in user synchronisation mode, saving
\litmus{} output files into the directory~\file{R}:
\begin{verbatim}
% mkdir -p R
% litmus7 -barrier user -vb true -o R 6.SB.litmus
% cd R
% make
\end{verbatim}
The additional command line option \opt{-vb true} activates the printing
of some timing information on synchronisations.

We then directly run the test executable \file{6.SB.exe}:
\begin{verbatim}
% ./6.SB.exe
Test 6.SB Allowed
Histogram (62 states)
7569  :>0:EAX=1; 1:EAX=0; 2:EAX=0; 3:EAX=0; 4:EAX=0; 5:EAX=0;
8672  :>0:EAX=0; 1:EAX=1; 2:EAX=0; 3:EAX=0; 4:EAX=0; 5:EAX=0;
...
326   :>0:EAX=1; 1:EAX=0; 2:EAX=1; 3:EAX=1; 4:EAX=1; 5:EAX=1;
907   :>0:EAX=0; 1:EAX=1; 2:EAX=1; 3:EAX=1; 4:EAX=1; 5:EAX=1;
No

Witnesses
Positive: 0, Negative: 1000000
Condition exists (0:EAX=0 /\ 1:EAX=0 /\ 2:EAX=0 /\ 3:EAX=0 /\ 4:EAX=0 /\ 5:EAX=0) is NOT validated
Hash=107f1303932972b3abace3ee4027408e
Observation 6.SB Never 0 1000000
Time 6.SB 0.85
\end{verbatim}
The targeted outcome --- reading zero in the \verb+EAX+ registers
of the $6$ threads --- is not observed.
We can observe synchronisation times for all tests runs
with the executable  command line option \opt{+vb}:
\begin{verbatim}
% ./6.SB.exe  +vb
99999: 162768 420978 564546   -894 669468
99998:    -93      3     81   -174   -651
99997:   -975    -30    -33     93   -192
99996:    990   1098    852   1176    774
...
\end{verbatim}
We see five columns of numbers that list, for each test run,
the starting delays of \P{1}, \P{2} etc. with respect to \P{0}, expressed
in timebase ticks. Obviously, synchronisation is rather loose,
there are always two threads whose starting delays differ of
about $1000$~ticks.

We now compile the same test in timebase synchronisation mode,
saving \litmus{} output files into the pre-existing directory~\file{RT}:
\begin{verbatim}
% mkdir -p RT
% litmus7 -barrier timebase -vb true -o RT 6.SB.litmus
% cd RT
% make
\end{verbatim}
And we run the test directly
(option \opt{-vb} disable the printing of any
synchronisation timing information):
\begin{verbatim}
% ./6.SB.exe -vb
Test 6.SB Allowed
Histogram (64 states)
60922 *>0:EAX=0; 1:EAX=0; 2:EAX=0; 3:EAX=0; 4:EAX=0; 5:EAX=0;
38299 :>0:EAX=1; 1:EAX=0; 2:EAX=0; 3:EAX=0; 4:EAX=0; 5:EAX=0;
...
598   :>0:EAX=0; 1:EAX=1; 2:EAX=1; 3:EAX=1; 4:EAX=1; 5:EAX=1;
142   :>0:EAX=1; 1:EAX=1; 2:EAX=1; 3:EAX=1; 4:EAX=1; 5:EAX=1;
Ok

Witnesses
Positive: 60922, Negative: 939078
Condition exists (0:EAX=0 /\ 1:EAX=0 /\ 2:EAX=0 /\ 3:EAX=0 /\ 4:EAX=0 /\ 5:EAX=0) is validated
Hash=107f1303932972b3abace3ee4027408e
Observation 6.SB Sometimes 60922 939078
Time 6.SB 1.62
\end{verbatim}
We now see that the test validates. Moreover
all of the $64$~possible outcomes are observed.

Timebase synchronisation works as follows: at every iteration,
\begin{enumerate}
\item one of the threads reads timebase~$T$;
\item all threads synchronise by the means of a polling synchronisation
barrier;
\item each thread computes $T_i = T + \delta_i$, where $\delta_i$ is
\emph{the timebase delay}, a thread
specific constant;
\item each thread loops, reading the timebase until the read value exceeds
$T_i$.
\end{enumerate}
By default the timebase delay $\delta_i$ is $2^{11} = 2048$ for all threads.

The precision of timebase synchronisation can be illustrated
by enabling the printing of all synchronisation timings:
\begin{verbatim}
% ./6.SB.exe +vb
99999: 672294[1] 671973[1] 672375[1] 672144[1] 672303[1] 672222[1]
99998:  4524[1]  4332[1]  4446[1]  2052[65]  2064[73]  4095[1]
...
99983:  4314[1]  3036[1]  3141[1]  2769[1]  4551[1]  3243[1]
99982:* 2061[36]  2064[33]  2067[11]  2079[12]  2064[14]  2064[24]
99981:  2121[1]  2382[1]  2586[1]  2643[1]  2502[1]  2592[1]
...
\end{verbatim}
For each test iteration and each thread, two numbers are shown (1) the last
timebase value read by and~(2) (in brackets \verb+[+\ldots\verb+]+) how many iterations of loop~4. were performed.
Additionally a star ``\verb+*+'' indicates the occurrence
of the targeted outcome.
Here, we see that a nearly perfect synchronisation can be achieved
(cf. line \verb+99982:+ above).


Once timebase synchronisation have been selected
(\litmus{} option \opt{-barrier timebase}),
test executable behaviour can be altered by the following
two command line options:
\begin{description}
\item[{\tt -ta <n>}]
Change the timebase delay~$\delta_i$ of all threads.
\item[{\tt -tb <0:n$_0$;1:$n_1$;$\cdots$>}]
Change the timebase delay~$\delta_i$ of individual threads.
\end{description}

The \litmus{} command line option~\opt{-vb true}
(verbose barrier)
governs the printing of synchronisation timings. It comes handy when
choosing values for the \opt{-ta} and \opt{-tb} options.
When set, the executable show synchronisation timings
for outcomes that validate the test final condition.
This default behaviour can be altered with
the following two command line options:
\begin{description}
\item[{\tt -vb}] Do not show synchronisation timings.
\item[{\tt +vb}] Show synchronisation timings for all outcomes.
\end{description}
Synchronisation timings are expressed in timebase ticks.
The format depends on the synchronisation mode
(\litmus{} option \opt{-barrier}).
This section just gave two examples for user mode
(timings are show as differences from thread~\P{0}); and for
timebase mode (timings are shown as differences
from a commonly agreed by all thread timebase value).
Notice that, when affinity control is enabled,
the running logical processors of threads are also shown.

\subsection{Advanced \label{preload:custom}prefetch control}
Supplying the tags
\opt{custom}, \opt{static}, \opt{static1} or~\opt{static2} to
\litmus{} command line option \opt{-preload}
commands the insertion of cache prefetch or flush
instructions before every test instance.

In custom mode the execution of such cache management instruction
is under total user control, the other, ``static'', modes offer
less control to the user, for the sake of not altering test code proper.

\subsubsection{Custom prefetch}
Custom prefetch mode offers complete control over
cache management instructions.
Users enable this mode by supplying the command line option
\opt{-preload custom} to \litmus{}. For instance one may compile
the x86 test \afile{6.SB.litmus} as follows:
\begin{verbatim}
% mkdir -p R
% litmus7 -mem indirect -preload custom -o R 6.SB.litmus
% cd R
% make
\end{verbatim}
Notice the test is compiled in \ahrefloc{defmemorymode}{indirect memory mode},
in order to reduce false sharing effects.

The executable \texttt{6.SB.exe} accepts two new command line options:
\opt{-prf} and~\opt{-pra}. Those options takes arguments that describe
cache management instructions.
The option \opt{-pra} takes one letter that stands for a cache
management instruction as we here describe:
\begin{center}
\texttt{I}: do nothing,
\texttt{F}: cache flush,
\texttt{T}: cache touch,
\texttt{W}: cache touch for a write.
\end{center}
All those cache management instructions are not provided by all architectures,
in case some instruction is missing, the letters behave as follows:
\begin{center}
\texttt{F}: do nothing,
\texttt{T}: do nothing,
\texttt{W}: behave as \texttt{T}.
\end{center}

With \opt{-pra $X$} the commanded action applies to all
threads and all variables, for instance:
\begin{verbatim}
% ./6.SB.exe -pra T
\end{verbatim}
will perform a run where every test thread touches the test locations
that it refers to
(\emph{i.e.} \texttt{x} and~\texttt{y} for Thread~0, \texttt{y}
and \texttt{z} for Thread~1, etc.)
before executing test code proper.
Although one may achieve interesting results by using
this \opt{-pra} option, the more selective
\opt{-prf} option should prove more useful.
The \opt{-prf} option takes a comma separated list of cache managment
directives.
A cache management directive is $n$:\textit{loc}=$X$,
where $n$ is a thread number, \textit{loc} is a program variable,
and $X$ is a cache management controle letter.
For instance, \texttt{-prf 0:y=T} instructs
thread~0 to touch location~\texttt{y}.
More generally, having each thread of the test
\ahref{6.SB.litmus}{\ltest{6.SB}} to touch the memory location
it reads with its second instruction would favor reading the initial value
of these locations,
and thus validating
the final condition of the test 
``\verb+(0:EAX=0 /\ 1:EAX=0 /\ 2:EAX=0 /\ 3:EAX=0 /\ 4:EAX=0 /\ 5:EAX=0)+''.

Notice that those locations can be found by looking
at the \ahref{6.SB.litmus}{test code}
or at the \ahref{6.SB.png}{diagram} of the target execution.
Let us have a try:
\begin{verbatim}
./6.SB.exe -prf 0:y=T,1:z=T,2:a=T,3:b=T,4:c=T,5:x=T               
Test 6.SB Allowed
Histogram (63 states)
10    *>0:EAX=0; 1:EAX=0; 2:EAX=0; 3:EAX=0; 4:EAX=0; 5:EAX=0;
...
Witnesses
Positive: 10, Negative: 999990
...
Prefetch=0:y=T,1:z=T,2:a=T,3:b=T,4:c=T,5:x=T
...
\end{verbatim}
As can be seen, the final condition is validated. Also notice
that the prefetch directives used during the run are reminded.
If given several times, \opt{-prf} options cumulate,
the rightmost directives taking precedence in case of ambiguity.
As a consequence, one may achieve the same prefetching
effect as above with:
\begin{verbatim}
% ./6.SB.exe -prf 0:y=T -prf 1:z=T -prf 2:a=T -prf 3:b=T -prf 4:c=T -prf 5:x=T
\end{verbatim}


\subsubsection{Prefetch metadata}
The source code of tests may include prefetch directives as metadata
prefixed with ``\verb+Prefetch=+''.
In particular, the generators of the \diy{}
suite  (see Part~\ref{part:diy}) produce such metadata.
For instance in the case of the
\ltest{6.SB} test (generated source \afile{6.SB+Prefetch.litmus}),
this metadata reads:
\begin{verbatim}
Prefetch=0:x=F,0:y=T,1:y=F,1:z=T,2:z=F,2:a=T,3:a=F,3:b=T,4:b=F,4:c=T,5:c=F,5:x=T
\end{verbatim}
That is, each thread flushes the location it stores to and touches
each location it reads from.
Notice that each thread starts with a memory location access
(here a store) and ends with another (here a load).
The idea simply is to accelerate the exit access (with a cache touch)
while delaying the entry access (with a cache flush).


When prefetch metadata is available, it acts as the default of
prefetch directives:
\begin{verbatim}
% litmus7 -mem indirect -preload custom -o R 6.SB+Prefetch.litmus
% cd R
% make
\end{verbatim}
Then we run the test by:
\begin{verbatim}
% ./6.SB+Prefetch.exe
Test 6.SB Allowed
Histogram (63 states)
674   *>0:EAX=0; 1:EAX=0; 2:EAX=0; 3:EAX=0; 4:EAX=0; 5:EAX=0;
...
Witnesses
Positive: 674, Negative: 999326
...
Prefetch=0:x=F,0:y=T,1:y=F,1:z=T,2:a=T,2:z=F,3:a=F,3:b=T,4:b=F,4:c=T,5:c=F,5:x=T
...
\end{verbatim}
One may notice that the prefetch directives from the source file
medata found its way to the test executable.

As with any kind of metadata,
one can change the prefetch metadata by editing the litmus source file,
or better by using the \opt{-hints} command line option.
The  \opt{-hints} command line option takes a filename as  argument.
This file is a \ahrefloc{defmapping}{mapping} that associates
new metadata to test names.
As an example, we reverse \diy{} scheme for cache management directives:
accelerating entry accesses and delaying exit accesses:
\begin{verbatim}
% cat map.txt
6.SB Prefetch=0:x=W,0:y=F,1:y=W,1:z=F,2:a=F,2:z=W,3:a=W,3:b=F,4:b=W,4:c=F,5:c=W,5:x=F
% litmus7 -mem indirect -preload custom -hints map.txt -o R 6.SB.litmus
% cd R
% make 
...
% ./6.SB.exe 
Test 6.SB Allowed
Histogram (63 states)
24    *>0:EAX=0; 1:EAX=0; 2:EAX=0; 3:EAX=0; 4:EAX=0; 5:EAX=0;
...
Prefetch=0:x=W,0:y=F,1:y=W,1:z=F,2:a=F,2:z=W,3:a=W,3:b=F,4:b=W,4:c=F,5:c=W,5:x=F
...
\end{verbatim}
As we see above, the final condition validates.
It does so in spite of
the apparently unfavourable cache management directives.

We can experiment further without recompilation, by
using the \opt{-pra} and~\opt{-prf} command line options of
the test executable. Those are parsed left-to-right, so that we can
(1) cancel any default cache management directive with
\opt{-pra I} and~(2) enable cache touch for the stores:
\begin{verbatim}
% ./6.SB.exe -pra I -prf 0:x=W -prf 1:y=W -prf 2:z=W -prf 3:a=W -prf 4:b=W -prf 5:c=W
Test 6.SB Allowed
...
Witnesses
Positive: 0, Negative: 1000000
...
Prefetch=0:x=W,1:y=W,2:z=W,3:a=W,4:b=W,5:c=W
\end{verbatim}
As we see, the final condition does not validate.

By contrast, flushing or touching the locations that the threads load permit to
repetitively achieve validation:
\begin{verbatim}
chi% ./6.SB.exe -pra I -prf 0:y=F -prf 1:z=F -prf 2:a=F -prf 3:b=F -prf 4:c=F -prf 5:x=F
Test 6.SB Allowed
Histogram (63 states)
211   *>0:EAX=0; 1:EAX=0; 2:EAX=0; 3:EAX=0; 4:EAX=0; 5:EAX=0;
...
% ./6.SB.exe -pra I -prf 0:y=T -prf 1:z=T -prf 2:a=T -prf 3:b=T -prf 4:c=T -prf 5:x=T
Test 6.SB Allowed
Histogram (63 states)
10    *>0:EAX=0; 1:EAX=0; 2:EAX=0; 3:EAX=0; 4:EAX=0; 5:EAX=0;
...
\end{verbatim}
As a conclusion, interpreting the impact of cache management
directives is not easy. However, custom preload mode
(litmus command line option \opt{-preload custom}) and test executable
options \opt{-pra} and~\opt{-prf} allow experimentation on specific tests.


\subsubsection{``Static'' prefetch control}
Custom prefetch mode comes handy when one wants to tailor cache management
directives for a particular test.
In practice, we run batches of tests using source metadata for
prefetch directives.
In such a setting, the code that interprets the
prefetch directives is useless,
as we do not use the \opt{-prf} option of the test executables.
As this code get executed before each test thread code, it may impact test
results.
It is desirable to supress this code from test executables, still
performing cache management instructions.
To that aim, \litmus{} provides some ``static'' preload modes, enabled
with command line options \opt{-preload static},
\opt{-preload static1} and~\opt{-preload static2}.

In the former mode  \opt{-preload static} and without any further user
intervention, each test thread executes the cache management
instructions commanded by the \verb+Prefetch+ metadata:
\begin{verbatim}
% mkdir -p S
% litmus7 -mem indirect -preload static -o R 6.SB+Prefetch.litmus
% make -C S
% S/6.SB+Prefetch.exe
Test 6.SB Allowed
Histogram (63 states)
804   *>0:EAX=0; 1:EAX=0; 2:EAX=0; 3:EAX=0; 4:EAX=0; 5:EAX=0;
...
Observation 804 999196
...
\end{verbatim}
As we can see above, the effect of the  cache management
instructions looks more favorable than in custom preload mode.

Users still have a limited control on the
execution of cache management instructions: produced executable accept
a new \opt{-prs <n>} option, which take a positive or null integer as argument.
Then, each test thread executes the cache
management instructions commanded by source metadata with probability~$1/n$,
the special value $n=0$ disabling prefetch altogether.
The default for the \opt{-prs} options is ``\opt{1}'' (always execute
the cache management instructions).
Let us try:
\begin{verbatim}
% S/6.SB+Prefetch.exe -prs 0 | grep Observation
Observation 6.SB Never 0 1000000
% S/6.SB+Prefetch.exe -prs 1 | grep Observation
Observation 6.SB Sometimes 901 999099
% S/6.SB+Prefetch.exe -prs 2 | grep Observation
Observation 6.SB Sometimes 29 999971
% S/6.SB+Prefetch.exe -prs 3 | grep Observation
Observation 6.SB Sometimes 16 999984
\end{verbatim}
In those experiments we show the ``\verb+Observation+'' field of \litmus{}
output: this field gives the count of outcomes that validate the final
condition, followed by the count of outcomes that do not validate
the final condition. The above counts confirm that
cache management instructions favor validation.

The remaining preload modes \opt{static1} and~\opt{static2} are similar,
except that they produce executable files that do not accept
the \opt{-prs} option. Furthermore, in the former
mode \opt{-preload static1} cache management instructions are always executed,
while in the latter mode \opt{-preload static2}
cache management instructions are  executed with probability~$1/2$.
Those modes thus act as pure static mode
(\litmus{} option \opt{-preload static}),
with runtime options \opt{-prs 1} and~\opt{-prs 2} respectively.
Moreover,
as the test scaffold includes no code to interpret the \opt{-prs <n>}
switch, the test code is less perturbed.
In practice and for the \ltest{6.SB} example, there is little difference:
\begin{verbatim}
% mkdir -p S1 S2
% litmus7 -mem indirect -preload static1 -o S1 6.SB+Prefetch.litmus
% litmus7 -mem indirect -preload static2 -o S2 6.SB+Prefetch.litmus
% make -C S1 && make -C S2
...
% S1/6.SB+Prefetch.exe | grep Observation
Observation 6.SB Sometimes 1119 998881
% S2/6.SB+Prefetch.exe | grep Observation
Observation 6.SB Sometimes 16 999984
\end{verbatim}

\section{Usage of \litmus{}}

\subsection*{Arguments}
\litmus{} takes file names as command line arguments.
Those files are either a single litmus test,
when having extension \file{.litmus},
or a list of file names, when prefixed by \file{@}.
Of course, the file names in \file{@}files can themselves be
\file{@}files.

\subsection*{Options}
There are many command line options.
We describe the more useful ones:

\paragraph*{General behaviour}
\begin{description}
\item[{\tt -version}] Show version number and exit.
\item[{\tt -libdir}] Show installation directory and exit.
\item[{\tt -v}] Be verbose, can be repeated to increase verbosity.
\item[{\tt -mach <name>}] Read configuration file~\file{name.cfg}.
See the \ahrefloc{config:sec}{next section}
for the syntax of configuration files.
\item[{\tt -o <dest>}]
Save C-source of test files into \opt{<dest>} instead of running them.
If  argument \opt{<dest>} is an archive (extension \opt{.tar})
or a compressed archive (extension \opt{.tgz}),
\litmus{} builds an archive: this is the ``cross~compilation feature''
demonstrated in Sec.~\ref{cross}.
Otherwise, \opt{<dest>} is interpreted as the name of an
existing directory and tests are saved in it.
\item[{\tt -driver (shell|C|XCode)}]
Choose the driver that will run the tests.
In the ``\opt{shell}'' (and default) mode,
each test will be compiled into an executable. A dedicated shell script
\file{run.sh} will launch the test executables.
In the ``\opt{C}'' mode, one executable \file{run.exe} is produced, which
will launch the tests.
%(see Sec.~\ref{driverc:example} for an example).
Finally, the \opt{XCode} mode is for inclusion of the tests into
a dedicated iOS App, which we do not distribute at the moment.
\item[{\tt -crossrun <(user@)?host(:port)?|adb|qemu>}]
When the shell driver is used (\opt{-driver shell} above),
the first two possible arguments \aname{crossrun}{instruct}
the \texttt{run.sh} script to run individual tests on a remote machine.
The remote host can be contacted by the means of \texttt{ssh}
or the Android Debug Bridge.
\begin{description}
\item[\tt ssh]
\opt{user} is a login name on the the remote host,
\opt{<host>} is the name of the remote host,
and \texttt{port} is a port-number which can be omitted when standard~(22).
\item[\tt adb]
Tests will be run in the remote directory \opt{/data/tmp}.
\end{description}
This option may be useful when the tested machine has little disk space or a
crippled installation.
Default is disabled --- \emph{i.e.} run tests on
the machine where the \file{run.sh} script runs.

The argument \opt{qemu} instruct the shell script to use the \texttt{qemu}
emulator. The \texttt{QEMU} environment variable must be defined to the
emulator path before running the \texttt{run.sh} script, as in (\texttt{sh} syntax): \verb+QEMU=qemu sh run.sh+.


\item[{\tt -index <@name>}] Save the source names of compiled files in index
file~\file{@name}.
\end{description}

\paragraph*{Test conditions\label{litmus:option:sec}}
The following options set the default values of
the options of the executable files produced:
\begin{description}
\item[\aname{avail}{{\tt -a <n>}}]Run maximal number of tests concurrently for $n$~available
logical processors ---
set default value for \opt{-a} of Sec.~\ref{exec:control}.
Default is~$1$ (run one test).
When affinity control is enabled, the value~$0$ has the special
meaning of having executables to set the
number of available logical processors according to
how many are actually present.
\item[{\tt -limit <bool>}] Do not process tests with more than~$n$
threads, where $n$ is the number of available cores defined above.
Default is \opt{true}.
\item[{\tt -r <n>}] Perform $n$ runs --- set default value for
option~\opt{-r} of Sec.~\ref{exec:control}.
The option accepts generalised syntax for integers and
default is~$10$.
\item[\aname{sizeoftest}{{\tt -s <n>}}]Size of a run --- set default value for
option \opt{-s} of Sec.~\ref{exec:control}.
The option accepts generalised syntax for integers and
default is $100000$ (or \opt{100k}).
\end{description}


The following additional options control the various modes described
in Sec.~\ref{sec:arch}, and more.
Those cannot be changed without running \litmus{} again:
\begin{description}
\item[{\tt -barrier (user|userfence|pthread|none|timebase)}] Set synchronisation mode, default \opt{user}. Synchronisation modes are described
in Sec.~\ref{defsynchronisation}
\item[{\tt -launch (changing|fixed)}] Set launch mode,
default \opt{changing}.
\item[{\tt -mem (indirect|direct)}] Set memory mode,
default \opt{indirect}.
It is possible to instruct executables compiled in indirect mode
to behave almost as if compiled in direct mode, see Sec.~\ref{rm}.
\item[{\tt -stride <n>}]
Specify a stride value of \opt{<n>} --- set default value for option \opt{-st} of Sec.~\ref{st}. See Sec.~\ref{defstride} for details on the stride parameter.
If <n> is negative or zero, restore the default,
which is stride feature disabled.
\item[{\tt -st <n>}] Alias for \opt{-stride <n>}.
\item[{\tt -para (self|shell)}]
Perform several tests concurrently, either by forking POSIX
threads (as described in Sec.~\ref{sec:arch}), or by forking
Unix processes. Only applies for cross compilation.
Default is \opt{self}.
\item[{\tt -alloc (dynamic|static|before)}]
Set memory allocation mode. In ``dynamic'' and ``before'' modes, the memory
used by test threads is allocated with \texttt{malloc} --- in ``before'' mode,
memory is allocated before forking test instances.
In ``static'' mode, the memory is pre-allocated as static arrays.
In that latter case, the size of allocated arrays depend upon
compile time defined parameters: the number of available logical processors
(see option \ahrefloc{avail}{\opt{-a <n>}})
and the size of a run (see option \ahrefloc{sizeoftest}{\opt{-s <n>}}).
It remains possible to change those those at execution time, provided
the resulting memory size does not exceed the compile time value.
Default is \texttt{dynamic}.

\item[{\tt -preload (no|random|custom|static|static1|static2)}]
Specify preload mode (see Sec.~\ref{defpreload}), default is \texttt{random}.
Starting from version 5.0 we provide additional ``custom'' and ``static''
modes for a finer control of prefetching and flushing of some memory locations
by some threads. See Sec~\ref{preload:custom}.
\item[{\tt -safer (no|all|write)}] Specify safer mode,
default is \texttt{write}.
When instructed to do so, executable files perform some consistency checks.
Those are intended both for debugging and for dynamically checking
some assumptions on POSIX threads that we rely upon.
More specifically the test harness checks for the stabilisation of
memory locations after a test round in the ``\texttt{all}'' and
``\texttt{write}'' mode, while
the initial values of memory locations are checked in ``\texttt{all}'' mode.
\item[{\tt -speedcheck (no|some|all)}] 
Quick condition check mode,
default is ``\texttt{no}''.
In mode ``\texttt{some}'', test executable will stop as soon as its
condition is settled.
In mode ``\texttt{all}'', the \texttt{run.sh} script will additionally
not run the test if invoked once more later.
\end{description}

\aname{affinity:control}{The} following optiondra commands affinity control:
\begin{description}
\item[{\tt -affinity (none|incr<n>|random|custom)}]
Enable (of disable with tag \opt{none}) affinity control,
specifying default affinity mode of executables.
Default is \opt{none}, \emph{i.e.} executables do not
include affinity control code.
The various tags are interpreted as follows:
\begin{enumerate}
\item \opt{incr<n>}:
integer \opt{<n>} is the increment  for allocating logical
processors to threads --- see Sec.~\ref{sec:affinity}.
Notice that with \opt{-affinity incr0}
the produced code features affinity control, which executable
files do not exercise by default.
\item \opt{random}: executables perform random allocation of
test threads to logical processors.
\item \opt{custom}: executables perform custom allocation of
test threads to logical processors.
\end{enumerate}
Notice that the default for executables can be overridden using
options \opt{-i},\opt{+ra} and~\opt{+ca}
of Sec.~\ref{exec:control}.

\item[{\tt -i <n>}] Alias for \opt{-affinity incr<n>}.
\end{description}
Notice that affinity control is not implemented for MacOs.

The following options are significant when affinity control is enabled.
Otherwise they are silent no-ops.
\begin{description}
\item[{\tt -p <ns>}]
Specify the sequence of logical processors.
The notation \opt{<ns>} stands for a comma separated list of integers.
Set default value for option \opt{-p} of Sec.~\ref{exec:control}.
Default for this \opt{-p} option
will let executable files compute the logical processor sequence
themselves.
\item[{\tt -force\_afffinity <bool>}]
Code that sets affinity will spin until all specified
cores (as given with option \opt{-avail <n>}) processors
are up. This option is necessary on devices that let core sleep
when the computing load is low. Default is false.
\end{description}

Custom affinity control (see Sec.~\ref{affinity:custom}) is enabled,
first by enabling affinity control (\emph{e.g.} with \opt{-affinity \ldots}),
and then by specifying a logical processor topology with options \texttt{-smt}
and \texttt{-smt\_mode}.
\begin{description}
\item[{\tt -smt <n>}] Specify that logical processors are close by groups
of $n$, default is \texttt{1}.
\item[{\tt -smt\_mode (none|seq|end)}] Specify how ``close'' logical processors
are numbered, default is \texttt{none}.
In mode~``\texttt{end}'', logical processors of the same core
are numbered as $c$, $c+A_c$ etc. where $c$ is a physical core number and
$A_c$ is the number of physical cores available.
In mode~``\texttt{seq}'', logical processors of the same core
are numbered in sequence.
\end{description}
Notice that custom affinity works only for those tests that include the proper
meta-information. Otherwise, custom affinity silently degrades
to random affinity.

Finally, a few miscellaneous options are documented:
\begin{description}
\item[{\tt -l <n>}]
Insert the assembly code of each thread in test in a loop of size \opt{<n>}.
Accepts generalised integer syntax, disabled by default.
Sets default value for option \opt{-l} of Sec.~\ref{exec:control}.

This feature may prove useful for measuring running times that are not
too much perturbed by the test harness, in combination
with options~\opt{-s 1 -r 1}.
\item[{\tt -vb <bool>}]
Disable/enable the printing of synchronisation timings, default is \opt{false}.

This feature may prove useful for analysing the synchronisation behaviour of
a specific test, see Sec.~\ref{timebase}.
\item[{\tt -ccopts <flags>}] Set \prog{gcc} compilation flags
(defaults: X86=\opt{"-fomit-frame-pointer -O2"}, PPC/ARM=\opt{"-O2"}).
\item[\aname{gcc}{{\tt -gcc <name>}}]
Change the name of C compiler, default \opt{gcc}.
\item[\aname{linkopt}{{\tt -linkopt <flags>}}] Set \prog{gcc} linking flags.
(default: void).
\item[\aname{gas}{{\tt -gas <bool>}}]
Emit Gnu as extensions  (default Linux/Mac=\opt{true}, AIX=\opt{false})
\end{description}

\paragraph*{Target architecture description}
Litmus compilation chain may slightly vary depending on the following
parameters:
\begin{description}
\item[{\tt -os (linux|mac|aix)}] Set target operating system.
This parameter mostly impacts some of \prog{gcc} options. Default \opt{linux}.
\item[{\tt -ws (w32|w64)}] Set word size.
This option first selects \prog{gcc} $32$ or~$64$ bits mode,
by providing it with the appropriate option (\texttt{-m32}
or \texttt{-m64} on linux, \texttt{-maix32}
or \texttt{-maix64} on AIX).
It also slightly impacts code generation in the corner case
where memory locations hold other memory locations.
Default is a bit contrived: it acts as \opt{w32} as regards code
generation, while it provides no $32$/$64$ bits mode selection option
to~\texttt{gcc}.
\end{description}

\paragraph*{Change\label{change:input} input}
Some items in
the source of tests can be changed at the very last moment.
\aname{defmapping}{The}
new items are defined in mapping files whose names are arguments to
the appropriate command line options.
Mapping files simply are lists of pairs, with one line starting with a test
name, and the rest of line defining the changed item.
The changed item may also contains several lines: in that case it should be
included in double quotes~``\texttt{"}.''.
\begin{description}
\item[{\tt -names <file>}] Run \litmus{} only on tests whose names are
listed in \texttt{<file>}.
\item[{\tt -rename <file>}] Change test names.
\item[{\tt -kinds <file>}] Change test kinds.
This amonts to changing the quantifier of final conditions, with
kind \texttt{Allow} being \verb+exists+,
kind \texttt{Forbid} being \verb+~exists+
and kind \texttt{Require} being \verb+forall+.
\item[{\tt -conds <file>}] Change the final condition of tests.
\item[{\tt -hints <file>}] Change meta-data, or hints.
Hints command avanced features such as custom affinity
(option \opt{-affinity custom} and Sec.~\ref{affinity:custom})
and prefech control
(option \opt{-preload custom} and Sec.~\ref{preload:custom}).

\end{description}
Observe that the rename mapping is applied first. As a result kind or condition
change must refer to new names. For instance, we can highlight
that a X86 machine is not sequentially consistent by first
renaming \atest{SB} into \ltest{SB+SC}, and then changing the
final condition.
The new condition expresses
that the first instruction (a store)
of one of the threads must come first:
\begin{center}
\begin{tabular}{p{.3\linewidth}p{.2\linewidth}p{.3\linewidth}}
\multicolumn{1}{c}{\afile{rename.txt}} & &
\multicolumn{1}{c}{\afile{cond.txt}}\\ \hline
\begin{verbatim}
SB SB+SC
\end{verbatim}
& \quad\quad\quad &
\begin{verbatim}
SB+SC "forall (0:EAX=1 \/ 1:EAX=1)"
\end{verbatim}
\end{tabular}
\end{center}
Then, we run litmus:
\begin{verbatim}
% litmus7 -mach x86 -rename rename.txt -conds cond.txt SB.litmus
%%%%%%%%%%%%%%%%%%%%%%%%%
% Results for SB.litmus %
%%%%%%%%%%%%%%%%%%%%%%%%%
X86 SB+SC
"Fre PodWR Fre PodWR"

{x=0; y=0;}

 P0          | P1          ;
 MOV [x],$1  | MOV [y],$1  ;
 MOV EAX,[y] | MOV EAX,[x] ;

forall (0:EAX=1 \/ 1:EAX=1)
Generated assembler
#START _litmus_P1
        movl $1,(%r8,%rdx)
        movl (%rdx),%eax
#START _litmus_P0
        movl $1,(%rdx)
        movl (%r8,%rdx),%eax

Test SB+SC Required
Histogram (4 states)
39954 *>0:EAX=0; 1:EAX=0;
3979407:>0:EAX=1; 1:EAX=0;
3980444:>0:EAX=0; 1:EAX=1;
195   :>0:EAX=1; 1:EAX=1;
No

Witnesses
Positive: 7960046, Negative: 39954
Condition forall (0:EAX=1 \/ 1:EAX=1) is NOT validated
Hash=7dbd6b8e6dd4abc2ef3d48b0376fb2e3
Observation SB+SC Sometimes 7960046 39954
Time SB+SC 0.48
\end{verbatim}
One sees that the test name and final condition have changed.

\paragraph*{Miscellaneous\label{misc}}
\begin{description}
\item[{\tt -sleep <n>}] Insert a delay of $n$ seconds between each individual test run.
\item[{\tt -exit <bool>}] Exit status of each individual test executable reflects the final condition success or failure, default \texttt{false} (test exit status always is success in absence  of errors).
\end{description}

\subsection*{\aname{config:sec}{Configuration} files}
The syntax of configuration files is minimal:
lines ``\textit{key} \texttt{=} \textit{arg}'' are interpreted
as setting the value of parameter~\textit{key} to \textit{arg}.
Each parameter has a corresponding option,
usually \opt{-}\textit{key}, except for single-letter options:
\begin{center}
\newenvironment{opts}
{\begin{tabular}{lll}
\hline \hline
\multicolumn{1}{c}{\quad\textit{option}\quad} &
\multicolumn{1}{c}{\quad\textit{key}\quad} &
\multicolumn{1}{c}{\quad\textit{arg}\quad} \\
\hline \hline}
{\hline \hline \end{tabular}}
\begin{opts}
\opt{-a} & avail & integer \\
\opt{-s} & size\_of\_test & integer\\
\opt{-r} & number\_of\_run & integer\\
\opt{-p} & procs & list of integers\\
\opt{-l} & loop & integer\\
\end{opts}
\end{center}
Notice that \litmus{} in fact accepts long versions of options
(\emph{e.g.} \opt{-avail} for~\opt{-a}).

As command line option are processed left-to-right,
settings from a configuration file (option \opt{-mach})
can be overridden by a later
command line option.
Some configuration files for the machines we have tested
are present in the distribution. As an example here is the configuration
file \file{hpcx.cfg}.
\verbatiminput{hpcx.cfg}
Lines introduced by \verb+#+ are comments and are thus ignored.

Configuration files are searched first in the current directory;
then in any directory specified
by setting the shell environment variable \texttt{LITMUSDIR};
and then in litmus installation directory, which is defined
while compiling~\litmus{}.


\clearpage
\part{Generating\label{part:diy} tests}
\cutname{gen.html}

The authors of~\prog{diy} are Jade Alglave and Luc Maranget
(INRIA Paris--Rocquencourt).

\section{Preamble} \label{preamble}

We wrote~\diy{} as part of our empirical approach to studying relaxed memory
models: developing in tandem testing tools and models of multiprocessor
behaviour.  In this tutorial, we attempt an independent tool presentation.
Readers interested by the companion formalism are invited
to refer to our CAV~2010 publication~\cite{ams10}.

The distribution includes additional test generators:
\ahrefloc{diyone:intro}{\prog{diyone}} for generating one test
and \ahrefloc{diycross:intro}{\diycross{}}
for generating simple variations on one test.

\subsection{Relaxation of Sequential Consistency}

Relaxation is one of the key concepts behind simple analysis of weak memory
models.  We define a candidate relaxation by reference to
the most natural model of
parallel execution in shared memory: Sequential Consistency (SC), as defined by
L.~Lamport \cite{lam79}. A parallel program running on a sequentially
consistent machine behaves as an interleaving of its sequential threads.

Consider once more the
example~\ahref{SB-labelled.litmus}{\file{SB.litmus}}:
\verbatiminput{SB-labelled.litmus} To focus on interaction through shared
memory, let us consider memory accesses, or \emph{memory events}.  A memory
event will hold a direction (write, written W, or read, written R), a memory
location (written x, y) a value and a unique label. In any run of the simple
example above, four memory events occur: two writes \lb{c}{W}{x}{1}
and~\lb{a}{W}{y}{1} and two reads \lb{b}{R}{x}{$v_1$} with a certain value $v_1$
and \lb{d}{R}{y}{$v_2$} with a certain value $v_2$.

If the program's
behaviour is modelled by the interleaving of its events, the first event must
be a write of value~\verb+1+ to location~\verb+x+ or \verb+y+ and at least one
of the loads must see a~\verb+1+. Thus, a SC machine would exhibit only three
possible outcomes for this test:

\begin{center}
\begin{tabular}{|l|}
\hline
Allowed: \mbox{0:EAX} $\mathord{=}$ \mbox{0} $\wedge$ \mbox{1:EAX} $\mathord{=}$ \mbox{1}\\\hline
Allowed: \mbox{0:EAX} $\mathord{=}$ \mbox{1} $\wedge$ \mbox{1:EAX} $\mathord{=}$ \mbox{0}\\\hline
Allowed: \mbox{0:EAX} $\mathord{=}$ \mbox{1} $\wedge$ \mbox{1:EAX} $\mathord{=}$ \mbox{1}\\\hline
\end{tabular}
\end{center}

However, running (see Sec.~\ref{x86:classic})
this test on a x86 machine yields an additional result:
\begin{center}
\begin{tabular}{|l|}
\hline
Allowed: \mbox{0:EAX} $\mathord{=}$ \mbox{0} $\wedge$ \mbox{1:EAX} $\mathord{=}$ \mbox{0}\\\hline
\end{tabular}
\end{center}

And indeed, x86 allows each write-read pair on both processors to be
reordered \cite{intel:white}: thus the write-read pair in program order is relaxed on each of these
architectures. We cannot use SC as an accurate memory model for modern
architectures.
Instead we analyse memory models as  \emph{relaxing} the
ordering constraints of the SC~memory model.


\subsection{Introduction\label{candidate:intro} to candidate relaxations}
Consider again our classical example,
from a~SC perspective.
We briefly argued that the outcome 
``\mbox{0:EAX} $\mathord{=}$ \mbox{0} $\wedge$
\mbox{1:EAX} $\mathord{=}$ \mbox{0}''
is forbidden by~SC.
We now present a more complete reasoning:
\begin{itemize}
\item From the condition on outcome, we get the values in
read events: \lb{b}{R}{x}{0} and~\lb{d}{R}{y}{0}.
\item Because of these values,
\lb{b}{R}{x}{0} must precede the write \lb{c}{W}{x}{1}
in the final interleaving of~SC.
Similarly,  \lb{d}{R}{y}{0} must precede the write \lb{a}{W}{y}{1}.
This we note $(b) \fr (c)$ and $(d) \fr (a)$.

\item Because of sequential execution order on one single processor
(a.k.a. \emph{program order}),
\lb{a}Wy1 must precede \lb{b}Rx0 (first processor);
while \lb{c}Wx1 must precede~\lb{d}Ry0 (second processor).
This we note  $(a) \po (b)$ and $(c) \po (d)$.

\item We synthesise the four constraints  above as the following graph:
\begin{center}
\cycle{classic-sc}
\end{center}
Constraint arrows or \emph{global} arrows are shown in brown colour.
As the graph contains a cycle of brown arrows, the events cannot be ordered.
Hence the execution presented is not allowed by~SC.
\end{itemize}

The key idea of \diy{} resides in producing programs from similar
cycles. To that aim, the edges in cycles must convey additional
information:
\begin{itemize}
\item For $\po$ edges, we consider whether the locations
of the events on both sides of the edge are the same or not
('s' or 'd'); and the direction of these events (W or~R).
For instance the two $\po$ edges in the example are PodWR.
(program order edge between a write and a read whose locations are
different).
\item For $\fr$ edges, we consider whether the processor
of the events on both sides of the edge are the same or not
('i' for internal, or 'e' for external).
For instance the two $\fr$ edges in the example are Fre.
\end{itemize}

So far so good, but our x86 machine produced the outcome
\(\mbox{0:EAX} \mathord{=} \mbox{0} \wedge
\mbox{1:EAX} \mathord{=} \mbox{0}\).
The Intel Memory Ordering White Paper~\cite{intel:white}
specifies: ``Loads may be reordered with older stores to different locations'',
which we rephrase as: PodWR is relaxed.
Considering Fre to be safe, we have the graph:
\begin{center}
\cycle{classic}
\end{center}
And the brown sub-graph becomes acyclic.

We shall see later why we choose to relax PodWR and not Fre.
At the moment, we observe that we can assume PodWR to be relaxed
and Fre not to be (\emph{i.e.} to be \emph{safe}) and test
our assumptions, by producing and running more litmus tests.
The \diy{} suite precisely provides tools for this approach.

\label{diyone:intro}As a first example,
\afile{SB.litmus}
can be created as follows:
\begin{verbatim}
% diyone -arch X86 -name SB Fre PodWR Fre PodWR
\end{verbatim}

As a second example, we can produce several similar tests
as follows:
\begin{verbatim}
% diy -arch X86 -safe Fre -relax PodWR -name SB
Generator produced 2 tests
Relaxations tested: {PodWR}
\end{verbatim}
\diy{} produces two litmus tests, \file{SB000.litmus}
and \file{SB001.litmus}, plus one index file~\file{@all}.
One of the litmus tests generated is the same as above, while
the new test is:
\begin{verbatim}
% cat SB001.litmus
X86 SB001
"Fre PodWR Fre PodWR Fre PodWR"
Cycle=Fre PodWR Fre PodWR Fre PodWR
Relax=PodWR
Safe=Fre
{ }
 P0          | P1          | P2          ;
 MOV [z],$1  | MOV [x],$1  | MOV [y],$1  ;
 MOV EAX,[x] | MOV EAX,[y] | MOV EAX,[z] ;
exists (0:EAX=0 /\ 1:EAX=0 /\ 2:EAX=0)
% cat @all
# diy -arch X86 -safe Fre -relax PodWR -name SB
# Revision: 3333
SB000.litmus
SB001.litmus
\end{verbatim}

\diy{} first generates cycles from the candidate relaxations
given as arguments, up to a limited size, and then generates
litmus tests from these cycles.

\subsection{More candidate relaxations}

We assume the memory to be \emph{coherent}.
Coherence implies that, in a given execution,
the writes to a given location are performed by following a sequence,
or \emph{coherence order},
and that all processors see the same sequence.


\label{sec:ws}In \diy{}, the coherence orders are specified indirectly.
For instance, the candidate relaxation Wse (resp. Wsi) specifies two writes,
performed by different processors (resp. the same processor),
to the same location~$\ell$, the first write preceding the second in
the coherence order of~$\ell$.
The condition of the produced test then selects the specified coherence orders.
Consider for instance:
\begin{verbatim}
% diyone -arch X86 -name x86-2+2W Wse PodWW Wse PodWW
\end{verbatim}
The cycle that reveals a violation of the SC memory model is:
\begin{center}
\cycle[.2\linewidth]{ws-sc}
\end{center}
So the coherence order is $0$ (initial store, not depicted),
$1$, $2$ for both locations x and~y.
While the produced test is:
\verbatiminput{x86-2+2W.litmus}
By the coherence hypothesis, checking the final
value of locations suffices to characterise those two coherence orders,
as expressed by the final condition of~\ahref{x86-2+2W.litmus}{\ltest{x86-2+2W}}:
\begin{verbatim}
exists (x=2 /\ y=2)
\end{verbatim}
See Sec.~\ref{observers} for alternative means
to identify coherence orders.

\label{iriw}Candidate relaxations Rfe and Rfi relate
writes to reads that load their value.
We are now equipped to generate the famous iriw test
(independent reads of independent writes):
\begin{verbatim}
% diyone -arch X86 Rfe PodRR Fre Rfe PodRR Fre -name iriw
\end{verbatim}
We generate its internal variation (\emph{i.e.} where all Rfe are replaced by Rfi) as easily:
\begin{verbatim}
% diyone -arch X86 Rfi PodRR Fre Rfi PodRR Fre -name iriw-internal
\end{verbatim}
We get the cycles of \myfig{\ref{fig:iriw}},
\begin{figure}[p]
\caption{\label{fig:iriw} Cycles for iriw and iriw-internal}
\begin{center}
\begin{tabular}{m{.40\linewidth}@{\quad}@{\quad}m{.40\linewidth}}
\begin{center}\cycle[.6\linewidth]{iriw-small-caps}\end{center}
&
\begin{center}\cycle[.6\linewidth]{iriw-internal}\end{center}
\end{tabular}
\end{center}
\end{figure}
and the litmus tests of~\myfig{\ref{fig:iriw:test}}.
\begin{figure}[p]
\caption{\label{fig:iriw:test} Litmus tests iriw and iriw-internal}
\begin{center}\small
\begin{tabular}{p{.55\linewidth}@{\quad}|@{\quad}p{.40\linewidth}}
\verbatiminput{iriw.litmus} & \verbatiminput{iriw-internal.litmus} \\
\end{tabular}
\end{center}
\end{figure}

Candidate relaxations given as arguments really are a
``concise specification''. As an example,
we get iriw for Power, simply by changing \opt{-arch X86} into
\opt{-arch PPC}.
\begin{verbatim}
% diyone -arch PPC Rfe PodRR Fre Rfe PodRR Fre
PPC a
"Rfe PodRR Fre Rfe PodRR Fre"
{
0:r2=y; 0:r4=x;
1:r2=x;
2:r2=x; 2:r4=y;
3:r2=y;
}
 P0           | P1           | P2           | P3           ;
 lwz r1,0(r2) | li r1,1      | lwz r1,0(r2) | li r1,1      ;
 lwz r3,0(r4) | stw r1,0(r2) | lwz r3,0(r4) | stw r1,0(r2) ;
exists (0:r1=1 /\ 0:r3=0 /\ 2:r1=1 /\ 2:r3=0)
\end{verbatim}
Also notice that without the \opt{-name} option, \prog{diyone} writes
its result to standard output.

\subsection{Summary of simple candidate relaxations}
We summarise the candidate relaxations available on all architectures.

\subsubsection{Communication\label{communication:cr} candidate relaxations}
We call communication candidate relaxations the relations between two events
communicating through memory, though they could belong to the same processor.
Thus, these events operate on the same memory location.
\begin{center}
\begin{tabular}{c|c|c|c|p{.4\linewidth}}
\diy{} syntax & Source & Target & Processor &
\multicolumn{1}{c}{Additional property}
\\\hline
\tt Rfi      & W      & R      & Same      &
Target reads its value from source \\\hline
\tt Rfe      & W      & R      & Different &
Target reads its value from source  \\\hline
\tt Wsi      & W      & W      & Same      & 
Source precedes target in coherence order
\\\hline
\tt Wse      & W      & W      & Different &
Source precedes target in coherence order \\\hline
\tt Fri     & R      & W      & Same      &
Source reads a value from a write that precedes target in coherence order
  \\\hline
\tt Fre  & R      & W      & Different &
Source reads a value from a write that precedes target in coherence order
\\\hline
\end{tabular}
\end{center}


\subsubsection{Program order candidate relaxations}

We call program order candidate relaxations each relation
between two events in the
program order. These events are on the same processor, since they are in
program order.
As regards code output,
\diy{} interprets a program order candidate relaxation by generating
two memory instructions (load or store) following one another.

Program order candidate relaxations have the following syntax: 
\begin{center}
Po(s\vbar{}d)(R\vbar{}W)(R\vbar{}W)
\end{center}
where:
\begin{itemize}
\item s (resp. d) indicates that the two events are to the same (resp. different)
location(s);
\item R (resp. W) indicates an event to be a  read (resp. a write);
\end{itemize}

In practice, we have:
\begin{center}
\begin{tabular}{c|c|c|c}
\prog{diy} syntax  &  Source & Target  & Location \\ \hline
\tt PosRR & R & R & Same \\ \hline
\tt PodRR & R & R & Diff  \\ \hline
\tt PosRW & R & W & Same \\ \hline
\tt PodRW & R & W & Diff \\ \hline
\tt PosWW & W & W & Same \\ \hline
\tt PodWW & W & W & Diff \\ \hline
\tt PosWR & W & R & Same \\ \hline
\tt PodWR & W & R & Diff \\ \hline
\end{tabular}
\end{center}
It is to be noticed
that PosWR, PosWW and PosRW are similar to Rfi, Wsi and~Fri, respectively.
More precisely, \diy{} is unable to consider a PosWR (or PosWW, or PosRW)
candidate relaxation as not being also a Rfi (or Wsi, or Fri) candidate
relaxation.
However, litmus tests conditions may be more informative in the case of
Rfi and~Fri.




\subsubsection{Fence candidate relaxations}
Relaxed architectures provide specific instructions, namely \emph{barriers}
or \emph{fences}, to enforce order of memory accesses.
In \diy{} the presence of a fence instruction is specified with
fence candidate relaxations, similar to program order candidate relaxations,
except that a fence instruction is inserted.
Hence we have FencedsRR, FenceddRR. etc.
The inserted fence is the strongest fence provided by
the architecture --- that is, \texttt{mfence} for x86 and
\texttt{sync}~for~Power.

Fences can also be specified by using specific names.
More precisely, we have  MFence for x86;
while on Power we have Sync, LwSync, Eieio and~Isync.
Hence, to yield two reads to different locations
and separated by the lightweight Power barrier~\texttt{lwsync},
we specify LwSyncdRR.
On ARM we have DMB, DSB and ISB.



\section{Testing\label{diy:intro} candidate relaxations with \diy}

The tool~\diy{} can probably be used in various, creative, ways;
but the tool first stems from our technique for testing relaxed memory
models.
The \opt{-safe} and \opt{-relax} options are crucial here.
We describe our technique by the means of an example: X86-TSO.

Notice that this style of model exploration is mechanised
by~\dont\prog{ (diy)} --- see Part~\ref{part:auto}.

\subsection{\aname{test2}{Principle}}

Before engaging in testing it is important to categorise
candidate relaxations as safe or relaxed.

This can done by interpretation of vendor's documentation.
For instance, the iriw test of~\mysec{\ref{iriw}} is the example 7.7 of
\cite{intel:white}
``Stores Are Seen in a Consistent Order by Other Processors'',
with a Forbid specification.
Hence we deduce that Fre, Rfe and PodRR are safe.
Then, from test~iriw-internal of~\mysec{\ref{iriw}},
which is Intel's test~7.5 ``Intra-Processor Forwarding Is Allowed''
with an allow specification,
we deduce that Rfi is relaxed.
Namely, the cycle of iriw-internal is
``Fre Rfi PodRR Fre Rfi PodRR''. Therefore, the only possibility is for
Rfi to be relaxed.

Overall, we deduce:
\begin{itemize}
\item Candidate relaxations PosWR (Rfi) and PodWR are relaxed
\item The remaining candidate relaxations PosRR, PodRR, PosWW (Wsi),  PodWW,
PosRW (Fri), Fre and Wse are safe.
Fence relaxations FencedsWR and FenceddWR are also safe
and worth testing.
\end{itemize}

Of course these remain assumptions to be tested.
To do so, we perform one series of tests per relaxed candidate relaxation,
and one series of tests for confirming safe candidate relaxations
as much as possible. Let $S$~be all safe candidate relaxations.
\begin{itemize}
\item Let $r$ be a relaxed candidate relaxation.
We produce tests for confirming $r$~being relaxed by
\texttt{diy -relax}~$r$~\texttt{-safe}~$S$.
We run these tests with \litmus{}. If one of the tests yields~\texttt{Ok},
then $r$~is confirmed to be relaxed, provided the experiments on~$S$ below
do not  fail.
\item For confirming the safe set, we produce tests
by \texttt{diy -safe}~$S$.
We run these tests as much as possible and expect never to see~\texttt{Ok}.
\end{itemize}
\label{diy:one:relax}Namely, \diy{} builds cycles as follows:
\begin{itemize}
\item \texttt{diy -relax}~$r$~\texttt{-safe}~$S$ build cycles
with at least one~$r$ taking other candidate relaxations from~$S$.
\item \texttt{diy -safe}~$S$ build cycles from the candidate relaxations in~$S$.
\end{itemize}
For the purpose of confirming relaxed
candidate relaxations, $S$ can be replaced by a subset.

\subsection{Testing~x86}
Repeating command line options is painful and error prone.
Besides, configuration parameters may get lost.
Thus, we regroup those in configuration files
that simply list the options to be passed to \diy, one option per line.
For instance here is the configuration file for testing the safe relaxations
of~x86, \ahref{x86-safe.conf}{\file{x86-safe.conf}}.
\verbatiminput{x86-safe.conf}
Observe that the syntax of candidate relaxations allows one shortcut:
the wildcard \texttt{*} stands for \texttt{W} and~\texttt{R}.
Thus \texttt{PodR*} gets expanded  to the two candidate
relaxations \texttt{PodRR} and~\texttt{PodRW}.

\label{safe:test:sec}%
We get safe tests by issuing the following command, preferably in a
specific directory, say \texttt{safe}.
\begin{verbatim}
% diy -conf x86-safe.conf
Generator produced 38 tests
Relaxations tested: {}
\end{verbatim}

Here are the configuration files for confirming
that Rfi and PodWR are relaxed, \afile{x86-rfi.conf}
and~\afile{x86-podwr.conf}.
\begin{center}\small
\begin{tabular}{p{0.7\linewidth}@{\quad}|@{\quad}p{0.2\linewidth}}
\verbatiminput{x86-rfi.conf} &
\verbatiminput{x86-podwr.conf}
\end{tabular}
\end{center}
Notice that we used the complete safe list in
\file{x86-rfi.conf} and a reduced list in~\file{x86-podwr.conf}.
Tests are to be generated in specific directories.
%HEVEA To that aim, we provide a convenient archive~\ahref{x86.tar}{\texttt{x86.tar}}.
\begin{verbatim}
% cd rfi
% diy -conf x86-rfi.conf 
Generator produced 11 tests
Relaxations tested: {Rfi}
% cd ../podwr
% diy -conf x86-podwr.conf 
Generator produced 2 tests
Relaxations tested: {PodWR}
% cd ..
\end{verbatim}

Now, \ahref{litmus.html}{let us run} all tests at once, with the
parameters of machine \texttt{saumur} (4 physical cores with hyper-threading):
\begin{verbatim}
% litmus -mach saumur rfi/@all > rfi/saumur.rfi.00
% litmus -mach saumur podwr/@all > podwr/saumur.podwr.00
% litmus -mach saumur safe/@all > safe/saumur.safe.00
\end{verbatim}
If your machine has 2 cores only, try \verb+litmus -a 2 -limit true+\ldots

\label{readRelax:intro}We now look for
the tests that have validated their condition
in the result files of~\litmus.
A simple tool, \prog{readRelax}, does the job:
\begin{verbatim}
% readRelax rfi/saumur.rfi.00 podwr/saumur.podwr.00 safe/saumur.safe.00
   .
   .
   .
** Relaxation summary **
{Rfi} With {Rfe, Fre, Wse, PodRW, PodRR} {Rfe, Fre, PodRR}\
{Fre, Wse, PodWW, PodRR} {Fre, PosWW, PodRR, MFencedWR}\
{Fre, PodWW, PodRR, MFencedWR} {Fre, PodRR} {Fre, PodRR, MFencedWR}
{PodWR} With {Fre}
\end{verbatim}
The tool \prog{readRelax} first lists the result of all tests
(which is omitted above), and then dumps a summary of the 
relaxations it found.
The sets of the candidate relaxations  that need to be safe for the tests to
indeed reveal  a relaxed candidate relaxation are also given.
Here, Rfi and~PodWR are confirmed to be relaxed, while no candidate relaxation
in the safe set is found to be relaxed.
Had it been the case, a line \verb+{} With {...}+ would have occurred
in the relaxation summary.
The safe tests need to be run a lot of times, to increase our
confidence in the safe set.

\section{Additional relaxations}
We introduce some additional candidate relaxations
that are specific to the Power architecture.
We shall not detail here our experiments on Power machines.
See our experience report~\ahrefurl{http://diy.inria.fr/phat/}
for more details.

\subsection{Intra-processor dependencies}
In a very relaxed architecture such as Power,
\emph{intra-processor dependencies} becomes significant.
Roughly, intra-processor dependencies fall into two categories:
\begin{description}
\item[Data dependencies] occur when a memory access instruction
reads a register whose contents depends upon a previous (in program order)
load. In \diy{} we specify such a dependency as:
\begin{center}Dp(s\vbar{}d)(R\vbar{}W)\end{center}
where, as usual,
s (resp. d) indicates that the source and target events
are to the same (resp. different) location(s);
and  R (resp. W) indicates that the target event is a read (resp. a write).
As a matter of fact, we do not need to specify the direction of
the source event, since it always is a read.

Finally, one may control the nature of the dependency:
address dependency (DpAddr(s\vbar{}d)(R\vbar{}W) or
data dependency (DpData(s\vbar{}d)W).


\item[Control dependencies] occur when the execution of a memory access
is conditioned by the contents of a previous load.
Their syntax is similar to the one of Dp~relaxations, with a Ctrl~tag:
\begin{center}Ctrl(s\vbar{}d)(R\vbar{}W)\end{center}
This default syntax expands to control dependencies as
guaranteed by the Power documentation.
For read to write, conditioning execution is enough
(expanded syntax, DpCtrl(s\vbar{}d)W).
But for read to read, an extra instruction, \texttt{isync}, is needed
(expanded syntax  DpCtrlIsync(s\vbar{}d)R, see \ahrefloc{ctrlisync}{below}).
The syntax DpCtrl(s\vbar{}d)R also exists,
it expresses the conditional execution of a load instruction and
does \emph{not} create ordering.

ARM has similar candidate relaxations, Isync being replaced by ISB.
\end{description}
In the produced code, \diy{} expresses
a data dependency by a \emph{false dependency} (or \emph{dummy dependency})
that operates on the address of the target memory access.
For instance:
\begin{verbatim}
% diyone DpdW Rfe DpdW Rfe
PPC a
"DpAddrdW Rfe DpAddrdW Rfe"
{
0:r2=y; 0:r5=x;
1:r2=x; 1:r5=y;
}
 P0            | P1            ;
 lwz r1,0(r2)  | lwz r1,0(r2)  ;
 xor r3,r1,r1  | xor r3,r1,r1  ;
 li r4,1       | li r4,1       ;
 stwx r4,r3,r5 | stwx r4,r3,r5 ;
exists (0:r1=1 /\ 1:r1=1)
\end{verbatim}
On \P{0}, the effective address of the indexed store \verb+stwx r4,r3,r5+
depends on the contents of the index register~\verb+r3+, which itself
depends on the contents of~\verb+r1+.
The dependency is a ``false'' one, since the contents of~\verb+r3+
always is zero, regardless of the contents of~\verb+r1+.
One may observe that DpdW is changed into DpAddrdW in the comment
field of the test.
As a matter of fact, DpdW is a macro for the address dependency
DpAddrW. We could have specified data dependency instead:
\begin{verbatim}
% diyone DpDatadW Rfe DpAddrdW Rfe
PPC a
"DpDatadW Rfe DpAddrdW Rfe"
{
0:r2=y; 0:r4=x;
1:r2=x; 1:r5=y;
}
 P0           | P1            ;
 lwz r1,0(r2) | lwz r1,0(r2)  ;
 xor r3,r1,r1 | xor r3,r1,r1  ;
 addi r3,r3,1 | li r4,1       ;
 stw r3,0(r4) | stwx r4,r3,r5 ;
exists
(0:r1=1 /\ 1:r1=1)
\end{verbatim}
On \P{0}, the value stored by the last (store) instruction
\verb+stw r3,0(r4)+ is now computed from the 
value read by the first (load) instruction \verb+lwz r1,0(r2)+.
Again, this is a ``false'' dependency.

\aname{ctrlisync}{A control}
dependency is implemented by the means of an useless compare
and branch sequence, plus the \texttt{isync} instruction when the target event
is a load. For instance
\begin{verbatim}
% diyone CtrldR Fre SyncdWW Rfe
PPC a
"DpCtrlIsyncdR Fre SyncdWW Rfe"
{
0:r2=y; 0:r4=x;
1:r2=x; 1:r4=y;
}
 P0           | P1           ;
 lwz r1,0(r2) | li r1,1      ;
 cmpw r1,r1   | stw r1,0(r2) ;
 beq  LC00    | sync         ;
 LC00:        | li r3,1      ;
 isync        | stw r3,0(r4) ;
 lwz r3,0(r4) |              ;
exists
(0:r1=1 /\ 0:r3=0)
\end{verbatim}
Also notice that CtrldR is interpreted as DpCtrlIsyncR in the comment
field of the test.

Of course, in all cases, we assume that ``false'' dependencies are not
``optimised out'' by the assembler or the hardware.

\subsection{\aname{composite}{Composite} relaxations and cumulativity}
Users may specify a small sequence of single candidate relaxations
as behaving as a single candidate relaxation to~\diy. The~syntax~is:
\begin{center}
[$r1$, $r2$, \ldots]
\end{center}
The main usage of the feature is to specify
\emph{cumulativity candidate relaxations},
that is, the sequence of Rfe and of a fence candidate relaxation
(A-cumulativity),
the sequence of a fence candidate relaxation and of~Rfe (B-cumulativity),
or both (AB-cumulativity).

Cumulativity candidate relaxations are best expressed by the following
syntactical shortcuts:
let $r$ be a fence candidate relaxation, then
\texttt{AC}$r$ stands for~\texttt{[Rfe,}$r$\texttt{]},
\texttt{BC}$r$ stands for~\texttt{[}$r$\texttt{,Rfe]},
while \texttt{ABC}$r$ stands for~\texttt{[Rfe,}$r$\texttt{,Rfe]},

Hence, a simple way to generate iriw-like (see~\mysec{\ref{iriw}})
litmus tests  with \texttt{lwsync} is as follows:
\begin{verbatim}
% diy -name iriw-lwsync -nprocs 8 -size 8 -relax ACLwSyncdRR -safe Fre
Generator produced 3 tests
Relaxations tested: {ACLwSyncdRR}
\end{verbatim}
where we have for instance:
\begin{verbatim}
% cat iriw-lwsync001.litmus 
PPC iriw-lwsync001
"Fre Rfe LwSyncdRR Fre Rfe LwSyncdRR Fre Rfe LwSyncdRR"
Cycle=Fre Rfe LwSyncdRR Fre Rfe LwSyncdRR Fre Rfe LwSyncdRR
Relax=ACLwSyncdRR
Safe=Fre
{
0:r2=z; 0:r4=x; 1:r2=x;
2:r2=x; 2:r4=y; 3:r2=y;
4:r2=y; 4:r4=z; 5:r2=z;
}
 P0           | P1           | P2           | P3           | P4           | P5           ;
 lwz r1,0(r2) | li r1,1      | lwz r1,0(r2) | li r1,1      | lwz r1,0(r2) | li r1,1      ;
 lwsync       | stw r1,0(r2) | lwsync       | stw r1,0(r2) | lwsync       | stw r1,0(r2) ;
 lwz r3,0(r4) |              | lwz r3,0(r4) |              | lwz r3,0(r4) |              ;
exists (0:r1=1 /\ 0:r3=0 /\ 2:r1=1 /\ 2:r3=0 /\ 4:r1=1 /\ 4:r3=0)
\end{verbatim}

\subsection{Detour \label{detour:def}candidate relaxations}
Detours combine a Pos candidate relaxation and a sequence
of two \emph{external} communication candidate relaxations.
More precisely detours are some constrained Pos candidate relaxations:
the source and target events must be related by a sequence of
two communication candidate relaxations, whose target and source
are a common event whose processor is new.
\begin{center}
\begin{tabular}{c|c|c|c}
\prog{diy} syntax  &  Source & Target  & Detour \\ \hline
\tt DetourR & R & R & Fre; Rfe \\ \hline
\tt DetourW & W & R & Wse; Rfe  \\ \hline
\tt DetourRW & R & W & Fre;Wse \\ \hline
\tt DetourWW & W & W & Wse;Wse \\ \hline
\end{tabular}
\end{center}
DetourRR and DetourWR are accepted as synonyms for
DetourR and DetourW respectively.

Graphically, we have:
\begin{center}
\img{DetourR}\quad
\img{DetourW}\quad
\img{DetourRW}\quad
\img{DetourWW}
\end{center}
Finally notice that ``internal'' detours need no special treatement
as they can be expressed by the sequences ``Fri; Rfi'', ``Wsi;Rfi'', etc.

\section{Test\label{diycross:intro} variations with \diycross{}}
The tool \diycross{} has an interface similar to \diyone,
except it accepts list of candidate relaxations where \diyone{} accepts
single candidate relaxations.
The new tool produces the test resulting by ``cross producing'' the lists.
For instance, one can generate all variations on the IRIW test
(see Sec.~\ref{iriw}) that
involve data dependencies and the lightweight barrier~\texttt{lwsync}
as follows:
\begin{verbatim}
% diycross -arch PPC -name IRIW Rfe DpdR,LwSyncdRR Fre Rfe DpdR,LwSyncdRR Fre
Generator produced 3 tests
% ls
@all  IRIW+addrs.litmus  IRIW+lwsync+addr.litmus  IRIW+lwsyncs.litmus
\end{verbatim}
\diycross{} outputs the index file \texttt{@all}
that lists the test source files, and three tests, with
names we believe to be self-explanatory:
\begin{verbatim}
% cat IRIW+lwsync+addr.litmus
PPC IRIW+lwsync+addr
"Rfe LwSyncdRR Fre Rfe DpAddrdR Fre"
Cycle=Rfe LwSyncdRR Fre Rfe DpAddrdR Fre
{
0:r2=y;
1:r2=y; 1:r4=x;
2:r2=x;
3:r2=x; 3:r5=y;
}
 P0           | P1           | P2           | P3            ;
 li r1,1      | lwz r1,0(r2) | li r1,1      | lwz r1,0(r2)  ;
 stw r1,0(r2) | lwsync       | stw r1,0(r2) | xor r3,r1,r1  ;
              | lwz r3,0(r4) |              | lwzx r4,r3,r5 ;
exists (1:r1=1 /\ 1:r3=0 /\ 3:r1=1 /\ 3:r4=0)
\end{verbatim}

Users may use the special keywords allRR, allRW, allWR and allWW to
specify the set of all existing program order candidate relaxations
between the specified ``R'' or ``W''. For instance, we get the complete
variations on IRIW by:
\begin{verbatim}
% diycross -arch PPC -name IRIW Rfe allRR Fre Rfe allRR Fre
Generator produced 28 tests
% ls
@all
IRIW.litmus
IRIW+addr+po.litmus
IRIW+lwsync+addr.litmus
...
IRIW+isyncs.litmus
\end{verbatim}

\section{Identifying\label{observers} coherence orders with observers}
We first produce the ``\emph{four writes}'' test~\ahref{2+2W.litmus}{\file{2+2W}}
for Power:
\begin{verbatim}
% diyone -name 2+2W -arch PPC PodWW Wse PodWW Wse
% cat 2+2W.litmus
PPC 2+2W
"PodWW Wse PodWW Wse"
{ 0:r2=x; 0:r4=y; 1:r2=y; 1:r4=x; }
 P0           | P1           ;
 li r1,2      | li r1,2      ;
 stw r1,0(r2) | stw r1,0(r2) ;
 li r3,1      | li r3,1      ;
 stw r3,0(r4) | stw r3,0(r4) ;
exists (x=2 /\ y=2)
\end{verbatim}
Test \ltest{2+2W} is the Power version of the x86 test
\atest{x86-2+2W} of Sec.~\ref{sec:ws}.
In that section, we argued that the final condition \verb+exists (x=2 /\ y=2)+ 
suffices to identify the coherence orders $0$, $1$, $2$
for locations \texttt{x} and~\texttt{y}.
As a consequence, a positive final condition reveals the occurrence
of the specified cycle: Wse PodWW Wse PodWW.

\subsection{Simple observers}
\emph{Observers} provide an alternative, perhaps more intuitive,
means to identify coherence orders: an observer simply is an additional thread
that performs several loads from the same location in sequence.
Here, loading value~$1$ and then value~$2$ from location~\texttt{x}
identifies the coherence order  $0$, $1$, $2$.
The command line switch \opt{-obs force} commands the production
of observers (test \ahref{2+2WObs.litmus}{\file{2+2WObs}}):
\begin{verbatim}
% diyone -name 2+2WObs -obs force -obstype straight -arch PPC PodWW Wse PodWW Wse
% cat 2+2WObs.litmus
PPC 2+2WObs
"PodWW Wse PodWW Wse"
{ 0:r2=x; 1:r2=y; 2:r2=x; 2:r4=y; 3:r2=y; 3:r4=x; }
 P0           | P1           | P2           | P3           ;
 lwz r1,0(r2) | lwz r1,0(r2) | li r1,2      | li r1,2      ;
 lwz r3,0(r2) | lwz r3,0(r2) | stw r1,0(r2) | stw r1,0(r2) ;
              |              | li r3,1      | li r3,1      ;
              |              | stw r3,0(r4) | stw r3,0(r4) ;
exists (0:r1=1 /\ 0:r3=2 /\ 1:r1=1 /\ 1:r3=2)
\end{verbatim}
Thread \texttt{P0} observes location~\texttt{x}, while
thread \texttt{P1} observes location~\texttt{y}.
With respect to \texttt{2+2W}, final condition has changed, the direct
observation of the final contents of locations \texttt{x} and~\texttt{y}
being replaced by two successive observations of the contents of
\texttt{x} and~\texttt{y}.

It should first be noticed that the reasoning above assumes
that having the same thread to read $1$ from say~\texttt{x}
and then $2$ implies that $1$ takes place before $2$
in the coherence order of~\texttt{x}.
This may not be the case in general --- although it holds for Power.
Moreover, running \file{2+2W} and~\file{2+2WObs}
yields contrasted results. While a positive conclusion is
immediate for \file{2+2W}, we were not able to reach a similar conclusion
for \file{2+2WObs}.
As a matter of fact, \file{2+2WObs} yielding Ok stems from the
still-to-be-observed coincidence
of several events: \emph{both} observers threads must run at the right pace to
observe the change from~$1$ to~$2$, while the cycle must indeed occur.

\subsection{More\label{sec:obstype} observers}

\subsubsection{Fences and loops in observers}
A simple observer consisting of loads performed in sequence is a
\emph{straight} observer. We define two additional sorts of observers:
\emph{fenced} observers, where loads are separated by the strongest fence
available, and \emph{loop} observers, which poll on location contents change.
Those are selected by the homonymous tags given as arguments to the command
line switch \opt{-obstype}. For instance, we get the test
\ahref{2+2WObsFenced.litmus}{\file{2+2WObsFenced}} by:
\begin{verbatim}
% diyone -name 2+2WObsFenced -obs force -obstype fenced -arch PPC PodWW Wse PodWW Wse
% cat 2+2WObsFenced.litmus
PPC 2+2WObsFenced
"PodWW Wse PodWW Wse"
{ 0:r2=x; 1:r2=y; 2:r2=x; 2:r4=y; 3:r2=y; 3:r4=x; }
 P0           | P1           | P2           | P3           ;
 lwz r1,0(r2) | lwz r1,0(r2) | li r1,2      | li r1,2      ;
 sync         | sync         | stw r1,0(r2) | stw r1,0(r2) ;
 lwz r3,0(r2) | lwz r3,0(r2) | li r3,1      | li r3,1      ;
              |              | stw r3,0(r4) | stw r3,0(r4) ;
exists (0:r1=1 /\ 0:r3=2 /\ 1:r1=1 /\ 1:r3=2)
\end{verbatim}
Invoking \diyone{} as ``\verb+diyone -obs force -obstype loop ...+''
yields the additional  test \ahref{2+2WObsLoop.litmus}{\file{2+2WObsLoop}}\ifhevea:
\verbatiminput{2+2WObsLoop.litmus}
A loop observer first busily waits for the observed location not
to hold its initial contents~$0$, and then busily waits for another change
of location contents.
The second loop is performed at most a finite number of times
(here~$200$), in order to ensure termination.
\else.
The \textsc{html}~version of this document provides details.
\fi

\subsubsection{Local observers}
With local observers, coherence order is observed by the test threads.
This implies changing the tests, and some care must be exercised when
interpreting results.

The idea is as follows: when two threads are connected by a Wse candidate
relaxation, meaning that the first thread ends by writing $v$ to some location~$\ell$ and that the second threads starts by writing $v+1$ to the same location~$\ell$, we add an observing read of location~$\ell$ at the end of
the first thread. Then, reading $v+1$ means that the write by the first thread
precedes the write by the second thread in $\ell$ coherence order.
More concretely, we instruct \prog{diy} generators to emit such local observers
with option \opt{-obs local}:
\begin{verbatim}
% diyone -name 2+2WLocal -obs local -obstype straight -arch PPC PodWW Wse PodWW Wse
% cat 2+2WLocal.litmus 
PPC 2+2WLocal
"PodWW Wse PodWW Wse"
{
0:r2=x; 0:r4=y;
1:r2=y; 1:r4=x;
}
 P0           | P1           ;
 li r1,2      | li r1,2      ;
 stw r1,0(r2) | stw r1,0(r2) ;
 li r3,1      | li r3,1      ;
 stw r3,0(r4) | stw r3,0(r4) ;
 lwz r5,0(r4) | lwz r5,0(r4) ;
exists
(0:r5=2 /\ 1:r5=2)
\end{verbatim}
With respect to \texttt{2+2W}, final condition has changed, the direct
observation of the final contents of locations \texttt{y} and~\texttt{x}
being replaced local observation of \texttt{y} by thread~0 and
local observation of~\texttt{x} by thread~1.

Based for instance on the test execution witness, whose only SC-violation
cycle is the same as as  for \ltest{2+2W},
\begin{center}
\cycle{2+2WLocal+DOT}
\end{center}
one may argue that tests
\ltest{2+2W} and~\ltest{2+2WLocal} are equivalent, in the sense that
both are allowed or both are forbidden by a model or machine.

Local observers can also be fenced or looping.
For instance, one produces
\ahref{2+2WLocalFenced.litmus}{\ltest{2+2WLocalFenced}},
the fenced local observer version of \ltest{2+2W}
as follows:
\begin{verbatim}
% diyone -name 2+2WLocalFenced -obs local -obstype fenced -arch PPC PodWW Wse PodWW Wse
% cat 2+2WLocalFenced.litmus 
PPC 2+2WLocalFenced
"PodWW Wse PodWW Wse"
{
0:r2=x; 0:r4=y;
1:r2=y; 1:r4=x;
}
 P0           | P1           ;
 li r1,2      | li r1,2      ;
 stw r1,0(r2) | stw r1,0(r2) ;
 li r3,1      | li r3,1      ;
 stw r3,0(r4) | stw r3,0(r4) ;
 sync         | sync         ;
 lwz r5,0(r4) | lwz r5,0(r4) ;
exists
(0:r5=2 /\ 1:r5=2)
\end{verbatim}
While one produces
\ahref{2+2WLocalLoop.litmus}{\ltest{2+2WLocalLoop}}, the looping local observer version of~\ltest{2+2W} as follows:
\begin{verbatim}
% diyone -name 2+2WLocalLoop -obs local -obstype loop -arch PPC PodWW Wse PodWW Wse
% cat 2+2WLocalLoop.litmus 
PPC 2+2WLocalLoop
"PodWW Wse PodWW Wse"
{
0:r2=x; 0:r4=y;
1:r2=y; 1:r4=x;
}
 P0            | P1            ;
 li r1,2       | li r1,2       ;
 stw r1,0(r2)  | stw r1,0(r2)  ;
 li r3,1       | li r3,1       ;
 stw r3,0(r4)  | stw r3,0(r4)  ;
 li r6,200     | li r6,200     ;
 L00:          | L02:          ;
 lwz r5,0(r4)  | lwz r5,0(r4)  ;
 cmpwi r5,1    | cmpwi r5,1    ;
 bne  L01      | bne  L03      ;
 addi r6,r6,-1 | addi r6,r6,-1 ;
 cmpwi r6,0    | cmpwi r6,0    ;
 bne  L00      | bne  L02      ;
 L01:          | L03:          ;
exists (0:r5=2 /\ 1:r5=2)
\end{verbatim}
In the code above,
observing loads are attempted at most 200~time or until a value different
from~$1$ is read.

\subsubsection{Performance of observers}
As an indication of the performance of the various sorts of observers,
the following table summarises a \litmus{} experiment performed on
a 8-cores 4-ways SMT Power7 machine
machine\ifhevea --- \ahref{W.00}{complete \litmus{} log}\fi.
\begin{center}
\begin{tabular}{l|*{7}{c}}
 & \ltest{2+2W} & \ltest{2+2WObs} & \ltest{2+2WObsFenced} & \ltest{2+2WObsLoop} &
\ltest{2+2WLocal} & \ltest{2+2WLocalFenced} & \ltest{2+2WLocalLoop} \\
\hline
Positive & 2.2M/160M & 0/80M & 326/80M & 25k/80M &
2/160M & 34k/160M & 111k/160M\\
\end{tabular}
\end{center}
The row ``Positive'' shows the number of observed positive outcomes/total
number of outcomes  produced.
For instance, in the case of \ltest{2+2W}, we observed the positive outcome
\verb+x=2 /\ y=2+
more than 2 millions times out of a total of 160 millions  outcomes.
As a conclusion, all techniques achieve decent results, except straight
observers.

\subsection{Three \label{sec:obs}stores or more}
In test \ltest{2+2W} the coherence orders sequence two writes.
If there are three writes or more to the same location,
it is no longer possible to
identify a coherence order by observing the final contents of the
memory location involved. In other words, observers are mandatory.

The argument to the \opt{-obs} switch commands
the production of observers. It can take four values:
\begin{description}
\item[{\tt accept}] Produce observers when absolutely needed.
More precisely, given memory location~\texttt{x}, no equality
on~\texttt{x} appears
in the final condition for zero or one write to~\texttt{x},
one such appears for two writes, and observers are produced for
three writes or more.
\item[{\tt avoid}] Never produce observers, \emph{i.e.} fail when there
are three writes to the same location.
\item[{\tt force}] Produce observers for two writes or more.
\item[{\tt local}] Always produce local observers.
\end{description}

With \diyone, one easily build a three writes test as for instance
the following~\ahref{W5.litmus}{\file{W5}}:
\begin{verbatim}
% diyone -obs accept -obstype fenced -arch PPC -name W5 Wse Wse PodWW Wse PodWW
% cat W5.litmus
PPC W5
"Wse Wse PodWW Wse PodWW"
{ 0:r2=y; 1:r2=y; 1:r4=x; 2:r2=x; 2:r4=y; 3:r2=y; }
 P0           | P1           | P2           | P3           ;
 lwz r1,0(r2) | li r1,3      | li r1,2      | li r1,2      ;
 sync         | stw r1,0(r2) | stw r1,0(r2) | stw r1,0(r2) ;
 lwz r3,0(r2) | li r3,1      | li r3,1      |              ;
 sync         | stw r3,0(r4) | stw r3,0(r4) |              ;
 lwz r4,0(r2) |              |              |              ;
exists (x=2 /\ 0:r1=1 /\ 0:r3=2 /\ 0:r4=3)
\end{verbatim}
As apparent from the code above, we have a fenced observer thread
on~\texttt{y} (\verb+P0+),
while the final state of~\texttt{x} is observed directly
(\verb+x=2+).
The command line switch \opt{-obs force} would yield two observers,
while \opt{-obs avoid} would lead to failure.

With command line switch \opt{-obs local} we get three local observations of coherence,
which suffice to reconstruct the complete coherence orders:
\begin{verbatim}
% diyone -obs local -obstype fenced -arch PPC -name W5Local Wse Wse PodWW Wse PodWW
chi% cat W5Local.litmus 
PPC W5Local
"Wse Wse PodWW Wse PodWW"
{
0:r2=x; 0:r4=y;
1:r2=y; 1:r4=x;
2:r2=x;
}
 P0           | P1           | P2           ;
 li r1,3      | li r1,2      | li r1,2      ;
 stw r1,0(r2) | stw r1,0(r2) | stw r1,0(r2) ;
 li r3,1      | li r3,1      | sync         ;
 stw r3,0(r4) | stw r3,0(r4) | lwz r3,0(r2) ;
 sync         | sync         |              ;
 lwz r5,0(r4) | lwz r5,0(r4) |              ;
exists (0:r5=2 /\ 1:r5=2 /\ 2:r3=3)
\end{verbatim}

\section{Command usage}

The~\diy{} suite consists in four main tools:
\begin{description}
\item[\prog{diyone}] generates one litmus test from the specification
of a violation of the sequential consistency memory model as a cycle
--- see~\mysec{\ref{diyone:intro}}.
\item[\prog{diycross}] generates variations of \diyone{} style tests
--- see~\mysec{\ref{diycross:intro}}.
\item[\prog{diy}] generates several tests, aimed at confirming
that candidate relaxations are relaxed or safe---see~\mysec{\ref{diy:intro}}.
\item[\prog{readRelax}] Extract relevant information from
the results of tests---see~\mysec{\ref{readRelax:intro}}.
\end{description}

\subsection{A note \label{sec:names}\label{naming}on test names}
We have designed a simple naming scheme for tests.
A normalised test name decomposes first as a family name,
and second as a description of program-order (or internal)
candidate relaxations.

\subsubsection{Family names}
Cycles (and thus tests) are first grouped by families.
Family names describe test structure,
based upon external communication candidates relaxations.
More specifically, external communication candidates relaxations
suffice to settle the directions (\texttt{W} or~\texttt{R}) of 
first and last events of threads,
considering the case when those two events are the same.
For instance, consider the cycle ``PodWW Rfe PodRR Fre'':
there are two threads
in the corresponding test (as there are two external communication candidate
relaxations), one thread starts and ends with a write (written \texttt{WW}),
while the other thread starts and ends with a read (written \texttt{RR}).
The family name is thus
\texttt{WW+RR}, (or \texttt{RR+WW}, but we choose the former).
For reference, a normalised family name is the minimal
amongst the representations of a given cycle, following the lexical order
derived from the order
$\texttt{W} < \texttt{WW} <
\texttt{RR} < \texttt{RW} < \texttt{WR} < \texttt{R}$.

The most common families have nicknames, which are defined
by \footahref{\urlfamilies}{this document}.
For instance, consider the test whose cycle is
``\ahref{SB.litmus}{PodWR  Fre PodWR Fre}''.
The family name is \texttt{WR+WR}, as
this is a two-thread test, both threads starting with
a write and ending with a read.
The nickname for this family is, as we already know, SB (store-buffering).
Here is the list of nicknames and family names for two thread tests:
\begin{center}
\begin{tabular}{ll|l}
\hline
2+2W & \texttt{WW+WW} & PodWW Wse PodWW Wse \\
LB  & \texttt{RW+RW} & PodRW Rfe PodRW Rfe \\
MP & \texttt{WW+RR} & PodWW Rfe PodRR Fre \\
R & \texttt{WW+WR} & PodWW Wse PodWR Fre \\
S & \texttt{WW+RW} & PodWW Rfe PodRW Wse \\
SB & \texttt{WR+WR} & PodWR Fre PodWR Fre \\
\hline
\end{tabular}
\end{center}

Isolated writes (and reads) originate from the combinations of
communication relaxations,
for instance [Fre,Rfe].
They appear as ``\texttt{W}'' (and~\texttt{R}) in family names.
For instance,  ``\ahref{tst-ppc/IRIW.litmus}{Rfe PodRR Fre Rfe PodRR Fre}''
contains two such isolated
writes, its name is thus \texttt{W+RR+W+RR} and its nickname is,
as we know, IRIW (Independent reads of independent writes).
The test ``\ahref{WRC.litmus}{Rfe PodRW Rfe PodRR Fre}''
contains one isolated write, as apparent from
this diagram:
\begin{center}\img{WRC}\end{center}
The family name is thus \texttt{W+RW+RR} and the nickname is
WRC (Write to Read Causality).

\subsubsection{Descriptive names for variants}
Every family has a prototype, homonymous test where every thread code
consists in one (for \texttt{W} or~\texttt{R}) or two memory accesses
to different locations (for \texttt{WW}, \texttt{WR} etc.).
For instance, the \ahref{MP.litmus}{\ltest{MP}} test
is derived from the cycle ``PodWW Rfe PodRR Fre''.
Variants are described by tags that illustrates the various
program-order relaxations: they appear after the family name, still with ``\texttt{+}'' as a separation.
For instance the test derived from ``LwSyncdWW Rfe DpAddrdR Fre''
is named \ahref{MP+lwsync+addr.litmus}{\ltest{MP+lwsync+addr}}.

When all threads have the same tag \emph{tag}, the test name is abbreviated
as \texttt{\emph{Family}+\emph{tag}s}. For instance,
the test \ltest{MP+lwsync+lwsync} (``LwSyncdWW Rfe LwSyncdRR Fre'') is in fact
\ahref{MP+lwsyncs.litmus}{\ltest{MP+lwsyncs}}.
Additionally, the tag \texttt{pos} (all \texttt{po}'s) is omitted,
in order to yield family names for the prototype tests --- cf.
 \atest{\ltest{MP}} whose name would have been
\texttt{MP+pos} otherwise.

For the sake of terseness, tags do not describe program-order relaxations
completely. For instance both DpAddrdR and DpAddrdW (address dependency to
read and write, respectively) have the same tag, \texttt{addr}.
It does not harm for simple tests, as the missing direction can be inferred from
the family name. Consider for instance
\ahref{MP+lwsync+addr.litmus}{\ltest{MP+lwsync+addr}} and
\ahref{LB+lwsync+addr.litmus}{\ltest{LB+lwsync+addr}}.
\begin{center}
\cycle{MP+lwsync+addr}\qquad\cycle{LB+lwsync+addr}
\end{center}

The naming scheme extends to cycles with consecutive program-order relaxations,
by separating tags with ``\texttt{-}'' when they follow one another:
for instance ``LwSyncdWW Rfe DpAddrdR PodRR Fre'' is named
\ltest{MP+lwsync+addr-po}. Unfortunately, the current naming scheme falls short
in supplying non-ambiguous names to all tests.
For instance,
``LwSyncdWW Rfe DpAddrdW PodWR Fre'' is also named \ltest{MP+lwsync+addr-po}.
In that situation tools will either fail or silently add a numeric suffix,
depending on the boolean \texttt{-addnum} option.
\begin{verbatim}
% diycross -addnum false LwSyncdWW Rfe [DpAddrdR,PodRR],[DpAddrdW,PodWR] Fre
Fatal error: Duplicate name MP+lwsync+addr-po
% diycross -addnum true LwSyncdWW Rfe [DpAddrdR,PodRR],[DpAddrdW,PodWR] Fre
Generator produced 2 tests
% cat @all
# diycross -addnum true LwSyncdWW Rfe [DpAddrdR,PodRR],[DpAddrdW,PodWR] Fre
MP+lwsync+addr-po.litmus
MP+lwsync+addr-po001.litmus
\end{verbatim}
As a result, we get the two tests:
\atest{MP+lwsync+addr-po} and~\atest{MP+lwsync+addr-po001}.
\begin{center}
\cycle{MP+lwsync+addr-po}\qquad\cycle{MP+lwsync+addr-po001}
\end{center}

Future versions of \prog{diy} may solve this issue in a more satisfying manner.
At the moment, users are advised not to rely too much
on the automatic naming scheme.
Users may name tests in a non-ambiguous fashion by (1) specifying
an explicit family name  (\texttt{-name \emph{name}}) and~(2)
selecting the numeric scheme (\texttt{-num true}):
\begin{verbatim}
% diycross -name MP+X -num true LwSyncdWW Rfe [DpAddrdR,PodRR],[DpAddrdW,PodWR] Fre
Generator produced 2 tests
\end{verbatim}
The \prog{diycross} generator outputs the same tests as above, with
names \atest{MP+X000} and~\atest{MP+X001}.
\ifhevea
\begin{center}
\cycle{MP+X000}\qquad\cycle{MP+X001}
\end{center}
\fi



\subsection{Common options}
All \prog{diy} test generators accept the following documented
command-line options:
\begin{description}
\item[{\tt -v}] Be verbose, repeat to increase verbosity.
\item[{\tt -version}]  Show version number and exit.
\item[{\tt -arch (X86|PPC|ARM)}] Set architecture. Default is \texttt{PPC}.
ARM support is experimental.
\item[{\tt -o <dest>}]
Redirect output to \opt{<dest>}.
This option applies when tools generate a set of tests and
an index file \texttt{@all}, \emph{.i.e.} in all situations
except for \prog{diyone} simplest operating mode.

If  argument \opt{<dest>} is an archive (extension \opt{.tar})
or a compressed archive (extension \opt{.tgz}),
the tool builds an archive.
Otherwise, \opt{<dest>} is interpreted as the name of an
existing directory.
Default is ``\opt{.}'', that is tool output goes into the
current~directory.
\item[{\tt -obs (accept|avoid|force|local)}] Management of observers,
default is \opt{avoid}. See Sec.~\ref{sec:obs}.
\item[{\tt -obstype (fenced|loop|straight)}] 
Style of observers, default is \opt{fenced}. See Sec.~\ref{sec:obstype}.
\item[{\tt -cond (cycle|uni|observe)}]
Control final condition of tests, default is \opt{cycle}.
In mode \opt{cycle}, the final condition identifies executions
that correspond to the generating cycle.
In mode \opt{unicond}, the final condition identifies executions
that are valid w.r.t. the uniproc model (see Sec.~\ref{defuniproc}).
In mode  \opt{observe} there is no final condition:
the \prog{litmus} and \prog{herd} tools will simply list the final
values of locations.
\item [{\tt -optcond}] Optimise conditions by disregarding the values
of loads that are neither the target of~Rf, nor the source of~Fr. This is
the default.
\item [{\tt -nooptcond}] Do not optimise conditions.
\item [{\tt -optcoherence}] Optimise conditions assuming that
the tested system (at least) follows the uniproc model
(see Sec.~\ref{defuniproc}).
\item [{\tt -nooptcoherence}] Do not optimise conditions assuming that
the tested system (at least) follows the uniproc model. This is the default.
\item [{\tt -neg <bool>}] Negate final condition, default is~\texttt{false}.
\item[{\tt -c <bool>}] Avoid equivalent cycles. Default is~\opt{true}.
Setting \opt{-c true} is intended for debug.
\end{description}

The naming of tests is controlled by the following options:
\begin{description}
\item[{\tt -name <name>}] Use name for naming tests, the exact consequences
depend on the generator. By default the generator has no name available.
\item[{\tt -num <bool>}] Use numeric names, \emph{i.e.} from a base name
<base> the generator will name tests as \texttt{<base>000}, \texttt{<base>001} etc. The default depends upon the generator.
\item[{\tt -addnum <bool>}] If true, when faced with tests whose name
\texttt{<name>} has already been given, use names 
\texttt{<name>001}, \texttt{<name>002}, etc.
Otherwise fail in the same situation.
The default depends upon the generator.
\item[{\tt -fmt <n>}] Size of numerical suffixes, default is 3.
\end{description}

\subsection{Usage\label{diyone:usage} of \prog{diyone}}
The tool \aname{diyone}{\prog{diyone}}
has two operating modes.
The selected mode depends on the presence of command-line arguments,

In the first operating mode, \prog{diyone}
takes a non-empty list of candidate relaxations as arguments
and outputs a litmus test.
Note that \prog{diyone} may fail to produce the test, with a message
that briefly details the failure.
\begin{verbatim}
% diyone Rfe Rfe PodRR
Test a [Rfe Rfe PodRR] failed:
Impossible direction PodRR Rfe
\end{verbatim}

In this mode, {\tt -name <name>} sets the name of the test to \texttt{<name>}
and output it into file \texttt{<name>.litmus}.
If absent, the test name is \texttt{A} and output goes to standard output.


Otherwise, \emph{i.e.} when there are no command-line arguments,
\prog{diyone} reads the standard input and generates the tests described
by the lines it reads.
Each line starts with a test name~\textit{name},
followed by ``\texttt{:}'', followed
by a list of candidate relaxations~\textit{RS}.
Then, \prog{diyone} acts as if invoked as \texttt{diyone}~\textit{opts}
\texttt{-name}~\textit{name}~\textit{RS}.

\aname{diyone:norm}{The tool}
\prog{diyone} accepts the following documented option:
\begin{description}
\item[{\tt -norm}] Normalise tests and give them normalised names.
In the first operating mode (when a cycle is explicitly given) the test
will be named with a family name and a descriptive name.
In the second operating mode,
numeric names are used, base being either given explicitly
(with option \texttt{-name <base>}) or being a normalised family name.
\end{description}


\subsection{Usage of \diycross}
\diycross{} produces several tests by ``cross~producing'' lists
of candidate relaxations given as arguments, see
Sec~\ref{diycross:intro}.
\diycross{} also produces an index file \texttt{@all} that lists all produced
litmus source files.

If option \opt{-name <name>} is given, it sets the family name of
generated tests, otherwise standard family names are used
(cf. Sec.~\ref{sec:names}).
By default descriptive names are used (\emph{i.e.} \texttt{-num false})
and \prog{diycross} will fail if two different tests have the same name
(\emph{i.e.} \texttt{-addnum false}):
\begin{verbatim}
% diycross PodWW Rfe [DpAddrdR,PodRR],[DpAddrdW,PodWR] Fre
Fatal error: Duplicate name MP+po+addr-po
\end{verbatim}
Should this happen users can resort either to numeric names,
\begin{verbatim}
%diycross -num true PodWW Rfe [DpAddrdR,PodRR],[DpAddrdW,PodWR] Fre
Generator produced 2 tests
con% ls
@all  MP000.litmus  MP001.litmus
\end{verbatim}
or to disambiguating numeric suffixes.
\begin{verbatim}
%diycross -addnum true PodWW Rfe [DpAddrdR,PodRR],[DpAddrdW,PodWR] Fre
Generator produced 2 tests
con% ls
@all  MP+po+addr-po001.litmus  MP+po+addr-po.litmus
\end{verbatim}


\subsection{Usage\label{diy:usage} of {\diy}}
As \diycross, \diy{} produce several files, hence naming issues are critical.
By default, \diy{} uses family names and the numeric naming scheme
(\texttt{-num true}).
Users can specify a family name~\emph{family}
for all tests with \texttt{-name \emph{family}},
or attempt using the descriptive names of Sec~\ref{sec:names}
with \texttt{-num false}.
Moreover, \diy{} produces an index file \file{@all} that lists
the file names of all tests produced.

The tool \diy{} also accepts the following, additional, documented options.
\begin{description}
\item[{\tt -conf <file>}] Read configuration file \opt{<file>}.
A configuration file consists in a list of options, one option per line.
Lines introduced by \verb+#+ are comments and are thus ignored.
\item[{\tt -size <n>}] Set the maximal size of cycles. Default is~$6$.
\item[{\tt -exact}] Produce cycles of size exactly~\opt{<n>},
in place of size up to~\opt{<n>}.
\item[{\tt -nprocs <n>}] Reject tests with more than \opt{<n>} threads.
Default is~$4$.
\item[{\tt -eprocs}] Produce tests with exactly \opt{<n>}
threads, where \opt{<n>} is set above.
\item[{\tt -ins <n>}] Reject tests as soon as the code of one thread
originates from \opt{<n>} edges or more. Default is~$4$.
\item[{\tt -relax <relax-list>}] Set relax list. Default is empty.
The syntax of \opt{<relax-list>} is a comma (or space)
separated list of candidate relaxations.
\item[{\tt -mix <bool>}] Mix the elements of the relax list
(see below), default \opt{false}.
\item[{\tt -maxrelax <n>}]
In mix mode, upper bound on the number of different candidate
relaxations tested together.
Default is~$100$
\item[{\tt -safe <relax-list>}] Set safe list. Default is empty.
\item[{\tt -mode (critical|sc|free|uni)}]
Control generation  of cycles, default~\opt{sc}.
Those tags command the activation of some constraints over cycle
generation, see \ahrefloc{mode:describe}{below}.

\item[{\tt -cumul <bool>}]
Permit implicit cumulativity,
\emph{i.e.} authorise building up the sequence Rfe followed by a fence,
or the reverse. Default is~\opt{true}.
\end{description}
The relax and safe lists command the generation of
cycles as follows:
\begin{enumerate}
\item When the relax list is empty,
cycles are built from the candidate relaxations of the safe list.
\item When the relax list is of size~$1$,
cycles are built from its single element~$r$ and from the elements of
the safe list. Additionally, the cycle produced contains~$r$ at least once.
\item
When the relax list is of size~$n$, with~$n > 1$,
the behaviour of~\diy{} depends on the mix mode:
\begin{enumerate}
\item 
By default (\opt{-mix false}),
\diy{} generates~$n$ independent sets of cycles,
each set being built with one relaxation from the relax list and all
the relaxations in the safe list.
In other words, \diy{} on a relax list of size~$n$ behaves similarly
to $n$~runs of~\diy{} on each candidate relaxation in the list.
\item Otherwise (\opt{-mix true}), \diy{} generates cycles that contains
at least one element from the relax list, including some cycles
that contain different relaxations from the relax list.
The cycles will contain at most~$m$ different elements from the relax list,
where $m$ is specified with option~``\opt{-maxrelax}~$m$''.
\end{enumerate}
\end{enumerate}

\label{mode:describe}Generally
speaking, \diy{} generates ``some'' cycles and does not generate
``all'' cycles (up to a certain size \emph{e.g.}).
In (default) sc mode, \diy{} performs some optimisation,
most of which we leave unspecified.
\label{compose:com}As an exception to
this non-specification, \diy{} in \opt{sc} (default) mode
is guaranteed not to
generate redundant elementary communication relaxation in the following sense:
let us call Com the union of Ws, Rf and Fr (the e\vbar{}i specification
is irrelevant here).
Ws being transitive and by definition of Fr,
one easily shows that the transitive closure Com+ of Com is the union
of Com plus [Ws,Rf] (Ws followed by Rf) plus [Fr,Rf].
As a consequence, maximal subsequences of communication
relaxations in \diy{} cycles are limited
to  single relaxations (\emph{i.e.} Ws, Rf and Fr)
and to the above mentioned two sequences
(\emph{i.e.} [Ws,Rf] and  [Fr,Rf]).
For instance, [Ws,Ws] and [Fr,Ws] should never appear in \diy{} generated
cycles.
However, such subsequences can be generated on an individual basis with
\diyone, see the example of \file{W5} in Sec~\ref{sec:obs}.

\label{critical:def}%
In critical mode (\opt{-mode critical}),  cycles are strictly specified as
follows:
\begin{enumerate}
\item Communication candidate relaxations sequences are
limited to Rf,Fr,Ws,[Ws,Rf] and~[Fr,Rf], as in sc mode.
\item No two internal\footnote{That is, the source and target
accesses are by the same processor.} candidate relaxations follow one another.
\item If the option \opt{-cumul false} is specified,
\diy{} will not construct
the sequence of~Rfe followed by a fence (or B-cumulativity) candidate
relaxation or
of a fence (or A-cumulativity) candidate relaxation followed by Rfe.
\item Cycles that access one single memory location are rejected.
\item None of the rules above applies to the internal sequences of
composite candidate relaxations. For instance, if [Rfi,PodRR]
is given as a candidate relaxation, the sequence ``Rfi,PodRR'' appears
in cycles.
\end{enumerate}
The cycles described above are the \emph{critical} cycles of~\cite{ss88}.

\label{free:def}%
In free mode (\opt{-mode free}), cycles are strictly specified as
follows:
\begin{enumerate}
\item Communication candidate relaxations sequences are
limited to Rf,Fr,Ws,[Ws,Rf] and~[Fr,Rf]. However, arbitrary sequences
of communication candidates are accepted when they are internal and external
or external and internal.
\item Cycles that access one single memory location are rejected.
\end{enumerate}

\label{uni:def}Finally, the \opt{uni} mode enforces the following constraints
on cycles:
\begin{enumerate}
\item Sequences of communication  candidate relaxations
are restricted in the same manner as for
\opt{free} mode (see \ahrefloc{free:def}{above}).
\item Sequences of Po candidate relaxation are rejected.
\end{enumerate}

\subsection{\aname{readRelax}{Usage} of \prog{readRelax}}
\prog{readRelax} is a simple tool to extract relevant information
out of \litmus{} run logs of tests produced by the \diy{} generator.
For a given run of a given litmus test, the relevant information is:
\begin{itemize}
\item Whether the test yielded \texttt{Ok} or not,
\item An optional candidate relaxation, which is the one
given as argument to \diy~option
\opt{-relax} at test build time, or none.
\item The safe list relevant to the given test, \emph{i.e.}
the safe candidate relaxations that appear in the tested cycle.
\end{itemize}
See \mysec{\ref{readRelax:intro}} for an example.

The tool \prog{readRelax} takes file names as arguments.
If no argument is present, it reads a list
of file names on standard input, one name per line.

\section{Additional tools: extracting cycles and classification}
When non-standard family names or numeric names are used, it proves convenient
to rename tests with the standard naming scheme.
We provide two tools to do so: \prog{mcycles} that extracts cycles from
litmus source files and \prog{classify} that normalises and renames cycles.

For instance, one can use \diy{} to generate all simple, critical, tests up
to three threads for X86 with the following configuration file~\afile{X.conf}
\verbatiminput{X.conf}
\begin{verbatim}
% diy -conf X.conf
Generator produced 23 tests
% ls
@all         X003.litmus  X007.litmus  X011.litmus  X015.litmus  X019.litmus  X.conf
X000.litmus  X004.litmus  X008.litmus  X012.litmus  X016.litmus  X020.litmus
X001.litmus  X005.litmus  X009.litmus  X013.litmus  X017.litmus  X021.litmus
X002.litmus  X006.litmus  X010.litmus  X014.litmus  X018.litmus  X022.litmus
\end{verbatim}
Cycles are extracted with \prog{mcycles}, which takes the index file \texttt{@all} as argument:
\begin{verbatim}
% mcycles @all
X000: Wse PodWR Fre PodWR Fre PodWW
X001: Rfe PodRR Fre PodWR Fre PodWW
X002: Wse PodWR Fre PodWW
X003: Wse PodWW Wse PodWR Fre PodWW
X004: Rfe PodRW Wse PodWR Fre PodWW
X005: Rfe PodRR Fre PodWW
X006: Wse PodWW Rfe PodRR Fre PodWW
X007: Rfe PodRW Rfe PodRR Fre PodWW
X008: Wse Rfe PodRR Fre PodWW
X009: Wse PodWW Wse PodWW
...
\end{verbatim}
The output of \prog{mcycles} can be piped into \prog{classify} for family
classification:
\begin{verbatim}
% mcycles @all | classify -arch X86
2+2W
  X009 -> 2+2W : PodWW Wse PodWW Wse
3.2W
  X010 -> 3.2W : PodWW Wse PodWW Wse PodWW Wse
3.LB
  X020 -> 3.LB : PodRW Rfe PodRW Rfe PodRW Rfe
3.SB
  X016 -> 3.SB : PodWR Fre PodWR Fre PodWR Fre
ISA2
  X007 -> ISA2 : PodWW Rfe PodRW Rfe PodRR Fre
LB
  X019 -> LB : PodRW Rfe PodRW Rfe
MP
  X005 -> MP : PodWW Rfe PodRR Fre
...
\end{verbatim}
Notice that \prog{classify} accepts the \opt{arch} option, as it needs
to parse cycles.

Finally, one can normalise tests, using normalised names by piping
\prog{mcycles} output into \prog{diyone} with options
\ahrefloc{diyone:norm}{\opt{-norm -num false}}:
\begin{verbatim}
% mkdir src
% mcycles @all | diyone -arch X86 -norm -num false -o src
Generator produced 23 tests
% ls src
2+2W.litmus  @all         R.litmus    WRC.litmus     WRW+WR.litmus  Z6.2.litmus
3.2W.litmus  ISA2.litmus  RWC.litmus  WRR+2W.litmus  WWC.litmus     Z6.3.litmus
3.LB.litmus  LB.litmus    SB.litmus   WRW+2W.litmus  Z6.0.litmus    Z6.4.litmus
3.SB.litmus  MP.litmus    S.litmus    W+RWC.litmus   Z6.1.litmus    Z6.5.litmus
\end{verbatim}
Alternatively, one may instruct \prog{classify} to produce output for
\diyone. In that case one should pass option \opt{-diyone} to \prog{classify}
so as to instruct it to produce output that is parsable by \prog{diyone}:
\begin{verbatim}
% rm -rf src && mkdir src
% mcycles @all | classify -arch X86 -diyone | diyone -arch X86 -o src
Generator produced 23 tests
% ls src
2+2W.litmus  @all         R.litmus    WRC.litmus     WRW+WR.litmus  Z6.2.litmus
3.2W.litmus  ISA2.litmus  RWC.litmus  WRR+2W.litmus  WWC.litmus     Z6.3.litmus
3.LB.litmus  LB.litmus    SB.litmus   WRW+2W.litmus  Z6.0.litmus    Z6.4.litmus
3.SB.litmus  MP.litmus    S.litmus    W+RWC.litmus   Z6.1.litmus    Z6.5.litmus
\end{verbatim}


\subsection{Usage of \prog{mcycles}}
The tool \prog{mcycles} has no options and takes litmus source files or
index files as arguments.
It outputs a list of lines to standard output.
Each line starts with a test name, suffixed by ``\texttt{:}'', then
the cycle of the named test. Notice that this format is the input format
to \prog{diyone} in its second operating mode --- see Sec.~\ref{diyone:usage}.

It is important to notice that, for \prog{mcycles} to extract cycles, those
must be present as meta-information in source files.
In practice, this means that  \prog{mcycles} operates normally on
sources produced by \diyone, \diycross{} and~\diy.
Moreover only one instance of a given cycle will be output.


\subsection{Usage of \prog{classify}}
The tool \prog{classify} reads its standard input, interpreting is as
a list of cycles in the output format of \prog{mcycles}.
It normalises and classifies those cycles.
The tool \prog{classify} accepts the following documented options:
\begin{description}
\item[{\tt -arch (X86|PPC|ARM)}] Set architecture. Default is \texttt{PPC}.
ARM support is experimental.
\item[{\tt -u}] Instruct \prog{classify} to fail when two tests have the same
normalised name. Otherwise \prog{classify} will output one line per test,
regardless of duplicate names.
\item[{\tt -diyone}] Output a normalised list of names and cycles,
which is legal input for \prog{diyone}.
\end{description}

\endinput


\clearpage
\part{Simulating\label{part:herd} memory models with \herd}
\cutname{herd.html}

The tool \herd{} is a memory model simulator.
Users may write simple, single events,
axiomatic models of their own and run litmus tests on top
of their model.
The \herd{} distribution already includes some models.

The authors of~\herd{} are Jade Alglave and Luc Maranget.


\section{Writing simple models}
This section introduces \cat{}, our language for describing memory models.
The \cat{} language is a domain specific language for writing and executing
memory models. From the language perspective, \cat{} is loosely inspired
by OCaml. That is, it is a functional language, with similar syntax
and constructs.
The basic values of \cat{} are sets of events, which include memory events
but also additional events such as fence events,
and relations over events.


\subsection{Sequential consistency}
The simulator \herd{} accepts models written in text files.
For instance here is \afile{sc.cat},
the definition of the sequentially consistent (SC) model in the partial-order
style:
\verbatiminput{sc.cat}
The model above illustrates some features of model definitions:
\begin{enumerate}
\item A model file starts with a tag (here \verb+SC+), which can also be a
string (in double quotes) in case the tag includes special characters or spaces.
\item Pre-defined bindings. Here \verb+po+ (program order)
and \texttt{rf} (read from) are pre-defined.
The remaining two communication relations (\texttt{co} and~\texttt{fr})
are computed by the included file \verb+cos.cat+, which we describe later
--- See Sec.~\ref{sec:cos}.
For simplicity, we may as well assume that \verb+co+
and~\verb+fr+ are pre-defined.
\item The computation of new relations from other relations,
and their binding to a name with the \verb+let+ construct.
Here, a new relation \verb+com+ is the union ``\texttt{|}'' of
the three pre-defined communication relations.
\item The peformance of some checks. Here the relation ``\verb+po | com+''
(\emph{i.e.} the union of program order \textrel{po} and of  communication
relations) is required to be acyclic.
Checks can be given names by suffixing them with
``\texttt{as~}\textit{name}''.
This last feature will be used in Sec.~\ref{name:check}
\end{enumerate}
%We postpone the discussion of the \verb+show+ instruction, see
%Sec.~\ref{sec:show}.

One can then run some litmus test, for instance \atest{SB}
(for \emph{Store Buffering},
see also Sec.~\ref{litmus:simple}), on top of the SC model:
\begin{verbatim}
% herd -model ./sc.cat SB.litmus
Test SB Allowed
States 3
0:EAX=0; 1:EAX=1;
0:EAX=1; 1:EAX=0;
0:EAX=1; 1:EAX=1;
No
Witnesses
Positive: 0 Negative: 3
Condition exists (0:EAX=0 /\ 1:EAX=0)
Observation SB Never 0 3
Hash=7dbd6b8e6dd4abc2ef3d48b0376fb2e3
\end{verbatim}
The output of \herd{} mainly consists in
the list of final states that are allowed by the simulated model.
Additional output relates to the test condition.
One sees that the test condition does not validate on top of SC,
as ``\texttt{No}'' appears just after the list of final states
and as there is no ``Positive'' witness.
Namely, the condition ``\verb+exists (0:EAX=0 /\ 1:EAX=0)+''
reflects a non-SC behaviour, see Sec.~\ref{intro:candidate}.

\label{intro:candidate}%
The simulator \herd{} works by generating all candidate executions
of a given test.
By ``candidate execution'' we mean a choice of events,
program order~\textrel{po}, of the read-from relation~\texttt{rf}
and of final writes to memory
(last write to a given location)\footnote{Alternatively,
we may adopt the simpler view that
a candidate execution includes a choice of all communication relations.}.
In the case of the \ltest{SB} example, we get the following four executions:
\begin{center}
\img{SB-00}\quad\quad
\img{SB-01}\quad\quad
\img{SB-02}\quad\quad
\img{SB-03}
\end{center}
Indeed, there is no choice for the program order \textrel{po}, as there are no
conditional jumps in this example; and  no choice for the final
writes either, as there is only one store per location, which
must be \textrel{co}-after the initial stores (pictured as small red dots).
Then, there are two read events from locations $x$ and~$y$ respectively,
which take their values either from the initial stores or from
the stores in program. As a result, there are four possible executions.
The model \afile{sc.cat} gets executed on each of the four
candidate executions. The three first executions
are accepted and the last one is rejected, as it presents a cycle
in \texttt{po | fr}.
On the following diagram,
the cycle  is obvious:
\begin{center}\img{SB+SC}\end{center}

\subsection{Total Store Order (TSO)}
However, the non-SC execution \ahrefloc{x86:classic}{shows up} on x86 machines,
whose memory model is TSO. As TSO relaxes the write-to-read order, we attempt
to write a TSO model \afile{tso-00.cat}, by simply removing write-to-read
pairs from the acyclicity check:
\verbatiminput{tso-00.cat}
This model illustrates several features
of model definitions:
\begin{itemize}
\item New predefined sets: \verb+W+, \verb+R+ and~\verb+M+, which are
the sets of read events, write events and of memory events, respectively.
\item The cartesian product operator ``\verb+*+'' that returns the cartesian
product of two event sets as a relation.
\item The intersection operator ``\verb+&+'' that operates on sets and
relations.
\end{itemize}
As a result, the effect of the declaration
\verb+let po-tso = po & (W*W | R*M)+ is to define \verb+po-tso+
as the program order on memory events minus write-to-read pairs.

We run \atest{SB} on top of the tentative TSO model:
\begin{verbatim}
% herd -model tso-00.cat SB.litmus 
Test SB Allowed
States 4
0:EAX=0; 1:EAX=0;
0:EAX=0; 1:EAX=1;
0:EAX=1; 1:EAX=0;
0:EAX=1; 1:EAX=1;
Ok
Witnesses
Positive: 1 Negative: 3
...
\end{verbatim}
\label{sb:image}The non-SC behaviour is now accepted, as write-to-read \textrel{po}-pairs
do not participate to the acyclicity check any more. In effect, this allows
the \ahref{SB-03.png}{last execution} above,
as $\textrel{ghb}$ (\emph{i.e.}
\verb+po-tso | com-tso+) is acyclic.
\begin{center}\img{SB+TSO}\end{center}


However,
our model \afile{tso-00.cat} is flawed: it is still to strict,
forbidding some behaviours that the TSO model should accept.
Consider the test \atest{SB+rfi-pos},
which is test \atest{STFW-PPC} for X86 from Sec.~\ref{stfw} with a normalised name (see Sec.~\ref{sec:names}).
This test targets the following execution:
\begin{center}\img{SB+rfi-pos}\end{center}
Namely the test condition
\verb+exists (0:EAX=1 /\ 0:EBX=0 /\ 1:EAX=1 /\ 1:EBX=0)+
specifies that Thread~$0$ writes~$1$ into location~$x$,
reads the value $1$~from the location~$x$ (possibly by store forwarding) and
then reads the value~$0$ from the location~$y$;
while Thread~$1$ writes~$1$ into~$y$,
reads~$1$ from~$y$ and then reads~$0$ from~$x$.
Hence, this test derives from the previour~\atest{SB}
by adding loads in the middle, those loads
being  satisfied from local stores.
As can be seen by running the test on top of the \afile{tso-00.cat}
model, the target execution is forbidden:
\begin{verbatim}
% herd -model tso-00.cat SB+rfi-pos.litmus 
Test SB+rfi-pos Allowed
States 15
0:EAX=0; 0:EBX=0; 1:EAX=0; 1:EBX=0;
...
0:EAX=1; 0:EBX=1; 1:EAX=1; 1:EBX=1;
No
Witnesses
Positive: 0 Negative: 15
..
\end{verbatim}
However, running the test with litmus demonstrates that the behaviour
is observed on some X86 machine:
\begin{verbatim}
% arch
x86_64
% litmus -mach x86 SB+rfi-pos.litmus
...
Test SB+rfi-pos Allowed
Histogram (4 states)
11589 *>0:EAX=1; 0:EBX=0; 1:EAX=1; 1:EBX=0;
3993715:>0:EAX=1; 0:EBX=1; 1:EAX=1; 1:EBX=0;
3994308:>0:EAX=1; 0:EBX=0; 1:EAX=1; 1:EBX=1;
388   :>0:EAX=1; 0:EBX=1; 1:EAX=1; 1:EBX=1;
Ok

Witnesses
Positive: 11589, Negative: 7988411
Condition exists (0:EAX=1 /\ 0:EBX=0 /\ 1:EAX=1 /\ 1:EBX=0) is validated
...
\end{verbatim}
As a conclusion, our tentative TSO model is too strong.
The following diagram pictures its \textrel{ghb} relation:
\begin{center}\img{SB+rfi-pos+TER}\end{center}
One easily sees that \textrel{ghb} is cyclic, wheras it should not.
Namely, the internal read-from relation~\textrel{rfi} does
not create global order in the TSO model.
Hence, \textrel{rfi} is not included in \textrel{ghb}.
We rephrase our tentative TSO model, resulting into the new model
\afile{tso-01.cat}:
\verbatiminput{tso-01.cat}
As can be observed above \texttt{rfi} (internal read-from) is no longer
included in \textrel{ghb}. However, \texttt{rfe} (external read-from)
still is. Notice that \texttt{rfe} and~\texttt{rfi} are pre-defined.


As intended, this new tentative TSO model allows the behaviour of test~\atest{SB+rfi-pos}:
\begin{verbatim}
%  herd -model tso-01.cat SB+rfi-pos.litmus
Test SB+rfi-pos Allowed
States 16
...
0:EAX=1; 0:EBX=1; 1:EAX=1; 1:EBX=0;
...
Ok
Witnesses
Positive: 1 Negative: 15
...
\end{verbatim}
And indeed, the global-happens-before relation is no-longer cyclic:
\begin{center}\img{SB+rfi-pos+BIS}\end{center}


We are not done yet, as our model is too weak in two aspects.
First, it has no semantics for fences.
As a result the test \atest{SB+mfences} is allowed, whereas it should
be forbidden, as this is the very purpose of the fence \texttt{mfence}.
\begin{center}\img{SB+mfences}\end{center}
One easily solves this issue by first defining the \verb+mfence+
that relates events with a \verb+MFENCE+ event \texttt{po}-in-between them;
and then by adding \verb+mfence+ to the definition of \verb+po-tso+:
\begin{verbatim}
let mfence = po & (_ * MFENCE) ; po
let po-tso = po & (W*W | R*M) | mfence
\end{verbatim}
Notice how the relation \verb+mfence+ is defined from two pre-defined sets:
``\verb+_+'' the universal set of all events and \verb+MFENCE+ the set
of fence events generated by the X86 \texttt{mfence} instruction.
An alternative, more precise definition, is possible:
\begin{verbatim}
let mem-to-mfence = po & M * MFENCE
let mfence-to-mem = po & MFENCE * M
let mfence = mem-to-mfence; mfence-to-mem
\end{verbatim}
This alternative definition of \texttt{mfence},
although yielding a smaller relation, is equivalent to the original one
for our purpose of checking \texttt{ghb} acyclicity.

But the resulting model is still too weak,
as it allows some behaviours that any model must
reject for the sake of single thread correctness.
The following test \atest{CoRWR} illustrates the issue:
\verbatiminput{CoRWR.litmus}
The test final condition targets the following excution candidate:
\begin{center}\img{CoRWR}\end{center}
The TSO check ``\verb+acyclic po-tso|com-tso+'' does not suffice to reject
two absurd behaviours pictured in the execution diagram above:
(1) the read~$a$ is allowed to
read from the \textrel{po}-after write~$b$, as \textrel{rfi} is not included
in \textrel{com-tso}; and~(2)
the read~$c$ is allowed to read the initial value of location~$x$
although the initial write~$d$ is \textrel{co}-before the write~$b$,
since \verb+po & (W * R)+ is not in \textrel{po-tso}.

\label{defuniproc}For any model, we rule out those very
untimely behaviours by the so-called
\textsc{uniproc}
check that states that executions projected on events that access one variable
only are SC.
In practice, having defined \verb+po-loc+ as \verb+po+ restricted to
events that touch the same address (\emph{i.e.}
as \verb+po & loc+), we further require the acyclicity
of the relation \verb+po-loc|fr|rf|co+.
In the TSO case, the \textsc{uniproc}~check can be
somehow simplified by considering only
the cycles in \verb+po-loc|fr|rf|co+ that 
are not already rejected by the main check of the model.
This amounts to design specific checks for the two relations that are
not global in TSO: \verb+rfi+ and \verb+po & (W*R)+.
Doing so, we finally produce a correct model for TSO \afile{tso-02.cat}:
\verbatiminput{tso-02.cat}
This last model illustrates another feature of \cat{}:
\herd{} may also performs irreflexivity checks with the keyword
``\verb+irreflexive+''.

\subsection{Sequential consistency, total order definition}
We now illustrate another style of model.
We consider the original definition of sequential consistency~\cite{lam79}.
An execution is SC when there exists a total order~\verb+S+ on events such that:
\begin{enumerate}
\item $S$ includes the program order~\verb+po+;
\item \label{rfcond}and read events read from the most recent write events in the past,
\emph{i.e.} a read~$r$ from location~$x$ reads the value stored by
the \verb+S+-maximal write amongst those writes to location~$x$
that are \verb+S+ smaller than~$r$.
\end{enumerate}
So we could just generate all total orders amongst memory events,
and filter those ``scheduling order candidates'' according to the two rules
above.

\label{sec:final:initial}Things are a bit more complex in~\herd{}, due to the presence of initial and final writes.
Up to now we have ignored initial and final writes, we are now going to
integrate them explicitly.

Initial writes are write events that initialise the memory locations.
Initial writes are not generated by the instructions of the test.
Instead, they are created by \herd{} machinery, and are available
from model text as the set \verb+IW+.


Final writes may be  generated by program instructions, and, when such,
they must be ordered by~$S$.
A final write is a write to a phantom read performed once
program execution is over.
The constraint on final writes
originates from \herd{} technique to enumerate execution candidates:
actual execution candidates also include a choice of final writes for
the locations that are observed in the test final condition\footnote{Doing
so permits pruning executions that are irrelevant to the test final condition,
see \herd{} option \ahrefloc{speedcheck:opt}{\opt{-speedcheck}}}.
As test outcome (\emph{i.e.} the final values of observed locations) is
settled before executing the model, it is important \emph{not} to accept
executions that yield a different outcome. Doing so may validate outcomes
that should be rejected.
In practice, the final write $w_f$
to location~$x$ must follow all other writes to~$x$ in~$S$.
Considering that the set of final writes is available to \cat{}~models
as the pre-defined set~\verb+FW+,
the  constraint on final writes
can be expressed as a relation:
\begin{verbatim}
let preSC = loc & (W \ FW) * FW
\end{verbatim}
Where \verb+loc+ is a predefined relation that relates all events
that access the same location.

By contrast with final writes, initial writes are not generated
by program instructions, and it is possible not to order them completely.
In particular, it is not useful to order initial writes to different locations,
nor the initial write to location~$x$ with any access to location~$y$.
Notice that we could include initial writes in~$S$ as we did for
final writes. Not doing so will improve efficiency.

\label{intro:linearisations}%
Finally, the order~$S$ is not just any order on memory events (predefined
set~\verb+M+, which includes initial and final writes writes),
it is a topological order of the program events (implemented as the set
\verb+M\IW+) that extends the pre-order~\verb+preSC+.
We can generate all such topological orders with the \cat{}~primitive
\verb+linearisations+:
\begin{verbatim}
let allS = linearisations(M\IW,preSC)
\end{verbatim}
The call \texttt{linearisation($E$,$r$)}, where $E$ is a set of events
and~$r$ is a relation on events, returns the set of all total
orders defined on~$S$ that extend~$r$. Notice that if~$r$ is cyclic,
the empty set is  returned.

\label{intro:with}We now need to iterate over the set~\verb+allS+.
We do so with the \verb+with+ construct:
\begin{verbatim}
with S from allS
\end{verbatim}
It is important to notice that the construct above extends the current
execution candidate (\emph{i.e.} a choice of events, plus a choice of
two relations~\texttt{po} and~\texttt{rf}) with a candidate order~$S$.
In other words, the scope of the iteration is the remainder of the model text.
Once model execution terminates for a choice of~$S$
(some element of~\texttt{allS}), model execution restarts just
after the \texttt{with} construct, with variable~$S$ bound to
the next choice picked in~\texttt{allS}.

As a first consistency check, we check that $S$ includes the program order:
\begin{verbatim}
empty po \ S as PoCons
\end{verbatim}
Notice that, to check for inclusion, we test the emptyness of relation
difference (operator ``\verb+\+'').


It remains to check that the \texttt{rf} relation of the execution candidate
is the same as the one defined by condition~\ref{rfcond}.
To that aim, we complement~$S$ with the constraint over initial
writes that  must precede all events to their location:
\begin{verbatim}
let S = S | loc & IW * (M \ IW)
\end{verbatim}
Observe that $S$ is no longer a total order. However, it is still a total
order when restricted to events that access a given location,
which is all that matters for condition~\ref{rfcond} to give a value
to all reads. As regards our SC model, we define \texttt{rf-S}
the read-from relation induced by~$S$ as follows:
\begin{verbatim}
let WRS = W * R & S & loc  (* Writes from the past, same location *)
let rf-S = WRS \ (S;WRS)   (* Most recent amongst them *)
\end{verbatim}
The definition is a two-step process: we first define
a relation~\texttt{WRS} from writes to reads (to the same location)
that follow them in~$S$. Observe that,
by complementing~$S$ with initial writes, we achieve that for any read~$r$
there exists at least a write~$w$ such thar $(w,r) \in \texttt{WRS}$.
It then remains to filter out non-maximal writes in \texttt{WRS}
as we do in the definition of \texttt{rf-S}, by the means of
the difference operator ``\verb+\+''.
We then check the equality of \texttt{rf} (pre-defined as part of the candidate
execution) and of \texttt{rf-S} by double inclusion:
\begin{verbatim}
empty rf \ rf-S as RfCons
empty rf-S \ rf as RfCons
\end{verbatim}

As an exemple, he show six attempts of \texttt{po} compatible $S$~orders
for the non-SC outcome of the test~\atest{SB} in figure~\ref{sblamport}.
\begin{figure}[htp]
\caption{\label{sblamport}Failed attempts of SC scheduling orders~$S$.}
\begin{center}
\img{SB+L-00}\quad\img{SB+L-01}\quad\img{SB+L-02}\\
\img{SB+L-03}\quad\img{SB+L-04}\quad\img{SB+L-05}
\end{center}
\end{figure}
Observe that all attempts fail as \texttt{rf} and \texttt{rf-S}
are different in all diagrams.

We also show all successfull SC scheduling in figure~\ref{sbok}.
\begin{figure}[htp]
\caption{\label{sbok}SC executions of test~\atest{SB}.}
\begin{center}
\img{SB+OK-00}\quad\img{SB+OK-01}\quad\img{SB+OK-02}\\
\img{SB+OK-03}\quad\img{SB+OK-04}\quad\img{SB+OK-05}
\end{center}
\end{figure}


For reference we provide our complete model~\afile{lamport.cat}
\verbatiminput{lamport.cat}

\subsection{Computing \label{sec:cos}coherence orders}
All the models seen so far include the file \afile{cos.cat} that define
``coherence relations'', written~\texttt{co}.
This section describes the file~\texttt{cos.cat}.
It can be skipped in first reading, as users may find sufficient
to include the file.

For a given location~$x$ the coherence order is a total order on the
write events to location~$x$. The coherence relation~\texttt{co} is the union
of those total orders for all locations.
In this section, we show how to compute all possible coherence orders for
a candidate execution.
We seize the opportunity to introduce advanced features of the \cat{}
language, such as functions and pattern matching over sets.

Possible coherence orders for a given location~$x$
are not totally arbitrary in two aspects:
\begin{enumerate}
\item The write events to location~$x$ include
the initial write event to location~$x$.  The initial write to~$x$ must come
first in any coherence order for~$x$.
\item One of the writes to~$x$ performed by the test (may) have been declared
to be final by \herd{} machinery prior to model execution.
In that case, the final write to~$x$ must come last in any coherence order
for~$x$.
\end{enumerate}
See Sec.~\ref{sec:final:initial} for details on initial and final writes.

We can express the two conditions above for all locations of the program
as a relation \texttt{co0}:
\begin{verbatim}
let co0 = loc & (IW*(W\IW)|(W\FW)*FW)
\end{verbatim}
Where the pre-defined sets \texttt{IW} and~\texttt{FW} are the sets
of all initial and final writes respectively.
%TODO exemple of co0 on 2+2W

Then, assuming that $W_x$ is the set of all writes to location~$x$, one
can compute the set of all possible coherence orders for~$x$ with
the \texttt{linearisations} primitive as \texttt{linearisations($W_x$,co0)}.
In practice, we define a function that takes the set~$W_x$ as an argument:
\begin{verbatim}
let makeCoX(Wx) = linearisations(Wx,co0)
\end{verbatim}
The \texttt{linearisations} primitive is introduced in
Sec.~\ref{intro:linearisations}. It returns all topological sorts
of the events  of the set~\texttt{Wx} that are compatible
with the relation~\texttt{co0}.


In fact, we want to compute the set of all possible \texttt{co} relations,
\emph{i.e.} all the unions of all the possible coherence orders for all
locations~$x$. To that end we use another \cat{} primitive:
\texttt{partition($S$)}, which takes a set of events as argument and
returns a set of set of events $T = \{S_1,\ldots,S_n\}$, where each
$S_i$ is the set of all events in $S$ that act on location $L_i$,
and, of course $S$ is the union $\bigcup_{i=1}^{i=n} S_i$.
Hence we shall compute the set of all \texttt{Wx} sets
as \texttt{partition(W)},
where \texttt{W} is the pre-defined set of all writes (including initial
writes).

For combining the effect of the \texttt{partition} and \texttt{linearisations}
primitives, we first define a \texttt{map} function that, given a set~$S=
\{e_1,\ldots,e_n\}$ and a function $f$, returns the set
$\{f(e_1),\ldots,f(e_n)\}$:
\begin{verbatim}
let map f  =
  let rec do_map S = match S with
  || {} -> {}
  || e ++ S -> f e ++ do_map S
  end in
  do_map
\end{verbatim}
The \texttt{map} function is written in curried style.
That is one calls it as \texttt{map~$f$~$S$}, parsed
as \texttt{(map~$f$)~$S$}. More precisely, the left-most function
call~\texttt{(map~$f$)} returns a function.
Here it returns~\texttt{do\_map} with free variable \texttt{f} being bound
to the argument~$f$.
The definition of~\texttt{map} illustrate several new features:
\begin{enumerate}
\item The empty set constant~``\verb+{}+'',
and the set addition operator \texttt{$e$ ++ $S$} that returns the set~$S$
augmented with element~$e$.
\item Recursive function definitions. The function~\verb+do_map+
is recursive as it calls itself.
\item Pattern matching on sets.
This construct, similar to OCaml pattern matching on lists, discriminates
between empty (\verb+|| {} ->+~$e_0$) and non-empty
(\verb!|| e ++ es ->!~$e_1$) sets.
In the second case of a non-empty set, the expression~$e_1$ is evaluated
in a context extended with two bindings: a binding from the variable~\texttt{e}
to an arbitrary element of the matched set, and a binding from
the variable~\texttt{es} to the matched set minus the arbitrary element.
\end{enumerate}

Then, we generate the set of all possible coherence orders
for all locations~$x$ as follows:
\begin{verbatim}
let allCoX = map makeCoX (partition(W))
\end{verbatim}
Notice that \texttt{allCoX} is a set of sets of relations,
each element being the set of all possible coherence orders
for a specific~$x$.  

We still need to generate all possible \texttt{co} relations,
that is all unions of the possible coherence orders for
all locations~$x$.  It can be done by another \cat{} function:
\texttt{cross}, which takes a set of sets $S = \{S_1, S_2, \ldots, S_n\}$ as
argument and returns all possible unions built by picking elements from each of
the $S_i$: 
$$
\left\{\, e_1 \cup e_2 \cup \cdots \cup e_n \mid
e_1 \in S_1, e_2 \in S_2, \ldots, e_n \in S_n \,\right\}
$$
One may notice that if $S$ is empty, then \texttt{cross} should
return one relation exactly: the empty relation, \emph{i.e.} the neutral
element of the union operator.
This choice for \texttt{cross($\emptyset$)} is natural
when we define \texttt{cross} inductively:
$$
\texttt{cross}(S_1 \mathop{\texttt{++}} S) =
\bigcup_{e_1 \in S_1, t \in \texttt{cross}(S)} \left\{ e_1 \cup t \right\}
$$
In the definition above, we simply build
\texttt{cross($S_1 \mathop{\texttt{++}} S$)} by building the set
of all unions of one relation~$e_1$ picked in~$S_1$
and of one relation~$t$ picked in $\texttt{cross}(S)$.

So as to write~\texttt{cross},
we first define a classical \texttt{fold} function over sets:
given a set $S = \{ e_1, e_2, \ldots, e_n\}$, an initial value~$y_0$
and a function $f$~that takes a pair $(e,y)$ as argument,
\texttt{fold} computes:
$$
f (e_{i_1},f (e_{i_2}, \ldots, f(e_{i_n},y_0)))  
$$
where $i_1, i_2, \ldots, i_n$ defines a permutation
of the indices $1, 2, \ldots, n$.
\begin{verbatim}
let fold f =
  let rec fold_rec (es,y) = match es with
  || {} -> y
  || e ++ es -> fold_rec (es,f (e,y))
  end in
  fold_rec
\end{verbatim}
The function~\texttt{fold} is written in the same curried style as~\texttt{map}.
Notice that the inner function~\verb+fold_rec+ takes one argument.
However this argument is a pair.
As a gentle example of \texttt{fold} usage, we could have
defined~\texttt{map} as:
\begin{verbatim}
let map f = fun S -> fold (fun (e,y) -> f e ++ y) (S,{})
\end{verbatim}
This example also introduce ``anonymous'' functions.


As a more involved example of \texttt{fold} usage, we
 write the function~\texttt{cross}.
\begin{verbatim}
let rec cross S = match S with
  || {} -> { 0 } (* 0 is the empty relation *)
  || S1 ++ S ->
      let ts = cross S in
      fold
        (fun (e1,r) -> map (fun t -> e1 | t) ts | r)
        (S1,{})
  end      
\end{verbatim}
The function~\texttt{cross} is a recursive function over a set (of sets).
Its code follows the inductive definition given above.

Finally, we generate all possible \texttt{co} relations by:
\begin{verbatim}
let allCo = cross allCoX
\end{verbatim}

The file~\afile{cos.cat} goes on by iterating over \texttt{allCo} using
the \texttt{with $x$ from~$S$} construct:
\begin{verbatim}
with co from allCo
\end{verbatim}
See Sec.~\ref{intro:with} for details on this construct.

Once~\texttt{co} has been defined, one defines~\texttt{fr} and
internal and external variations:
\begin{verbatim}
(* From now, co is a coherence relation *)
let coi = co & int
let coe = co & ext

(* Compute fr *)
let fr = rf^-1 ; co
let fri = fr & int
let fre = fr & ext
\end{verbatim}
The pre-defined relation \texttt{ext} (resp. \texttt{int}) relates
events generated by different (resp. the same) threads.


\section{Producing pictures of executions}
The simulator \herd{} can be instructed to produce pictures of
executions.
Those pictures are instrumental in understanding and
debugging models.
It is important to understand that \herd{} does not produce pictures
by default. To get pictures one must instruct \herd{} to produce
pictures of some executions with the \opt{-show} option.
This option accepts specific keywords, its default being ``\opt{none}'',
instructing \herd{} not to produce any picture.

A frequentlty used keyword is ``\opt{prop}'' that means ``show the executions
that validate the proposition in the final condition''.
Namely, the final condition in litmus test is a quantified
boolean proposition as for instance ``\verb+exists (0:EAX=0 /\ 1:EAX=0)+'' at the end of test \atest{SB}.

But this is not enough, users also have to specify what to do with the picture:
save it in file in the DOT format of the
\ahref{http://graphviz.org/}{\prog{graphviz} graph visualization software}, or
display the image,\footnote{This option requires
the Postscript visualiser \ahref{\urlgv}{\prog{gv}}.} or both.
One instructs  \herd{} to save images with the \opt{-o }\textit{dirname} option,
where \textit{dirname} is the name of a directory, which must exists.
Then, when processing the file \textit{name}\texttt{.litmus},
\herd{} will create a file \textit{name}\texttt{.dot} into the
directory~\textit{dirname}.
For displaying images, one uses the \opt{-gv} option.

\label{sec:show}As an example,
so as to display the image of the non-SC behaviour of \atest{SB}, one
should invoke \herd{} as:
\begin{verbatim}
% herd -model tso-02.cat -show prop -gv SB.litmus
\end{verbatim}
\aname{sb:cluster}{As}
a result, users should see a window popping and displaying this image:
\begin{center}\img{SB+CLUSTER}\end{center}
Notice that we got the PNG version of this image as follows:
\begin{verbatim}
% herd -model tso-02.cat -show prop -o /tmp SB.litmus
% dot -Tpng /tmp/SB.dot -o SB+CLUSTER.png
\end{verbatim}
That is, we applied the \prog{dot} tool from the
\ahref{\urlgraphviz}{\prog{graphviz}} package, using the appropriate option
to produce a PNG image.

One may observe that there are \verb+ghb+ arrows in the diagram.
This results from the \verb+show ghb+ instruction
at the end of the model file~\afile{tso-02.cat}.

\subsection{Graph modes}
The image \ahrefloc{sb:cluster}{above} much differs from
the one in Sec.~\ref{sb:image} that describes the same execution
and that is reproduced in Fig.~\ref{fig:sb}
\begin{figure}
\caption{\label{fig:sb}The non-SC behaviour of \atest{SB} is allowed by TSO}
\begin{center}
\img{SB+TSO}
\end{center}
\end{figure}

\label{mode:example}In effect, \herd{} can produce three styles
of pictures, \prog{dot} clustered pictures, \prog{dot} free pictures,
and \prog{neato} pictures with explicit placement of the
events of one thread as a colum.
The style is commanded by the \opt{-graph} option that accepts three
possible arguments: \opt{cluster} (default), \opt{free} and~\opt{columns}.
The following pictures show
the effect of graph styles on the \atest{SB}~example:
\begin{center}
\begin{tabular}{*{3}{p{.25\linewidth}}}
\multicolumn{1}{c}{\opt{-graph cluster}} &
\multicolumn{1}{c}{\opt{-graph free}} &
\multicolumn{1}{c}{\opt{-graph columns}}\\
\img{SB+SQUISHED} \qquad &
\img{SB+FREE}\qquad &
\img{SB+COLUMNS}
\end{tabular}
\end{center}
Notice that we used another option \opt{-squished true} that much reduces
the information displayed in nodes. Also notice that
the first two pictures are formatted by \prog{dot},
while the rightmost picture is formatted by \prog{neato}.

One may also observe that the ``\opt{-graph columns}''  picture does not
look exactly like Fig.~\ref{fig:sb}. For instance the
\textrel{ghb} arrows are thicker in the figure.
There are many parameters to control \prog{neato} (and~\prog{dot}),
many of which are accessible to \herd{} users by the means of appropriate
options. We do not intend to describe them all.
However, users can reproduce the style of the diagram of this manual using
yet another feature of \herd: \ahrefloc{herd:configfile}{configuration files}
that contains settings for \herd{} options and that are loaded with the
\opt{-conf~}\textit{name} option.
In this manual we mostly used the \afile{doc.cfg} configuration file.
As this file is present in \herd{} distribution, users
can use the diagram style of this manual:
\begin{verbatim}
% herd -conf doc.cfg ...
\end{verbatim}

\subsection{\label{show:forbidden}Showing forbidden executions}
Images are produced or displayed once the model has been executed.
As a consequence,
forbidden executions won't appear by default.
Consider for instance the test \atest{SB+mfences},
where the \texttt{mfence} instruction is used to forbid
\atest{SB} non-SC execution. Runing \herd{} as
\begin{verbatim}
% herd -model tso-02.cat -conf doc.cfg -show prop -gv SB+mfences.litmus
\end{verbatim}
will produce no picture, as the TSO model forbids the target execution
of~\textsf{SB+mfences}.

To get a picture, we can run \textsf{SB+mfences} on top of the mininal
model, a pre-defined model that allows all executions:
\begin{verbatim}
% herd -model minimal -conf doc.cfg -show prop -gv SB+mfences.litmus
\end{verbatim}
And we get the picture:
\begin{center}\img{SB+mfences}\end{center}
It is worth mentioning again  that although the minimal model allows all
executions, the final condition
selects the displayed picture, as we have specified the
\opt{-show prop} option.


\label{name:check}The picture above shows \verb+mfence+ arrows, as all
fence relations are displayed by the minimal model.
However, it  does not show the \verb+ghb+ relation, as the minimal
model knows nothing of it.
To display~\verb+ghb+ we could write another model file that would be just as
\afile{tso-02.cat}, with checks erased.
The simulator \herd{} provides a simpler technique:
one can instruct \herd{} to ignore
either all checks (\opt{-through invalid}), or a selection of checks
(\opt{-skipchecks~\textit{name$_1$},\ldots,\textit{name$_n$}}).
Thus, either of the following two commands
\begin{verbatim}
% herd -through invalid -model tso-02.cat -conf doc.cfg -show prop -gv SB+mfences.litmus
% herd -skipcheck tso -model tso-02.cat -conf doc.cfg -show prop -gv SB+mfences.litmus
\end{verbatim}
will produce the picture we wish:
\begin{center}\img{SB+mfences+GHB}\end{center}
Notice that \verb+mfence+ and~\verb+ghb+ are displayed because
of the instruction ``\verb+show mfence ghb+'' (fence relation are not shown
by default);
while \opt{-skipcheck tso} works because the \afile{tso-02.cat} model
names its main check with ``\verb+as tso+''.

The image above is barely readable.
For such graphs with many relations, the \verb+cluster+ and~\verb+free+ modes
are worth a try. The commands:
\begin{verbatim}
% herd -skipcheck tso -model tso-02.cat -conf doc.cfg -show prop -graph cluster -gv SB+mfences.litmus
% herd -skipcheck tso -model tso-02.cat -conf doc.cfg -show prop -graph free -gv SB+mfences.litmus
\end{verbatim}
will produce the images:
\begin{center}
\begin{tabular}{p{.33\linewidth}@{\hspace*{6em}}p{.33\linewidth}}
\img{SB+mfences+CLUSTER}
&
\img{SB+mfences+FREE}
\end{tabular}
\end{center}
Namely, command line options are scanned left-to-right,
so that most of the settings of \afile{doc.cfg} are kept\footnote{The setting of \opt{showthread} is also changed, by the omitted \opt{-showthread true} command line option}
(for instance thick \verb+ghb+ arrows), while the graph mode is overriden.


\section{\label{herd:language}Model definitions}

We describe our \cat{}~langage for defining models.
The syntax of the language is given in BNF-like notation. Terminal
symbols are set in typewriter font (\synt{\T{like} \T{this}}).
Non-terminal symbols are set in italic font (\synt{\NT{like} \NT{that}}).
An unformatted vertical bar \synt{\ldots\orelse\ldots}
denotes alternative.
Square brackets \synt{\boption{}\ldots\eoption{}} denote optional components. Curly brackets
\synt{\brepet{}\ldots\erepet{}} denotes zero,
one or several repetitions of the enclosed
components.
Parentheses \synt{\bparen{}\ldots\eparen{}} denote grouping.


Model source files may contain comments of the OCaml type
(\verb+(*+\ldots \verb+*)+, can be nested), or line comments starting with
``\verb+//+'' and running until end of line.

\subsection{\label{overview}Overview}
The \cat{} language is much inspired by OCaml, featuring immutable bindings,
first-class functions, pattern matching, etc.
However, \cat{} is a domain specific language, with important differences
from OCaml.
\begin{enumerate}
\item Base values are specialised, they are sets of events and relations
over events. There are also tags, akin to C~enumerations or OCaml
``constant'' constructors and first class functions. 
There are two structured values: tuples of values and sets of values.
\item There is a distinction between expressions that evaluate
to some value, and instructions that are executed for their effect.
\end{enumerate}
A model, or \cat{} program is a sequence of instructions.
At startup, pre-defined identifiers are bound to event sets and relations
over events.
Those pre-defined identifiers describe a candidate execution
(in the sense of the memory model).
Executing the model means allowing or forbiding that candidate
execution.


\subsection{\label{language:identifier}Identifiers}
\begin{syntax}
\NT{letter} \is \T{a} \ldots\T{z}
\orelse  \T{A} \ldots\T{Z}
\sep
\NT{digit} \is \T{0} \ldots\T{9}
\sep
\NT{id} \is \NT{letter} \brepet{} \NT{letter} \orelse \NT{digit}
\orelse \T{\_} \orelse \T{.} \orelse \T{-} \erepet
\end{syntax}
Identifiers are rather standard: they are a sequence of letters, digits,
``\texttt{\_}'' (the underscore character), ``\texttt{.}'' (the dot character)
and ``\texttt{-}'' (the minus character),
starting with a letter.
Using the minus character inside identifiers  may look a bit surprising.
We did so as to allow identifiers such as \texttt{po-loc}.

\label{sec:predef}At startup, pre-defined identifiers are bound to
event sets and to relations
between events.

Those pre-defined identifiers first describe the events of
the candidate execution as various sets, as described by the first table
of figure~\ref{predefset}.
\begin{figure}[htp]
\caption{\label{predefset}Pre-defined event sets.}
\begin{idtable}
\textrel{W} & write events \\
\textrel{R} & read events \\
\textrel{M} & memory events &
we have $\textrel{M} = \textrel{W} \cup \textrel{R}$\\
\textrel{IW} & initial writes &
feed reads that read from the initial state\\
\textrel{FW} & final writes & writes that are observed at the end of test execution\\
\textrel{B} & branch events\\
\textrel{RMW} & read-modify-write events\\
\textrel{F} & fence events\\
\textit{NAME} & specific fence events & those depend on the test architecture\\
\end{idtable}

\begin{desctable}{architecture}{fence sets}
\textrel{X86} & \textrel{MFENCE}, \textrel{SFENCE}, \textrel{LFENCE}\\
\textrel{PPC} & \textrel{SYNC}, \textrel{LWSYNC}, \textrel{EIEIO}, \textrel{ISYNC}\\
\textrel{ARM} &  \textrel{DMB}, \textrel{DMB.ST}, \textrel{DSB}, \textrel{DSB.ST}, \textrel{ISB}\\
 \textrel{MIPS} & \textrel{SYNC}\\
\textrel{AArch64} & \ldots\\
\end{desctable}
\end{figure}
Specific fence event sets depends on the test architecture,
their name is always uppercase and derive from the mnemonic of
the instruction that generates them.
The second table of figure~\ref{predefset} shows
a (non-exhaustive) list.


Other pre-defined identifiers are relations.
Most of those are the program order~\tid{po} and its refinements:
\begin{idtable}
$\po$    & program order & instruction order lifted to events \\
$\addr$ & address dependency & the address of the second event depends on
the value loaded by the first (read) event\\
$\data$ & data dependency & the value stored by the second (write)
event depends on
the value loaded by the first (read) event\\
$\ctrl$ & control dependency &
the second event is in a branch controled by the value loaded by the
commfirst (read) event\\
$\rmwr$ & read-exclusive write-exclusive pair &
relate the read and write events emitted
by matching successful load-reserve store conditional instructions.
\end{idtable}


Finally, a few pre-defined relations describe the execution
candidate structure and write-to-read communication:
\begin{idtable}
$\id$ & identity & relates each event to itself\\
$\locr$ & same location & events that touch the same address\\
$\extr$ & external & events from different threads\\
$\intr$ & internal & events from the same thread\\

$\rf$  & read-from & links a write $w$ to a read $r$ taking its value from $w$ \\
\end{idtable}

Some additional relations are defined by library files written in the \cat{}
language, see Sec.~\ref{sec:library}.

\subsection{\label{language:expression}Expressions}
Expressions are evaluated by \herd, yielding a value.
\begin{syntax}
\NT{expr} \is{} \T{0}
\alt \NT{id}
\alt \NT{tag}
\alt \T{(}\T{)} \orelse \T{(} \NT{expr} \T{,} \NT{expr} \brepet{} \T{,} \NT{expr} \erepet \T{)}
\alt \T{\{}\T{\}} \orelse \T{\{} \NT{expr} \brepet{} \T{,} \NT{expr} \erepet \T{\}}
\alt \NT{expr}\T{*} \orelse \NT{expr}\T{+} \orelse \NT{expr}\T{?}
\orelse \NT{expr}\T{\textasciicircum-1}
\alt \T{\textasciitilde}\NT{expr}
\alt \NT{expr}\T{|}\NT{expr} \orelse
\NT{expr}\T{++}\NT{expr} \orelse
\NT{expr}\T{;}\NT{expr} \orelse
\NT{expr}\T{\textbackslash}\NT{expr} \orelse
\NT{expr}\T{\&}\NT{expr} \orelse
\NT{expr} \T{*} \NT{expr}
\alt \NT{expr} \NT{expr}
\alt \T{fun} \NT{pat} \T{->} \NT{expr}
\alt \T{let} \boption{} \T{rec} \eoption \NT{binding} \brepet{} \T{and} \NT{binding} \erepet{} \T{in} \NT{expr}
\alt \T{match} \NT{expr} \T{with} \NT{clauses} \T{end}
\alt \T{(}\NT{expr}\T{)} \orelse  \T{begin} \NT{expr} \T{end}
\alt \T{instructions} \NT{id}\NT{[taglist]}
\sep
\sep
\NT{tag} \is \T{'} \NT{id}
\sep
\NT{taglist} \is \NT{tag, taglist}
\sep
\NT{pat} \is \NT{id} \orelse \T{(}\T{)} \orelse \T{(} \NT{id} \brepet{} \T{,} \NT{id} \erepet \T{)}
\sep
\NT{binding} \is \NT{valbinding} \orelse \NT{funbinding}
\sep
\NT{valbinding} \is \NT{id} \T{=} \NT{expr}
\sep
\NT{funbinding} \is \NT{id} \NT{pat} \T{=} \NT{expr}
\sep
\sep
\NT{clauses} \is \NT{tagclauses} \orelse \NT{setclauses}
\sep
\NT{tagclauses} \is \boption{} \T{||} \eoption \NT{tag} \T{->} \NT{expr}
\brepet{} \T{||} \NT{tag} \T{->} \NT{expr} \erepet
\boption \T{\_} \T{->} \NT{expr} \eoption
\sep
\NT{setclauses} \is \boption{} \T{||} \eoption \T{\{}\T{\}} \T{->} \NT{expr}
\T{||} \NT{id} \T{++} \NT{id} \T{->} \NT{expr}
\end{syntax}

\subsubsection*{Simple expressions}
Simple expressions are the empty relation (keyword~\synt{\T{0}}),
identifiers~\synt{\NT{id}} and tags~\synt{\NT{tag}}.  Identifiers are bound to
values, either before the execution (see pre-defined identifiers in
Sec.~\ref{sec:predef}), or by the model itself.  Tags are constants similar to
C enum values or OCaml constant constructors.  Tags must be declared with the
\T{enum} instruction. We go back to \T{enum} and tags in Sec.~\ref{sec:enum}
and \ref{sec:bell}.

\subsubsection*{Tuples}
Tuples include a constant, the empty tuple \synt{\T{(}\T{)}},
and constructed tuples
\synt{\T{(} \NT{expr}_1 \T{,} \NT{expr}_1 \T{,}\ldots \T{,} \NT{expr}_n\T{)}},
with $n \geq 2$. In other words there is no tuple of size one.
Syntax \synt{\T{(} \NT{expr} \T{)}} denotes grouping and has the same
value as~\synt{\NT{expr}}.


\subsubsection*{Explicit sets of values}
Explicit sets are written as the comma separated
list of their elements between curly braces:
\synt{\T{\{} \NT{expr}_1 \T{,} \NT{expr}_1 \T{,}\ldots \T{,} \NT{expr}_n\T{\}}},
with $n \geq 0$.
As events are not values, one cannot build a set of events using
explicit set expressions.
However, by exception, the empty set~\synt{\T{\{}\T{\}}}
also is the empty set of events and the empty relation.
Sets are homogenous, in the sense that sets hold elements of the same type.

\subsubsection*{Operator expressions}
The transitive and reflexive-transitive closure of an expression are performed
by the postfix operators \T{+} and~\T{*}.
The  postfix operator \T{\textasciicircum-1} performs relation inversion.
The construct \synt{\NT{expr}\T{?}} (option) evaluates to the union
of  \NT{expr} value and of the identity relation.
Notice that postfix operators operate on relations only.

There is one prefix operator~\T{\textasciitilde} that performs
relation and set complement.

Infix operators are
\T{|} (union), \T{++} (set addition),
\T{;} (sequence), \T{\&} (intersection), \T{\textbackslash} (set difference),
and~\T{*} (cartesian product).
Infix operators  are listed in order of decreasing precedence,
while  postfix and prefix operators bind tighter than infix operators.
All infix operators are right-associative,
except set difference which is left-associative, and cartesian product
which is non-associative.

The union, intersection and difference operators apply to relations
and all kinds of sets.
The addition operator \synt{\NT{expr}_1 \T{++} \NT{expr}_2} operates on
sets: the value of \synt{\NT{expr}_2} must be a set of values
~$S$ and the operator returns the set~$S$ augmented with the value of
\synt{\NT{expr}_1}.

For the record, given two relations $r_1$ and~$r_2$,
the sequence $r_1; r_2$ is defined
as $\{ (x,y) \mid \exists z, (x,z) \in r_1 \wedge (z,y) \in r_2\}$.

\subsubsection*{Function calls}
Functions calls are written \synt{\NT{expr}_1 \NT{expr}_2}.
That is, functions are of arity one and the application operator
is left implicit. Notice that function application binds tighter
than all binary operators and looser that postfix operators.
Furthermore the implicit application operator is left-associative.

The \cat{} language has call-by-value semantics. That is,
the effective parameter
\synt{\NT{expr}_2} is evaluated before being bound to the
function formal parameter(s).

N-ary functions can be encoded either using tuples as arguments
or by curryfication (\emph{i.e.} as functions that return functions).
Considering binary functions, in the former case,
a function call is written
\synt{\NT{expr}_1 \T{(} \NT{expr}_2 \T{,} \NT{expr}_3\T{)}};
while in the latter case, a function call is written
\synt{\NT{expr}_1~\NT{expr}_2~\NT{expr}_3}
(which by left-associativity, is to be understood
as \synt{\T{(}\NT{expr}_1~\NT{expr}_2\T{)}~\NT{expr}_3}).
The two forms of function call are not interchangeable, using one or the
other depends on the definition of the function.


\subsubsection*{Functions}
Functions are first class values, as reflected by the anonymous
function construct \synt{\T{fun} \NT{pat}  \T{->} \NT{expr}}.
A function takes one argument only.

In the case where this argument is a tuple, it may be destructured
by the means of a tuple pattern. That is \synt{\NT{pat}}
above is \synt{\T{(} \NT{id}_1 \T{,} \ldots \NT{id}_n\T{)}}.
For instance here is a function that takes a tuple of
relations (or sets) as argument and return their symmetric difference:
\begin{verbatim}
fun (a,b) -> (a\b)|(b\a)
\end{verbatim}

Functions have the usual static scoping semantics:
variables that appear free in function bodies 
(\synt{\NT{expr}} above) are bound to
the value of such free variable at function creation time.
As a result one may also write the symmetric difference function
as follows:
\begin{verbatim}
fun a -> fun b -> (a\b)|(b\a)
\end{verbatim}

\subsubsection*{\label{bindings}Local bindings}
The local binding construct
\synt{\T{let} \boption{} \T{rec} \eoption{} \nt{bindings} \T{in} \NT{expr}}
binds the names defined by \nt{bindings} 
for evaluating the expression \NT{expr}.
Both non-recursive and recursive bindings are allowed.
The function binding
\synt{\NT{id} \NT{pat}  \T{=} \NT{expr}} is syntactic sugar
for \synt{\NT{id} \T{=} \T{fun} \NT{pat} \T{->} \NT{expr}}.

The construct
\begin{center}
\synt{\T{let} \NT{pat}_1 \T{=} \NT{expr}_1 \T{and} \ldots \T{and} \NT{pat}_n \T{=} \NT{expr}_n} \T{in} \NT{expr}
\end{center}
evaluates \synt{\nt{expr}_1,\ldots, \nt{expr}_n},
and binds the names in the patterns
\synt{\nt{pat}_1,\ldots, \nt{pat}_n} to the resulting values.
The bindings for \synt{\nt{pat} \T{=} \nt{expr}} are as follows:
if \nt{pat} is \T{(}\T{)}, then \nt{expr} must evaluate to the empty
tuple;
if \nt{pat} is \synt{\nt{id}} or \synt{\T{(}\nt{id}\T{)}},
then \nt{id} is bound to the value of~\synt{\nt{expr}};
if \nt{pat} is a proper tuple pattern
\synt{\T{(}\nt{id}_1\T{,}\ldots \T{,}\nt{id}_n\T{)}} with $n \geq 2$,
then \synt{\nt{expr}} must evaluate to a tuple value of size~$n$
$(v_1,\ldots,v_n)$ and the names $\nt{id}_1,\ldots,\nt{id}_n$ are
bound to the values  $v_1,\ldots,v_n$.

\aname{letrec}{The} construct
\begin{center}
\synt{\T{let} \T{rec} \NT{pat}_1 \T{=} \NT{expr}_1 \T{and} \ldots \T{and} \NT{pat}_n \T{=} \NT{expr}_n} \T{in} \NT{expr}
\end{center}
computes the least fixpoint of the equations
$\nt{pat}_1 = \nt{expr}_1$,\ldots, $\nt{pat}_n = \nt{expr}_n$.
It then binds the names in the patterns
\synt{\nt{pat}_1,\ldots, \nt{pat}_n} to the
resulting values.
The least fixpoint computation applies to set and relation values,
(using inclusion for ordering); and to
functions (using the usual definition ordering).


\subsubsection*{Pattern matching over tags}
The syntax for pattern matching over tags is:
\begin{center}
\synt{\T{match} \NT{expr} \T{with} \NT{tag}_1 \T{->} \NT{expr}_1
\T{||} \cdots \T{||} \NT{tag}_n \T{->} \NT{expr}_n
\T{||} \T{\_} \T{->} \NT{expr}_d
\T{end}}
\end{center}
The value of the match expression is computed as follow: first evaluate
\synt{\nt{expr}} to some value~$v$, which must be a tag~$t$.
Then $v$ is compared with the tags \synt{\nt{tag}_1,\ldots,\nt{tag}_n},
in that order.
If some tag pattern~\synt{\nt{tag}_i} equals~$t$, then the value of the
match is the value of the corresponding expression~\synt{\nt{expr}_i}.
Otherwise, the value of the match is the value of the default
expression~\synt{\nt{expr}_d}.
As the default clause~\synt{\T{\_} \T{->} \NT{expr}_d} is optional,
the match construct may fail.

\subsubsection*{Pattern matching over sets}
The syntax for pattern matching over sets is:
\begin{center}
\synt{\T{match} \NT{expr} \T{with}
\T{\{}\T{\}} \T{->} \NT{expr}_1
\T{||} \NT{id}_1 \T{++} \NT{id}_2 \T{->} \NT{expr}_2
\T{end}}
\end{center}
The value of the match expression is computed as follow: first evaluate
\synt{\nt{expr}} to some value~$v$, which must be a set of values.
If $v$ is the empty set, that the value of the match is the
value of the corresponding expression~\synt{\nt{expr}_1}.
Otherwise, $v$ is a non-empty set, then let $v_e$ be some element in~$v$
and $v_r$ be the set~$v$ minus the element~$v_e$.
The value of the match is the value of \synt{\nt{expr}_2} in a context
where \synt{\nt{id}_1} is bound to~$v_e$ and \synt{\nt{id}_2} is bound
to~$v_r$.


\subsubsection*{Parenthesised expressions}
The expression \synt{\T{(}\NT{expr}\T{)}}
has the same value as \synt{\NT{expr}}.
Notice that a parenthesised expression
can also be written as  \synt{\T{begin} \NT{expr} \T{end}}.

\subsection{\label{language:instruction}Instructions}
Instruction are executed for their  effect.
There are three kinds of effects: adding new bindings,
checking a condition, and specifying relations that are shown in pictures.
\begin{syntax}
\NT{instruction} \is{}  \T{let} \boption{} \T{rec} \eoption \NT{binding} \brepet{} \T{and} \NT{binding} \erepet{}
\alt \boption \T{flag} \eoption \NT{check} \NT{expr} \boption \T{as} \NT{id}\eoption
\alt \T{enum} \NT{id} \T{=} \boption \T{||} \NT{tag}
\brepet \T{||} \NT{tag} \erepet
\alt \T{procedure} \NT{id} \NT{pat} \T{=} \brepet \NT{instruction} \erepet \T{end}
\alt \T{call} \NT{id} \NT{expr}  \boption \T{as} \NT{id}\eoption
\alt \T{show} \NT{expr} \T{as} \NT{id}
\alt \T{show} \NT{id} \brepet \T{,} \NT{id} \erepet
\alt \T{unshow} \NT{id} \brepet \T{,} \NT{id} \erepet
\alt \T{forall} \NT{id} \T{in} \NT{expr} \T{do} \brepet{} \NT{instruction} \erepet \T{end}
\alt \T{with} \NT{id} \T{from} \NT{expr}
\alt \T{include} \nt{string}
\sep
\NT{check} \is \NT{checkname} \orelse \T{\textasciitilde} \NT{checkname}
\sep
\NT{checkname} \is \T{acyclic} \orelse \T{irreflexive} \orelse \T{empty}
\end{syntax}

\subsubsection*{Bindings}
The \T{let} and \T{let}~\T{rec} constructs bind value names for the rest
of model execution.
See the subsection on \ahrefloc{bindings}{bindings}
in Section~\ref{language:expression}
for additional information on the syntax and semantics of bindings.


Recursive definitions computes fixpoints of relations.
For instance, the following fragment computes the transitive closure of
all communication relations:
\begin{verbatim}
let com = rf | co | fr
let rec complus = com | (complus ; complus)
\end{verbatim}
Notice that the instruction \verb-let complus = (rf|co|fr)+- is equivalent.
Notice that \herd{} assumes that recursive definitions are well-formed,
\emph{i.e.} that they yield an increasing functional.
The result of ill-formed definitions is undefined.

Although \herd{} features recursive functions, those cannot be used
to compute a transitive closure, due to the lack of some construct
say to test relation equality. Nevertheless, one can
write a generic transitive closure
function by using a local recursive binding:
\begin{verbatim}
let tr(r) = let rec t = r | (t;t) in t
\end{verbatim}
Again, notice that the instruction \verb-let tr (r) = r+- is equivalent.

Thanks to pattern matching constructs,
recursive functions are useful to compute over sets (and tags).
For instance here is the definition of a function \texttt{power} that compute
power sets:
\begin{verbatim}
let rec power S = match S with
|| {} -> { {} }
|| e ++ S ->
    let rec add_e RR = match RR with
    || {} -> { }
    || R + RR -> R ++ (e ++ R) ++ add_e RR
    end in
    add_e (power S)
end
\end{verbatim}

\subsubsection*{\label{sec:check}Checks}
The construct
\begin{center}\synt{\NT{check} \NT{expr}}\end{center}
evaluates \nt{expr} and applies the check \nt{check}.
There are six checks: the three basic acyclicity (keyword~\T{acyclic}),
irreflexivity (keyword~\T{irreflexive})
and emptyness (keyword~\T{empty}); and their
negations.
If the check succeeds, execution goes on. Otherwise, execution stops.

\label{name:check:def}The performance of a
check can optionally be named  by appending
\synt{\T{as} \NT{id}} after it.
The feature permits not to perform some checks at user's will,
thanks to the \ahrefloc{skipchecks}{\opt{-skipchecks~}\nt{id}}
command line~option.

A check can also be flagged, by prefixing it with the \T{flag}
keyword. Flagged checks must be named with the \T{as} construct.
Failed flagged checks do \emph{not} stop execution.
Instead successful flagged checks are recorded under their name,
for \herd{} machinery to handle flagged executions later.
Flagged checks are useful for models that define conditions
over executions that impact the semantics of the whole program.
This is typically the case of data races.
Let us assume that some relation \verb+race+ has been defined,
such that an non-empty \verb+race+ relation in  some execution
would make the whole program undefined. We would then write:
\begin{verbatim}
flag ~empty race as undefined
\end{verbatim}
Then, \herd{} will indicate in its output that some
execution have been flagged as \verb+undefined+.




\subsubsection*{Procedure definition and call}
Procedures are similar to functions except that they have no results:
the body of a procedure is a list of instructions
and the procedure will be called for the effect of executing
those instructions. Intended usage of procedures is to define checks
that are executed later. However, the body of a procedure may
consist in any kind of instructions.
Notice that procedure calls can be named with the \T{as} keyword.
The intention is to control the performance of procedure calls
from  the command line, exactly as for checks (see
\ahrefloc{name:check:def}{above}).


As an example of procedure,
one may define the following \verb+uniproc+ procedure with
no arguments:
\begin{verbatim}
procedure uniproc() =
  let com = fr | rf | co in
  acyclic com | po
end
\end{verbatim}
Then one can perform the acyclicity check (see
\ahrefloc{sec:check}{previous section}) by executing the instruction:
\begin{verbatim}
call uniproc()
\end{verbatim}
As a result the execution will stop if the acyclicity check fails,
or continue otherwise.

Procedures are lexically scoped as functions are.
Additionally, the bindings performed during the execution of a procedure call
are discarded when the procedure returns, all other effects performed
(namely flags and shows) are retained.

\subsubsection*{Show (and unshow) directives}
\label{show:def}The constructs:
\begin{center}
\synt{\T{show} \NT{id} \brepet \T{,} \NT{id} \erepet}\quad{and}\quad\synt{\T{unshow} \NT{id} \brepet \T{,} \NT{id} \erepet}
\end{center}
take (non-empty, comma separated) lists of identifiers as arguments.
The \T{show} construct adds the present values of identifiers for being
shown in pictures.
The \T{unshow} construct removes the identifiers from shown relations.

The more sophisticated construct
\begin{center}\synt{\T{show} \NT{expr} \T{as} \NT{id}}\end{center}
evaluates \nt{expr} to a relation, which will be shown in pictures with
label~\nt{id}.
Hence  \synt{\T{show} \nt{id}} can be viewed as a shorthand
for \synt{\T{show} \nt{id} \T{as} \nt{id}}

\subsubsection*{Iteration over sets}
The \T{forall} iteration construct 
permits the iteration of checks (in fact of any kind of instructions)
over a set. Syntax is:
\begin{center}
\T{forall} \NT{id} \T{in} \NT{expr} \T{do} \nt{instructions} \T{end}
\end{center}
The expression \synt{\nt{expr}} must evaluate to a set~$S$.
Then, the list of instructions \nt{instructions} is executed
for all bindings of the name~\nt{id} to some element of~$S$.
In practice, as failed checks stop execution, this amounts
to check the conjunction of the checks performed by \nt{instructions}
for all the elements of~$S$.
Similarly to procedure calls,
the bindings performed during the execution of an iteration
are discarded at iteration ends, all other effects performed are
retained.

\subsubsection*{Candidate execution extension}
This construct permits the extension of the current candidate
execution by one binding.
Syntax is \synt{\T{with} \NT{id} \T{from} \NT{expr}}.
The expression \nt{expr} is evaluated to a set~$S$.
Then the remainder of the model is executed for each choice
of element~$e$ in~$S$ in a context extended by a binding
of the name~\nt{id} to~$e$.
An example of the construct usage is described in Sec.~\ref{intro:with}.

\subsubsection*{Model inclusion}
The construct \synt{\T{include} \T{"}\nt{filename}\T{"}} is interpreted as
the inclusion of the model contained in the file whose name is given as
an argument to the \synt{\T{include}} instruction.
In practice the list of intructions defined by the included model file
are executed.
The string argument is delimited by double quotes ``\verb+"+'',
which, of course, are not part of the filename.
\iftrue
Files are searched according to \herd{} rules --- see Sec.~\ref{herd:searchpath}.
\else
Notice that:
\begin{itemize}
\item Files are searched according to \herd{} rules --- see Sec.~\ref{herd:searchpath}.
\item Model options defined in the model file \synt{\nt{filename}} are ignored.
\end{itemize}
\fi

\subsection*{Bell extensions}

Users can attain more genericity in their models by defining a {\tt bell} file,
as an addendum, or rather preamble, to a {\tt cat} file.

\subsubsection*{\label{sec:enum}Enumerations}
The \T{enum} construct defines a set of enumerated values or tags. Syntax is
\begin{center}
\synt{\T{enum} \NT{id} \T{=} \NT{tag}_1 \T{||} \cdots \T{||} \NT{tag}_n}
\end{center}
The construct has two main effects.
It first defines the tags \synt{\nt{tag}_1,\ldots,\nt{tag}_n}.
Notice that tags do not exist before being defined, that is
evaluating the expression \nt{tag} is an error without a prior
\T{enum} that defines the tag~\nt{tag}. Tags are typed in the sense
that they belong to the tag type \nt{id} and that tags from
different types cannot be members of the same set.
The second effect of the construct is to define a set of tags~\nt{id}
as the set of all tags listed in the construct.
That is, the \T{enum} construct performs the binding of~\nt{id}
to \synt{\T{\{} \nt{tag}_1,\ldots,\nt{tag}_n\T{\}}}.

\emph{Scopes} are a special case of enumeration: the construct {\tt enum
scopes} must be used to define hierarchical models such as Nvidia GPUs (see
e.g.~\cite{abd15}).

An {\tt enum scopes} declaration must be paired with two functions {\tt narrower} and {\tt wider} that implement the hierarchy amongst scopes. For example:

\begin{verbatim}
enum scopes = 'discography || 'I || 'II || 'III || 'IV

let narrower(t) = match t with
  || 'discography -> {'I, 'II, 'III, 'IV} 
end

let wider(t) = match t with
  || 'I -> 'discography
  || 'II -> 'discography
  || 'III -> 'discography
  || 'IV -> 'discography
end
\end{verbatim}

Here we define five scopes, where the first one, {\tt discography}, is wider
than all the other ones.

\subsubsection*{Instructions}
The predefined sets of events \textrel{W}, \textrel{R}, \textrel{RMW},
\textrel{F}, and \textrel{B} can be \emph{annotated} with user-defined tags
(see Sec.~\ref{sec:enum}).

The constructs:

\begin{center}
\T{instructions}\NT{id}\NT{[taglist]}  
\end{center}

take the identifier of a pre-defined set and a possibly empty, square bracketed
list of tags. 

%jade@Luc: du coup il faudrait peut etre les distinguer dans la def des
%expressions?
The primitive {\tt tag2instrs} yields, given a tag {\tt 't}, the set of
instructions bearing the annotation {\tt t} that was previously declared in an
enumeration type.
%\fixme{jade: enumeration type? c'est comme ca que ca se dit?} 

The primitive {\tt tag2scope} yields, given a tag {\tt 't}, the relation
between instructions TODO  
 
\subsection{\label{language:model}Models}
\begin{syntax}
\NT{model} \is \NT{model-comment}  \NT{instruction} \erepet
\sep
\NT{model-comment} \is \NT{id} \orelse  \nt{string}
\end{syntax}
A model is a list of instruction preceded by a small comment,
which can be either a name that follows \herd{} conventions for identifiers,
or a string enclosed in double quotes~``\verb+"+''.

Models operate on candidate executions
(see Sec.~\ref{sec:predef}),
instructions are executed in sequence,
until one instruction stops, or until the end of the instruction list.
In that latter case, the model accepts the execution.
The accepted execution is then passed over to the rest of \herd{}
engine, in order to collect final states of locations
and to display pictures.

\iffalse
\subsubsection*{Model options}

Model options control some experimental features of \herd.
More precisely, by default, \herd{} includes a complete
coherence order relation in every candidate execution,
and does not represent initial writes by plain memory
write events. Said otherwise, by default,
model files have options \T{withco} and \T{whithoutinit}.

The generation of all possible coherence orders by \herd{} engine
is a source of inefficiency that can be alleviated by having the
model itself compute the sub-relation of \texttt{co} that is really useful.
Such models must have option \T{withoutco}, so as to
prevent \herd{} engine from generating all coherence orders.
Instead, \herd{} will represent initial writes as plain write events
(\emph{i.e.} option \T{withoutco} implies \T{withinit})
identify last writes in coherence oders, and pass the model a
reduced~\texttt{co} relation, named \texttt{co0}, that will,
for any memory location~$x$,
relate the initial write to~$x$ to all writes to~$x$, and all writes
to~$x$ to the final write to~$x$.
It is then the model responsability  to compute the remainder
of \texttt{co} from the program read events.
The model \ahref{uniproccat.txt}{\texttt{uniproccat.cat}} from the distribution
gives an example of such an advanced model.

The option \T{withinit} can also be given alone so as to instruct
\herd{} engine to represent initial writes as plain write events.
In such a situation, \herd{} will compute complete coherence
orders~\texttt{co} that include those explicit initial writes as
minimal elements.
Observe that the representation of initial writes as events
can be also controlled from the
command-line (see option~\ahrefloc{opt:initwrites}{\opt{-initwrites}})
and that command line settings override model options.
\fi

\subsection{\label{sec:primitive}Primitives}

TODO:

\subsection{\label{sec:library}Library}

\subsubsection*{Standard library}
The standard library is a \cat{} file~\texttt{stdlib.cat}
which all models include by default.
It defines a a few convenient relations that are thus
available to all models.
\begin{idtable}
$\poloc$ & \po{} restricted to the same address &
events are in \po{} and touch the same address, namely $\po \cap \locr$\\
$\rfe$ & external read-from & read-from by different threads, namely $\rf \cap \extr$\\
$\rfi$ & internal read-from & read-from by the same thread, namely $\rf \cap \intr$\\
\end{idtable}

\subsubsection*{Coherence orders}
\bgroup\let\rln\tid
For most models, a complete list of communication relations would
also include \co{} and~\fr{}.
Those can be defined by including the file \texttt{cos.cat}
(see Sec.~\ref{sec:cos}).
\begin{idtable}
$\co$    & coherence     & total order over writes to the same address \\
$\fr$    & from-read     & links a read $r$ to a write $w'$ $\co$-after the write $w$ from which $r$ takes its value \\
\coi, \fri & internal communications &
communication between events of the same thread\\
\coe, \fre & external communications &
communication between events of different threads
\end{idtable}
Notice that the internal and external sub-relations of \co{} and~\fr{}
are also defined.
\egroup

\subsubsection*{Fences}
\bgroup\let\rln\tid
Fence relations denote the presence of a specific
fence (or barrier) in-between two events.
Those can be defined by includinc architecture specific files.
\begin{desctable}{file}{relations}
x86fences.cat &  \mfence{}, \sfence, \lfence{}\\
ppcfences.cat &  \sync, \lwsync, \eieio, \isync, \ctrlisync{}\\
armfences.cat & \dsb, \dmb, \dsbst, \dmbst, \isb{}, \ctrlisb{}\\
mipsfences.cat & \sync\\
aarch64fences.cat & \ldots
\end{desctable}
In other words, models for, say, ARM machines should include the following
instruction:
\begin{verbatim}
include "armfences.cat"
\end{verbatim}

Notice that for the Power (PPC) (resp. ARM) architecture,
an additional relation \ctrlisync{} (res. \ctrlisb) is defined.
The relation \ctrlisync{} reads control +\isync{}.
It means that the branch to the instruction that generates the
second event additionnaly contains
a \texttt{isync} fence preceeding that instruction.
For reference, here is a possible definition of \ctrlisync:
\begin{verbatim}
let ctrlisync  = ctrl & (_ * ISYNC); po
\end{verbatim}

One may define all fence relations by including the file
\texttt{fences.cat}. As a result, fence relations that are
relevant to the architecture of the test being simulated  are properly defined,
while irrelevant fence relations are the empty relation.
This feature proves convenient for writing generic models that apply
to several concrete architectures.
\egroup



\section{Usage of \herd}

\subsection{Arguments}
The command \herd{} handles its arguments like \litmus.
That is, \herd{} interprets its argument as file names.
Those files are either a single litmus test
when having extension \file{.litmus}, or a list of file names
when prefixed by \file{@}.


\subsection{Options}
There are many command line options.
We describe the more useful ones:

\paragraph*{General behaviour}
\begin{description}
\item[{\tt -version}] Show version number and exit.
\item[{\tt -libdir}] Show installation directory and exit.
\item[{\tt -v}] Be verbose, can be repeated to increase verbosity.
\item[{\tt -q}] Be quiet, suppress any diagnostic message.
\item[{\tt -conf <name>}] Read configuration file~\opt{name}.
\ahrefloc{herd:configfile}{Configuration files} have a very simple syntax:
a line ``\textit{opt}\texttt{  }\textit{arg}'' has the same effect as
the command-line option ``\texttt{-}\textit{opt} \textit{arg}''.
\item[{\tt -o <dest>}] Output files into directory \opt{<dest>}.
Notice that \opt{<dest>} must exist.
At the moment \herd{} may output one \texttt{.dot} file per processed test:
the file for test \textit{base}\texttt{.litmus}
is named  \textit{base}\texttt{.dot}.
By default \herd{} does not generate \texttt{.dot} files.
\item[{\tt -suffix <suf>}] Change the name of \texttt{.dot} files
into \textit{base}\textit{suff}\texttt{.dot}. Useful when several \texttt{.dot} files derive from the same test. Default is the empty string (no suffix).
\item[{\tt \aname{opt:gv}{-gv}}] Fork the \ahref{\urlgv}{\prog{gv} Postscript viewer} to display execution diagrams.
\item[{\tt \aname{opt:evince}{-evince}}] Fork the evince document viewer to display execution diagrams. This option provides an alternative to the
{\tt gv} viewer.
\item[{\tt -dumpes <bool>}]
Dump genererated event structures and exit. Default is \opt{false}.
Event structures will be dumped in a \texttt{.dot} file whose
name is determined as usual --- See options \opt{-o} and \opt{-suffix} above.
Optionally the event structures can be displayed with the \opt{-gv} option.
\item[{\tt -unroll <int>}] The setting \opt{-unroll }$n$ performs backwards
jumps $n$ times. This is a workaround for one of \herd{} main limitation:
\herd{} does not really handle loops. Default is~\opt{2}.
\item[{\tt -hexa <bool>}] Print numbers in hexadecimal. Default is \opt{false}
(numbers are printed in decimal).
\end{description}

\paragraph{Engine control}
The main purpose of \herd{} is to run tests on top of memory models.
For a given test, \herd{} performs a three stage process:
\begin{enumerate}
\item Generate candidate executions.
\item For each candidate execution, run the model.
The model may reject or accept the execution.
\item For each candidate execution that the model accepts,
record observed locations and, if so instructed,
a diagram of the execution.
\end{enumerate}
We now describe options that control those three stages.

\begin{description}
\item[{\tt -model (cav12|minimal|uniproc|<filename>.cat)}]
Select model, this option accept one tag or one file name
with extension~\texttt{.cat}.
Tags instruct \herd{} to select an internal model,
while file names are read for a model definition.
Documented model tags are:
\begin{itemize}
\item \opt{cav12}, the model of~\cite{mms12} (Power);
\item \opt{minimal}, the minimal model that allows all executions;
\item \opt{uniproc}, the uniproc model that checks single-thread correctness.
\end{itemize}

In fact, \herd{} accepts potentially infinitely many models,
as models can given in text files in an adhoc language described in
Sec.~\ref{herd:language}.
The \herd{} distribution includes several such models:
\afile{herd.cat}, \afile{minimal.cat}, \afile{uniproc.cat}
and \afile{x86tso.cat} are
text file versions of the homonymous internal models, but may
produce pictures that show different relations.
Model files are searched according to  the same
\ahrefloc{herd:searchpath}{rules}
as configuration files.
Some architectures have a default model:
\opt{arm.cat} model for ARM, \opt{ppc.cat} model for PPC,
\opt{x86tso.cat} for X86, and \opt{mips-pso.cat} for MIPS.

\item[{\tt -through (all|invalid|none)}]
Let additional executions reach the final stage of \herd{} engine.
This option permits users to generate pictures of forbidden executions, which
are otherwise rejected at an early stage of \herd{} engine --- see Sec.~\ref{show:forbidden}.
Namely, the default~``\opt{none}'' let only valid (according to the
active model) executions through.
The behaviour of this option differs between internal and text file models:
\begin{itemize}
\item For internal models:
the tag~\opt{all} let all executions go through;
while the tag~\opt{invalid} will reject executions that violate uniproc,
while letting other forbidden execution go through.
\item Text file models: the tags \opt{all} and~\opt{invalid} let all
executions go through. For such models, a more precise control over
executions that reach \herd{} final phase can be achieved
with the option~\opt{-skipcheck} --- see  next option.
\end{itemize}
Default is~\opt{none}.

\item[{\tt -skipchecks <\textit{name}$_1$,\ldots,\textit{name}$_n$>}]
\aname{skipchecks}{This option}
applies to text file models. It instructs \herd{} to ignore
the outcomes of the given checks. For the option to operate, checks must
be named in the model file with the \texttt{as }\textit{name} construct --
see Sec.~\ref{name:check:def}.
Notice that the arguments to \opt{-skipcheck} options cumulate.
That is, ``\opt{-skipcheck }\textit{name}$_1$ \opt{-skipcheck }\textit{name}$_2$'' acts like ``\opt{-skipcheck }\textit{name}$_1$\texttt{,}\textit{name}$_2$''. 

\item [{\tt -strictskip <bool>}] Setting this option (\opt{-strictskip true}),
will change the behaviour of the previous option \opt{-skipcheck}:
it will let executions go through when the skipped checks yield
false and the unskipped checks yield true. This option comes handy
when one want to observe the executions that fail one (or several) checks
while passing others. Default is \opt{false}.

\item[{\tt -optace <bool>}] Optimise the axiomatic candidate execution stage.
When enabled by \opt{-optace true},  \herd{} does not generate candidate
executions that fail the uniproc test. The default is ``\opt{true}''
for internal models (except the minimal model), and ``\opt{false}'' for
text file models. Notice that \opt{-model uniproc.cat}
and \opt{-model minimal.cat -optace true} should yield identical results,
the second being faster.
Setting \opt{-optace true} can lower the execution time significantly,
but one should pay attention not to design models that forget the uniproc
condition.

\item[{\tt \aname{opt:show}{-show} (prop|neg|all|cond|wit|none)}]
Select execution diagrams for picture display and generation.
Execution diagrams are shown according to
the final condition of test. The final condition is a quantified boolean
proposition \verb+exists +$p$, \verb+~exists +$p$, or \verb+forall +$p$.
The semantics of recognised tags is as follows:
\begin{itemize}
\item \opt{prop} Picture executions for which $p$ is true.
\item \opt{neg} Picture executions for which $p$ is false.
\item \opt{all}  Picture all executions.
\item \opt{cond} Picture executions that validate  the condition,
\emph{i.e.} $p$ is true for \verb+exists+ and \verb+forall+, and false
for \verb+~exists+.
\item \opt{wit} Picture ``\emph{interesting}'' executions,
\emph{i.e.} $p$ is true for \verb+exists+ and \verb+~exists+,
and false for \verb+forall+.
\item \opt{none} Picture no  execution.
\end{itemize}
Default is \opt{none}.

\item[{\tt \aname{opt:initwrites}{-initwrites} <bool>}]
Represent init writes as plain write events, default is \opt{false} except
for specifically tagged generic models --- see ``Model options''
in Sec.~\ref{language:model}.


\end{description}

\paragraph*{Discard some observations}
Those options intentionally omit some of the final states that \herd{} would
normally generate.

\begin{description}
\item[{\tt -speedcheck (false|true|fast)}] 
\aname{speedcheck:opt}{When} enabled by \opt{-speedcheck true}
or \opt{-speedcheck fast}, attempt to settle the test condition.
That is, \herd{} will
generate a subset of executions (those named ``\emph{interesting}'' above)
in place of all executions.
With setting \opt{-speedcheck fast},
\herd{} will additionally stop as soon as a condition \verb+exists +$p$ is validated, and as soon as a condition \verb+~exists +$p$ or
\verb+forall +$p$ is invalidated. Default is \opt{false}.

\item[{\tt -nshow <int>}]
Stop once \verb+<int>+ pictures have been collected. Default is to
collect all (specified, see option \ahrefloc{opt:show}{\opt{-show}}) pictures.
\end{description}

\paragraph*{Control \texttt{dot} pictures}
These options control the content of DOT images.

We first describe options that act at the general level.
\begin{description}
\item[{\tt -graph (cluster|free|columns)}] Select main mode for graphs.
See Sec.~\ref{mode:example}. The default is \opt{cluster}.
\item[{\tt -dotmode (plain|fig)}] The setting \opt{-dotmode fig}
produces output that includes the proper escape
sequence for translating \texttt{.dot} files
to \texttt{.fig} files (\emph{e.g.} with \texttt{dot -Tfig\ldots}).
Default is \opt{plain}.
\item[{\tt -dotcom (dot|neato|circo)}] Select the command that formats
graphs displayed by the \ahrefloc{opt:gv}{\opt{-gv}} option.
The default is \opt{dot} for the \opt{cluster} and~\opt{free} graph modes,
and \opt{neato} for the \opt{columns} graph mode.

\item[{\tt -showevents (all|mem|noregs)}] Control which events are
pictured:
\begin{itemize}
\item \opt{all} Picture all events.
\item \opt{mem} Picture memory events.
\item \opt{noregs} Picture all events except register events,
\emph{i.e.} memory, fences and branch events.
\end{itemize}
Default is \opt{noregs}.

\item[{\tt -showinitwrites <bool>}] Show initial write events
(when existing, see option~\ahrefloc{opt:initwrites}{-initwrites})
in pictures. Default is \opt{true}.

\item[{\tt -mono <bool>}] The setting \opt{-mono true} commands monochrome
pictures. This option acts upon default color selection. Thus, it
has no effect on colors given explicitely with the
\ahrefloc{opt:edgeattr}{\opt{-edgeattr}} option.

\item[{\tt -scale <float>}]
Global scale factor for graphs in \opt{columns} mode.
Default is \opt{1.0}.
\item[{\tt -xscale <float>}]
Global scale factor for graphs in \opt{columns} mode, x direction.
Default is \opt{1.0}.
\item[{\tt -yscale <float>}]
Global scale factor for graphs in \opt{columns} mode, y direction.
Default is \opt{1.0}.
\item[{\tt -showthread <bool>}] Show thread numbers in figures.
In \opt{cluster} mode where the events of a thread are clustered,
thread cluster have a label.
In \opt{free} mode \textrel{po} edges are suffixed by a thread number.
In \opt{columns} mode,  columhs have a header node that shows
the thread number. Default is~\opt{true}.
\item[{\tt -texmacros <bool>}] Use latex commands in some text of pictures.
If activated (\opt{-showthread true}), thread numbers are shown as
\verb+\myth{+$n$\verb+}+. Assembler instructions are locations in nodes
are argument to an \verb+\asm+ command. It user responsability to define
those commands in their \LaTeX{} documents that include the pictures.
Possible definitions are \verb+\newcommand{\myth}[1]{Thread~#1}+
and \verb+\newcommand{\asm}[1]{\texttt{#1}}+.
Default is~\opt{false}.
\end{description}

A few options control picture legends.
\begin{description}
\item[{\tt -showlegend <bool>}]
Add a legend to pictures. By default legends show the test name and
a comment from the executed model.
This comment is the first item
of model syntax  --- see Sec~\ref{language:model}.
Default is~\opt{true}.
\item[{\tt -showkind <bool>}]
Show test kind in legend.
The kind derive from the quantifier of test final condition,
kind \texttt{Allow} being \verb+exists+,
kind \texttt{Forbid} being \verb+~exists+,
and kind \texttt{Require} being \verb+forall+.
Default is~\opt{false}.
\item[{\tt -shortlegend <bool>}]
Limit legend to test name. Default is~\opt{false}.
\end{description}

A few options control what is shown in nodes
and on their sizes, \emph{i.e.}
on how events are pictured.
\begin{description}
\item[{\tt -squished <bool>}] The setting \opt{-squished true} drastically
limits the information displayed in graph nodes. This is usually what
is wanted in modes \opt{free} and~\opt{columns}. Default is~\opt{false}.
\item[{\tt -fixedsize <bool>}] This setting is meaningfull in
\opt{columns} graph mode and for squished nodes. When set by
\opt{-fixedsize true} it forces node width to be $65\%$ of the space
between columns. This may sometime yield a nice edge routing. Default is~\opt{false}
\item[{\tt -extrachars <float>}] This setting is meaningful in
\opt{columns} graph mode and for squished nodes.
When the size of nodes is not fixed (\emph{i.e.} \opt{-fixedsize false} and default), \herd{} computes the width of nodes by counting caracters in node
labels and scaling the result by the font size.
The setting \opt{-extrachars~}$v$ commands adding the value $v$ before scaling.
Negative values are of course accepted. Default is \opt{0.0}.
\item[{\tt -showobserved <bool>}] Highlight observed memory read events with
stars ``\texttt{*}''. A memory read is observed when the value it reads
is stored in a register that appears in final states.
Default is~\opt{false}.
\item[{\tt -brackets <bool>}] Show brackets around locations. Default
is~\opt{false}.
\end{description}

Then we list options that offer some control on which edges are shown.
We recall that the main controls over the shown and unshown edges are
the \verb+show+ and \verb+unshow+ directives in model definitions ---
see Sec.~\ref{show:def}.
However, some edges can be controled only with options (or configuration
files) and the \opt{-unshow} option proves convenient.
\begin{description}
\item[{\tt -showpo <bool>}] Show program order (\textrel{po}) edges.
Default is~\opt{true}.
Default is~\opt{false}.
\item[{\tt -showinitrf <bool>}] Show read-from edges from initial state.
Default is~\opt{false}.
\item[{\tt -showfinalrf <bool>}] Show read-from edges to the final state,
\emph{i.e} show the last store to locations. Default is~\opt{false}.
\emph{i.e} show the last store to locations. Default is~\opt{false}.
\item[{\tt -showfr <bool>}] Show from-read edges. Default is~\opt{true}.
\item[{\tt -doshow <\textit{name}$_1$,\ldots,\textit{name}$_n$>}]
Do show edges labelled with \textit{name}$_1$,\ldots,\textit{name}$_n$.
This setting applies when names are bound in model definition.
\item[{\tt -unshow <\textit{name}$_1$,\ldots,\textit{name}$_n$>}]
Do not show edges labelled with \textit{name}$_1$,\ldots,\textit{name}$_n$.
This setting applies at the very last momement and thus cancels any
\verb+show+ directive in model definition and any \opt{-doshow} command
line~option.
\end{description}
Other options offer some control over some of the attributes defined in
\ahref{\urlgraphviz}{Graphviz software} documentation.
Notice that the controlled attributes are omitted from DOT files
when no setting is present.
For instance in the absence of a \opt{-spline <tag>} option, \herd{}
will generate no definition for the \texttt{splines} attribute thus
resorting to DOT tools defaults.
Most of the following
options accept the \opt{none}~argument that restores their
default behaviour.
\begin{description}
\item[{\tt -splines (spline|true|line|false|polyline|ortho|curved|none)}]
Define the value of the \texttt{splines} attribute. Tags are replicated in
output files as the value of the attribute, except for \opt{none}.
\item[{\tt -margin <float|none>}] Specifies the \texttt{margin} attribute of graphs.
\item[{\tt -pad <float|none>}] Specifies the \texttt{pad} attribute of graphs.
\item[{\tt -sep <string|none>}] Specifies the \texttt{sep} attribute of graphs.
Notice that the argument is an arbitray string, so as to allow DOT general
syntax for this attribute.
\item[{\tt -fontname <string|none>}] Specifies the graph fontname attribute.
\item[{\tt -fontsize <int|none>}] Specifies the fontsize attribute~$n$ of all
text in the graph.
\item[{\tt -edgefontsizedelta <int>}] option \opt{-edgefontsizedelta }$m$ sets
the fontsize attributes of edges to $n+m$, where $n$ is the argument to
the \opt{-fontsize} option. Default is \opt{0}. This option has  no effect if
fontsize is unset.
\item[{\tt -penwidth <float|none>}] Specifies the \texttt{penwidth} attribute of
edges.
\item[{\tt -arrowsize <float|none>}] Specifies the \texttt{arrowsize}
attribute of edges.
\item[{\tt \aname{opt:edgeattr}{-edgeattr} <label,attribute,value>}]
Give value \opt{value} to attribute \opt{attribute}  of all edges labelled
\opt{label}. This powerful option  permits alternative styles for edges.
For instance, the \textrel{ghb} edges of the diagrams of this document
are thick purple (blueviolet) arrows thanks to the settings:
\opt{-edgeattr ghb,color,blueviolet}
\opt{-edgeattr ghb,penwidth,3.0}
\opt{-edgeattr ghb,arrowsize,1.2}. Notice that the settings performed
by the \opt{-edgeattr} option override other settings.
This option has no default.
\end{description}

\paragraph*{Change input}
Those options are the same as the ones
of~\litmus{} --- see Sec.~\ref{change:input}.

\begin{description}
\item[{\tt -names <file>}] Run \herd{} only on tests whose names are
listed in \texttt{<file>}.
\item[{\tt -rename <file>}] Change test names.
\item[{\tt -kinds <file>}] Change test kinds.
This amonts to changing the quantifier of final conditions, with
kind \texttt{Allow} being \verb+exists+,
kind \texttt{Forbid} being \verb+~exists+
and kind \texttt{Require} being \verb+forall+.
\item[{\tt -conds <file>}] Change the final condition of tests.
This is by far the most useful of these options:
in combination with option \opt{-show prop} it permits a fine grain
selection of execution pictures --- see Sec.~\ref{example:invalid}.
\end{description}

\subsection{\label{herd:configfile}{Configuration files}}
The syntax of configuration files is minimal:
lines ``\textit{key} \textit{arg}'' are interpreted
as setting the value of parameter~\textit{key} to \textit{arg}.
Each parameter has a corresponding option,
usually \opt{-}\textit{key}, except for the single letter
option \opt{-v} whose parameter is \opt{verbose}.

As command line option are processed left-to-right,
settings from a configuration file (option \opt{-conf})
can be overridden by a later command line option.
Configuration files will be used mostly for controling pictures.
Some configuration files are
are present in the distribution.
As an example, here is the configuration file \afile{apoil.cfg},
which can be used to display images in \opt{free} mode.
\verbatiminput{apoil.cfg}
The configuration above is commented with line comments that starts
with ``\verb+#+''.
The above configuration file comes handy to eye-proof model output,
even for relatively complex tests, such as \atest{IRIW+lwsyncs}
and \atest{IRIW+syncs}:
\begin{verbatim}
% herd -conf apoil.cfg -show prop -gv IRIW+lwsyncs.litmus
% herd -through invalid -conf apoil.cfg -show prop -gv IRIW+syncs.litmus
\end{verbatim}
We run the two tests on top of the default model that computes,
amongst others, a \texttt{prop} relation. The model rejects executions
with a cyclic \textrel{prop}.
One can then see that the relation \textrel{prop} is acyclic
for \ltest{IRIW+lwsyncs} and cyclic for \ltest{IRIW+syncs}:
\begin{center}\img{IRIW+lwsyncs+APOIL}\quad\quad\img{IRIW+syncs+APOIL}\end{center}
Notice that we used the option~\opt{-through invalid} in the case
of \ltest{IRIW+syncs} as we would otherwise have no image.


\subsection{\label{herd:searchpath}File searching}
Configuration and model files are searched first in the current directory;
then in any directory specified
by setting the shell environment variable \texttt{HERDDIR};
and then in herd installation directory, which is defined
while compiling~\herd.

%\section{Extensions to Herd}
%
%\begin{quote}\it
%This section describes several extensions to \herd{} that have been implemented by Tyler Sorensen and John Wickerson in collaboration with the original authors. In due course, the contents of this section will probably be merged into the previous sections to form a cohesive manual.
%\end{quote}
%
%\subsection{Additional command line options}
%
%\begin{description}
%\item[{\tt \aname{opt:dumplem}{-dumplem}}] Convert the given herd model to Lem format, and exit. The resultant Lem file is generated on {\tt stdout}. 
%\begin{quote}\emph{Note.} It is necessary to provide a litmus test when invoking \herd{} in this way, even though the litmus test will not be examined. This is due to a minor technical problem.\end{quote}
%\end{description}
%
%
%
%\subsection{Curried function application}
%
%A test-instruction is used to enforce a contract between the programming language and the programmer. There are two types: \emph{provided conditions} and \emph{undefined-unless conditions}. The latter are indicated with the \synt{\T{undefined\_unless}} keyword. A provided condition is an obligation on the programming language; that is, the programmer can assume that every execution of their program will satisfy every provided condition. An undefined-unless condition is an obligation on the programmer; that is, the compiler-writer can assume that every execution of the program will satisfy every undefined-unless condition. In other words, if a program has an execution that violates an undefined-unless condition, then its behaviour is completely undefined.

%The syntax for instructions thus becomes:
%
%\begin{syntax}
%\nt{instruction} \is{} \ldots
%\alt \T{undefined\_unless} \NT{check} \NT{expr} \boption \T{as} \NT{id}\eoption
%\end{syntax}

%\subsection{Additional identifiers}
%
%\begin{quote}\it
%This is an appendix to Section~\ref{language:identifier}.
%\end{quote}

%\paragraph*{Pre-defined relations}
%
%Here are some more pre-defined relations.
%\begin{idtable}
%{\tt unv}    & universal relation & relates every event in the structure to every other event %\\
%{\tt int-$s$} & internal at given scope & \emph{(applicable only to scoped memory models)} relates events that are in the same part of the execution hierarchy \\
%{\tt ext-$s$} & external at given scope & \emph{(applicable only to scoped memory models)} relates events that are in different parts of the execution hierarchy \\
%\end{idtable}
%Here, $s$ ranges over the following values:
%
%\begin{center}
%\begin{tabular}{ll}
%value of $s$ & description (in OpenCL terminology) \\ 
%\hline
%{\tt wi}, {\tt thread} & work-item \\
%{\tt sg}, {\tt warp} & sub-group \\
%{\tt wg}, {\tt block}, {\tt cta} & work-group \\
%{\tt kernel} & kernel \\
%{\tt dev} & device \\
%\end{tabular}
%\end{center}
%
%For example, {\tt int-cta} relates all events that are in the same work-group, while {\tt ext-wi} relates all events that are in different work-items (threads).
%
%We provide the following additional fence relations: \verb"membar.cta", \verb"membar.gl", \verb"membar.sys" (PTX). C++ and OpenCL fences do not appear in this list because those fences are modelled as events rather than relations. By modelling these fences as events, we are better able to attach parameters to them, such as the memory-order (C++ and OpenCL) and the memory-scope (OpenCL).
%
%In C++ models, the following relations are pre-defined.
%\begin{idtable}
%{\tt asw}    & additional synchronises-with & links every initial write to every event that is not an initial write \\
%{\tt lo}     & lock order & a total order on all mutex operations (similar to {\tt co}, but for mutexes instead of writes) \\
%\end{idtable}

%\paragraph*{Pre-defined sets}
%Here are some pre-defined sets, available in all models.
%\begin{idtable}
%{\tt \_}    & universal set & the set of all events in the structure \\
%{\tt R} & read events & set of all reads from memory \\
%{\tt W} & write events & set of all writes to memory \\
%{\tt M} & memory events & set of all reads from and writes to memory \\
%{\tt B} & barrier events & is a barrier \\
%{\tt A} & atomic events & is an atomic event \\
%{\tt P} & plain events & is not an atomic event \\
%{\tt I} & initial writes & is an initial write event \\
%\end{idtable}

%Having defined these sets, it is now possible to write expressions of the form {\tt RW($e$)} as {\tt [R * W] \& $e$}. However, the simulation in the latter case may be less efficient, owing to the need to construct the intermediate relation {\tt [R * W]}.

%In C++ and OpenCL models, the following sets are pre-defined.
%
%\begin{idtable}
%{\tt rmw} & read-modify-writes & the set of all read-modify-write events \\
%{\tt brmw} & blocked read-modify-writes & events representing an attempted read-modify-write operation that has become stuck \\
%{\tt F} & fences & the set of all fences \\
%{\tt acq} & acquire & atomic events with ``acquire'' memory-order \\
%{\tt rel} & release & atomic events with ``release'' memory-order \\
%{\tt acq\_rel} & acquire/release & atomic events with ``acquire/release'' memory-order \\
%{\tt rlx} & relaxed & atomic events with ``relaxed'' memory-order \\
%{\tt sc} & sequentially consistent & atomic events with ``sequentially consistent'' memory-order \\
%{\tt na} & non-atomics & non-atomic events \\
%\end{idtable}

%In C++ models, the following sets are pre-defined.
%
%\begin{idtable}
%{\tt lk} & locks & the set of all lock events \\
%{\tt ls} & successful locks & the set of all successful lock events \\
%{\tt ul} & unlocks & the set of all unlock events \\
%{\tt con} & consume & atomic events with ``consume'' memory-order \\
%{\tt atomicloc} & atomic locations & events acting on atomic locations \\
%{\tt nonatomicloc} & non-atomic locations & events acting on non-atomic locations \\
%{\tt mutexloc} & mutex locations & events acting on mutex locations \\
%\end{idtable}

%In OpenCL models, the following sets are pre-defined.
%
%\begin{idtable}
%{\tt gF} & global fences & the set of all global fences \\
%{\tt lF} & local fences & the set of all local fences \\
%{\tt wi} & work-item scope & events with ``work-item'' memory-scope \\ 
%{\tt sg} & sub-group scope & events with ``sub-group'' memory-scope \\ 
%{\tt wg} & work-group scope & events with ``work-group'' memory-scope \\ 
%{\tt dev} & device scope & events with ``device'' memory-scope \\ 
%{\tt all\_dev} & all-devices scope & events with ``all\_svn\_devices'' memory-scope \\ 
%\end{idtable}


\clearpage
\part{Converting\label{part:jingle} tests with \jingle}
\cutname{jingle.html}

The tool \jingle{} is a litmus test convertor.
Users may define convertions rules of instructions and have the tool convert
tests meant for some architecture to ones for another.

\section{Writing convertions rules}
For the tool to know how to convert each instruction as for allowing users to
freely define optimisations and workarounds, a file gathering all the allowed
convertions is necessary.

\subsection{Minimal conversions}
Such files are writen in a simple manner, first we tell from which to which
architectures the file stands for, then associate patterns of instructions
for the source architecture to assumed equivalent patterns
for the target one.\newline
Since the main purpose of the tool is the use of our abstract language LISA,
the examples in this tutorial will be conversions from LISA to a concrete
assembly language (namely AArch64).

\begin{verbatim}
LISA to AArch64

"r[] %x %y" -> "LDR %x,[%y]"

"w[] %x %y" -> "STR %y,[%x]"
\end{verbatim}

The first line \textbf{must} be of the form
\verb+[source_arch] to [target_arch]+.\newline
Each rule is code from the given architectures in double-quotes,
separated by an arrow (or the key word \verb+maps to+).
Such code is the same as seen in any well-formed tests with some
abstract expressions.

There would be no point to have a rule for each different usage of a same
usage of instruction therefore we can express various elements in the code
by identification.
Representing registers with symbolics is the most common use of identification:
in the rules above, all that matters is to have the registers \verb+x+
and \verb+y+ of the source program be at the right place in the converted test.

\subsection{Multiple instructions pattern and other identifications}
Sometimes, a single instruction with a couple of symbolic register
is not nearly enough to express every conversion in a good fashion.

A single rule ought to be enough for anybody to understand:
\begin{verbatim}
"w[] x &c" -> "MOV %tmp,&c;
               STR %tmp,[%x]"
\end{verbatim}

In AArch64, it is not possible to directly store a constant value in memory,
thus we have to express the LISA instruction in two AArch64 instructions
with a register picked on the fly.

This example illustrates at the same time:
\begin{itemize}
\item how to identify addresses for languages which can directly mention it,
such as LISA, where they are simply represented by a name. A address can
be converted to a register which would contain the said address value.
\item how to identify constants. For every assembly languages, a constant can be
represented by a name preceded by the character \verb+&+.
\item how to define patterns with multiple instructions by separating them
with a semicolon.
\item that identifiers are not necessarily bound to appear on both side of
a rule if they are only needed in one.
\end{itemize}

In case of ambiguity, \jingle{} choose the rule to apply according to their order
in the file: the higher the rule, the higher its priority.

\subsection{Multiple level pattern and structured languages}
The last section covers what is necessary to convert most tests from any assembly language
to another. However, we might want to allow our tool to work on higher level languages.

The suite currently support a relevant subset of the C language.
That means our tool must not only convert sequences of instructions but also
potential control structures.

For this purpose, we must allow the expression of chunks of code:

\begin{verbatim}
C to LISA

"if(x==constvar:c)
   codevar:t;
 else
   codevar:e;"     -> "mov %test (eq %x &c);
                      b[] %test then;
                      codevar:e;
                      b[] 1 end;
                      then : codevar:t;
                      end :"
\end{verbatim}

This awfully looks like a compilation process (and it is!), but in practice,
conditionals are used to express control dependencies which can have a much
simpler form in assembly code.

Now, the important point here is the use of \verb+codevar:+ to state that
arbitrary code is expected. Such code will also be converted by the same
given set of rules, thus allowing us to convert arbitrarily deep code.

Notice that labels too are subject to identification and that is
perfectly fine to end a pattern with it since the tool will see
a nop-like instruction.

The special keyword \verb+constvar:+ is used in C for the obvious reason
that \verb+&+ has an entirely different meaning.

\section{Rewriting algorithm}
Now with a well defined file, we can let the burden of converting our
thousands of tests to \jingle{}.

In order to fully understand its behavior, we shall explore
more in detail its mechanisms.

\subsection{Rule application and substitution mechanism}
The rules we define are nothing but generic patterns, for them
to hold any meaning we have to find an application in the source program.
Such application is simply an instance of the conversion of a part
of the source program. The source part must match the pattern in the
left side of the rule, the instance of the conversion is the code
defined in the right side plus a set of what we call substitutions.

The substitutions are the link between the identifiers in the rule
and their actual representations in both the source and target programs.

To roughly formalize,
$App(R, P) = \left(R_{right}, \left\{\left(id,Src_{id},Tgt_{id}\right) \mid id \text{is an identifier of} R\right\}\right)$
would be the application of the rule $R$ on a part of the source program $P$,
where $P$ is a possible instance of $R_{left}$ and $Src_{id}$ ($Tgt_{id}$) is the source
(respectively the target) representation.

\subsection{Linear processing}
Now that we have a first step of local rewriting, we want to convert an entire
program. Thanks to relative simplicity of the supported languages, decomposing
a source program linearly is a good enough approach for our needs.

The process can be divided in two steps:
\begin{itemize}
\item The actual decomposition of the source program into a sequence of rule applications.
\item The recomposition into a target program by instancing the previous sequence
according to its substitutions.
\end{itemize}

\paragraph{Decomposing}
Starting at the beginning, the first rule (in the priority order) to match
a part of the program, in other words, the first possible application od the rule,
is considered and the process repeat itself on the rest of the program until
eventually it has been completely consumed. We end up with a sequence of
patterns and associed substitutions.

A recursive definition would be:
\begin{align*}
& Decomp(\bullet,Rs) = \bullet && where R is the highest possible element of the rule set Rs\\
& Decomp(P | Source_{rest},Rs) = App(R,P) | Decomp(Source_{rest},Rs) && and P \in Instances(R_{left})
\end{align*}

Of course, we assume the given rule set is sufficient to assure that all part of the
program will be matched. If not, users have to refine it.

\paragraph{Recomposing}
With the result of the previous step, which already looks like a converted program,
we have to actually substitute the abstract identifiers for their representation
in the target language given by the associed substitutions, for each part.
Then simply append the results to one another in order.

\begin{align*}
& Recomp(\bullet) = \bullet\\
& Recomp((R_{right}, Subs) | Parts_{rest}) = R_{right}[\{id \mapsto Tgt_{id} \mid (id,Src_{id},Tgt_{id})\in Subs\}] | Recomp(Parts_{rest})
\end{align*}

\begin{remark}
The substitutions can also include code since patterns allow it. This code is converted following the same
procedure thus any code substitution have the form $(id,Src_{id},Recomp(Decomp(Src_{id},Rs)))$.
\end{remark}

\subsection{Representation coherence and environment}
There is one key aspect that we have yet to cover. Until now, the coherence between the source et target representation of
a substitution in regards to the others was assumed, i.e.:
\begin{equation*}
\forall id_1,id_2, \; Src_{id_1} = Src_{id_2} \Rightarrow Tgt_{id_1} = Tgt_{id_2}
\end{equation*}
However, if this property in the substitutions of a single application is given by the rule itself,
ensuring it between different applications is not as obvious because it would not make sense to
compare pattern identifiers from different rules. Thus, we need to keep track of any $Src,Tgt$
association made by applications through the whole program.

To do so, we define a global environment that preserves the property by delivering a target
language representation for each source value:
\begin{align*}
& Get\_repr(\Gamma,Src_{id}) = fresh\_repr(Src_{id}) & if Src_{id} \notin \Gamma\\
& Get\_repr(Src_{id}\mapsto Tgt_{id}\mid\Gamma,Src_{id}) = Tgt_{id} &
\end{align*}

Nothing exotic here, this is a part of the application of a rule as it is supposed
to be the only safe way to obtain a target representation.

\section{Usage of \jingle{}}

\subsection{Arguments}
The command \jingle{} handles its arguments as file names, just as \herd{}.
Those files are either a single litmus test when having extension .litmus,
or a list of file names when prefixed by @.

\subsection{Options}
There is one option that must always be used:

\begin{description}
\item[{\tt -theme <name>}] Read the conversion rules file <name>.
By convention, such files have the extention \verb+.theme+.
\end{description}

\paragraph{General behavior}
\begin{description}
\item[{\tt -v}] Be verbose.
\item[{\tt -o <dest>}] Instead of printing the result on the standard output,
output test files in the existing <dest> directory. Those files have the same
name as the input tests.
\end{description}

\subsection{Regarding conversion errors}
When the tool fails to find a conversion in a program, it will print the remaining instructions.
It makes easy for the user to pin down missing rules in his \verb+.theme+ file
as the first instruction printed is likely the one that cannot be matched.

Using those error might be helpful to build such file instead of trying to figure it out
as a whole beforehand.


\clearpage
\part{Some \label{part:examples} examples}
\cutname{examples.html}
In the following experiment reports we describe both how we generate tests
and how we run them on various machines under various conditions.

\section{Running several tests at once, changing critical parameters}
In this section we describe an experiment on changing the stride
(cf Sec.~\ref{defstride}).
This usage pattern applies to many situations, where a series of
test is compiled once and run many times under changing conditions.

We assume a directory \ahref{tst-x86}{\texttt{tst-x86}}%
\ifhevea{} (\ahref{tst-x86.tar}{archive})\fi,
that contains a series of litmus tests
and an index file~\ahref{tst-x86/@all}{\file{@all}}.
Those tests where produced by the \prog{diy} tool (see Sec.~\ref{diy:intro}).
They are two thread tests that exercise various relaxed behaviour
of x86~machines.
More specifically, \diy{} is run as ``\texttt{diy -conf X.conf}'',
where \ahref{tst-x86/X.conf}{\texttt{X.conf}} is the
following configuration file
\verbatiminput{tst-x86/X.conf}
As described in Sec.~\ref{diy:usage}, \diy{} will generate all
\emph{critical} cycles of size at most 5, built from the given lists
of candidate relaxations, spanning other two threads,
and including at least one occurrence of PodWR, [Rfi,PodRR] or both.
In effect, as x86 machines follow the TSO model that relaxes write to read
pairs, all produced tests should \emph{a priori} validate.


We test some \texttt{x86-64} machine, using the following 
\ahref{x86-64.cfg}{\texttt{x86-64.cfg}} \litmus{} configuration file:
\verbatiminput{x86-64.cfg}
The number of available logical processors is unspecified,
it thus defaults to~$1$,
leading to running one instance of the test only (cf parameter $a$ in Sec.~\ref{defa})

We invoke \litmus{} as follows,
where \texttt{run} is a pre-existing empty directory:
\begin{verbatim}
% litmus -mach x86-64 -o run tst-x86/@all
\end{verbatim}
The directory \texttt{run} now contains C-source files for the tests,
as well as some additional files:
\begin{verbatim}
% ls run
comp.sh   outs.c  README.txt  utils.c  X000.c  X002.c  X004.c  X006.c
Makefile  outs.h  run.sh      utils.h  X001.c  X003.c  X005.c
\end{verbatim}
One notices a short \texttt{README.txt} file, two scripts to compile
(\texttt{com.sh}) and run the tests (\texttt{run.sh}), and a \texttt{Makefile}.
We use the latter to build test executables:
\begin{verbatim}
% cd run
% make -j 8
gcc -Wall -std=gnu99 -fomit-frame-pointer -O2 -m64 -pthread -O2 -c outs.c
gcc -Wall -std=gnu99 -fomit-frame-pointer -O2 -m64 -pthread -O2 -c utils.c
gcc -Wall -std=gnu99 -fomit-frame-pointer -O2 -m64 -pthread -S X000.c
...
gcc -Wall -std=gnu99 -fomit-frame-pointer -O2 -m64 -pthread  -o X005.exe outs.o utils.o X005.s
gcc -Wall -std=gnu99 -fomit-frame-pointer -O2 -m64 -pthread  -o X006.exe outs.o utils.o X006.s
rm X005.s X004.s X006.s X000.s X001.s X002.s X003.s
\end{verbatim}
This builds the seven tests \texttt{X000.exe} to~\texttt{X006.exe}.
The size parameters (\verb+size_of_test = 1000+ and
\verb+number_of_run = 10+) are rather small, leading to fast tests:
\begin{verbatim}
% ./X000.exe 
Test X000 Allowed
Histogram (2 states)
5000  :>0:EAX=1; 0:EBX=1; 1:EAX=1; 1:EBX=0;
5000  :>0:EAX=1; 0:EBX=0; 1:EAX=1; 1:EBX=1;
No
...
Condition exists (0:EAX=1 /\ 0:EBX=0 /\ 1:EAX=1 /\ 1:EBX=0) is NOT validated
...
Observation X000 Never 0 10000
Time X000 0.01
\end{verbatim}
However, the test fails, in the sense that the relaxed outcome targeted by
\texttt{X000.exe} is not observed, as can be seen quite easily from
the ``\texttt{Observation Never\ldots}'' line above .


To observe the relaxed outcome,
it happens it suffices to change the stride value to~$2$:
\begin{verbatim}
% ./X000.exe -st 2
Test X000 Allowed
Histogram (3 states)
21    *>0:EAX=1; 0:EBX=0; 1:EAX=1; 1:EBX=0;
4996  :>0:EAX=1; 0:EBX=1; 1:EAX=1; 1:EBX=0;
4983  :>0:EAX=1; 0:EBX=0; 1:EAX=1; 1:EBX=1;
Ok
...
Condition exists (0:EAX=1 /\ 0:EBX=0 /\ 1:EAX=1 /\ 1:EBX=0) is validated
...
Observation X000 Sometimes 21 9979
Time X000 0.00
\end{verbatim}

We easily perform a more complete experiment with the stride changing
from~$1$ to~$8$, by running the \texttt{run.sh} script,
which transmits its command line options to all test executables:
\begin{verbatim}
% for i in $(seq 1 8)
> do
> sh run.sh -st $i > X.0$i
> done
\end{verbatim}
Run logs are thus saved into files \texttt{X.01} to~\texttt{X.08}.
The following table summarises the results:
\begin{center}\let\handletest\xhandletest
\begin{tabular}{|l||r|r|r|r|r|r|r|r|}
\hline
& X.01 & X.02   & X.03 & X.04  & X.05 & X.06  & X.07 & X.08   \\
\hline
\hline
\handletest{X000} &  0/10k& 21/10k & 0/10k& 17/10k& 0/10k& 19/10k& 2/10k& 40/10k \\
\hline
\handletest{X001} &  0/10k& 108/10k& 0/10k& 77/10k& 2/10k& 29/10k& 0/10k& 29/10k \\
\hline
\handletest{X002} &  0/10k& 2/10k  & 0/10k& 6/10k & 0/10k& 7/10k & 0/10k& 5/10k  \\
\hline
\handletest{X003} &  0/10k& 4/10k  & 2/10k& 1/10k & 0/10k& 5/10k & 0/10k& 11/10k \\
\hline
\handletest{X004} &  0/10k& 4/10k  & 0/10k& 33/10k& 0/10k& 10/10k& 0/10k& 8/10k  \\
\hline
\handletest{X005} &  0/10k& 1/10k  & 0/10k& 0/10k & 0/10k& 5/10k & 0/10k& 4/10k  \\
\hline
\handletest{X006} &  0/10k& 8/10k  & 0/10k& 9/10k & 0/10k& 11/10k& 1/10k& 12/10k \\
\hline
\end{tabular}


\end{center}
For every test and stride value cells show how many times the targeted relaxed
outcome was observed/total number of outcomes.
One sees that even stride value perfom better --- noticeably $2$, $6$ and~$8$.
Moreover variation of the stride parameters permits the observation of
the relaxed outcomes targeted by all tests.


We can perform another, similar, experiment changing the $s$ (\verb+size_of_test+) and~$r$ (\verb+number_of_run+) parameters.
Notice that the respective default values of $s$ and~$r$ are
$1000$ and~$10$, as specified in the \ahref{x86-64.cfg}{\file{x86-64.cfg}}
configuration file.
We now try the following settings:
\begin{verbatim}
% sh run.sh -a 16 -s 10 -r 10000 > Y.01
% sh run.sh -a 16 -s 100 -r 1000 > Y.02
% sh run.sh -a 16 -s 1000 -r 100 > Y.03
% sh run.sh -a 16 -s 10000 -r 10 > Y.04
% sh run.sh -a 16 -s 100000 -r 1 > Y.05
\end{verbatim}
The additional \opt{-a 16}  command line option informs test executable
to use $16$ logical processors, hence running $8$ instances of
the ``\ltest{X}'' tests concurrently, as those tests all are two thread tests.
This technique of flooding the tested machine obviously
yields better ressource usage
and, according to our experience, favours outcome variability.

The following table summarises the results:
\begin{center}\let\handletest\xhandletest
\begin{tabular}{|l|l||r|r|r|r|r|}
\hline
& Y.01     & Y.02    & Y.03     & Y.04     & Y.05      \\
\hline
\hline
\handletest{X000} &  2.3k/800k& 602/800k& 465/800k & 551/800k & 297/800k  \\
\hline
\handletest{X001} &  2.9k/800k& 632/800k& 774/800k & 667/800k & 315/800k  \\
\hline
\handletest{X002} &  633/800k & 55/800k & 5/800k   & 7/800k   & 0/800k    \\
\hline
\handletest{X003} &  1.2k/800k& 182/800k& 152/800k & 390/800k & 57/800k   \\
\hline
\handletest{X004} &  2.4k/800k& 974/800k& 1.5k/800k& 2.4k/800k& 1.6k/800k \\
\hline
\handletest{X005} &  239/800k & 21/800k & 8/800k   & 0/800k   & 1/800k    \\
\hline
\handletest{X006} &  912/800k & 129/800k& 102/800k & 143/800k & 14/800k   \\
\hline
\end{tabular}


\end{center}
Again, we observe all targeted relaxed outcomes. In fact,
x86 relaxations are relatively easy to observe on our $16$
logical core machine.

Another test statistic of interest is
\emph{efficiency}, that is the number of targeted outcomes observed per
second:
\begin{center}\let\handletest\xhandletest
\begin{tabular}{|l||r|r|r|r|r|}
\hline
& Y.01& Y.02& Y.03& Y.04& Y.05 \\
\hline
\hline
\handletest{X000}& 285 & 2.2k& 6.6k& 9.2k& 4.2k \\
\hline
\handletest{X001}& 366 & 2.4k& 13k & 11k & 5.2k \\
\hline
\handletest{X002}& 78  & 212 & 71  & 140 &      \\
\hline
\handletest{X003}& 150 & 650 & 2.5k& 7.8k& 950  \\
\hline
\handletest{X004}& 288 & 3.7k& 25k & 59k & 32k  \\
\hline
\handletest{X005}& 28  & 72  & 114 &     & 17   \\
\hline
\handletest{X006}& 118 & 461 & 1.7k& 2.9k& 280  \\
\hline
\end{tabular}


\end{center}
As we can see, although the setting \opt{-s 10 -r 10000} yields the most
relaxed outcomes, it may not be considered as the most efficient.
Moreover, we see that tests \ltest{X002} and~\ltest{X005}
look more challenging than others.

Finally, it may be interesting to classify the ``\texttt{X}'' tests:
\begin{verbatim}
% mcycles @all | classify -arch X86
R
  X003 -> R+po+rfi-po : PodWW Wse Rfi PodRR Fre
  X006 -> R : PodWW Wse PodWR Fre
SB
  X000 -> SB+rfi-pos : Rfi PodRR Fre Rfi PodRR Fre
  X001 -> SB+rfi-po+po : Rfi PodRR Fre PodWR Fre
  X002 -> SB+mfence+rfi-po : MFencedWR Fre Rfi PodRR Fre
  X004 -> SB : PodWR Fre PodWR Fre
  X005 -> SB+mfence+po : MFencedWR Fre PodWR Fre
\end{verbatim}
One sees that two thread non-SC tests for x86 are basically of two kinds.




\section{Cross compiling, \label{affinity:experiment}affinity experiment}
In this section we describe how to produce the C~sources of tests
on a machine, while running the tests on another.
We also describe a sophisticated affinity experiment.

We assume a directory \ahref{tst-ppc}{\texttt{tst-ppc}}%
\ifhevea{} (\ahref{tst-ppc.tar}{archive})\fi,
that contains a series of litmus tests
and an index file~\ahref{tst-ppc/@all}{\file{@all}}.
Those tests where produced by the \prog{diycross}
tool. They illustrate variations of the
classical \ahref{tst-ppc/IRIW.litmus}{\ltest{IRIW}} test.
\ifhevea\begin{center}\img{IRIW}\end{center}\fi
More specifically, the \ltest{IRIW} variations are produced as follows
(see also Sec.~\ref{diycross:intro}):
\begin{verbatim}
% mkdir tst-ppc
% diycross -name IRIW -o tst-ppc Rfe PodRR,DpAddrdR,LwSyncdRR,EieiodRR,SyncdRR Fre Rfe PodRR,DpAddrdR,LwSyncdRR,EieiodRR,SyncdRR Fre
Generator produced 15 tests
\end{verbatim}


We target a Power7 machine described by the configuration file
\ahref{power7.cfg}{\file{power7.cfg}}:
\verbatiminput{power7.cfg}
One may notice the SMT (\emph{Simultaneaous Multi-Threading}) specification:
$4$-ways SMT (\verb+smt=4+), logical processors pertaining
to the same core being numbered in sequence (\verb+smt_mode = seq+) ---
that is, logical processors from the first core are $0$, $1$ ,$2$ and~$3$;
logical processors from the second core are $4$, $5$ ,$6$ and~$7$; etc.
The SMT specification is necessary to enable
custom affinity mode
(see Sec.~\ref{affinity:custom}).

One may also notice the specification of $0$ available logical processors
(\verb+avail=0+).
As affinity support is enabled (\verb+affinity=incr0+),
test executables will find themselves
the number of logical processors available on the target machine.


We compile tests to C-sources packed in archive \file{a.tar}
and upload the archive to the target power7 machine as follows:
\begin{verbatim}
% litmus -mach power7 -o a.tar tst-ppc/@all
% scp a.tar power7:
\end{verbatim}
Then, on \texttt{power7} we unpack the archive and produce executable tests
as follows:
\begin{verbatim}
power7% tar xmf a.tar
power7% make -j 8
gcc -D_GNU_SOURCE -Wall -std=gnu99 -O -m64 -pthread -O2 -c affinity.c
gcc -D_GNU_SOURCE -Wall -std=gnu99 -O -m64 -pthread -O2 -c outs.c
gcc -D_GNU_SOURCE -Wall -std=gnu99 -O -m64 -pthread -S IRIW+eieios.c
...
\end{verbatim}

As a starter, we can check the effect of available logical processor detection
and custom affinity control (option \opt{+ca})
by passing the command line option \opt{-v} to one test executable,
for instance
\texttt{IRIW.exe}:
\begin{verbatim}
power7% ./IRIW.exe -v +ca
./IRIW.exe -v +ca
IRIW: n=8, r=10, s=1000, st=1, +ca, p='0,1,2,3,4,5,6,7,8,9,10,11,12,13,14,15,16,17,18,19,20,21,22,23,24,25,26,27,28,29,30,31'
thread allocation: 
[23,22,3,2] {5,5,0,0}
[7,6,15,14] {1,1,3,3}
[11,10,5,4] {2,2,1,1}
[21,20,27,26] {5,5,6,6}
[9,8,25,24] {2,2,6,6}
[31,30,13,12] {7,7,3,3}
[19,18,29,28] {4,4,7,7}
[1,0,17,16] {0,0,4,4}
...
\end{verbatim}
We see that our machine \texttt{power7} features $32$ logical processors
numbered from $0$ to~$31$
(cf \verb+p=...+ above) and will thus run \verb+n=8+ concurrent
instances of the $4$~thread IRIW~test.
Additionally allocation of threads to logical processors is shown:
here, the four threads of the test are partitioned into two groups, which are
scheduled to run on different cores. For example, threads $0$ and~$1$ of
the first instance of the test will run on logical processors $23$ and~$22$
(core~$5$); while threads $2$ and~$3$ will run on logical
processors $3$ and~$2$ (core~$0$).

Our experiment consists in running all tests
with affinity increment (see Sec.~\ref{defi}) being from $0$
and then $1$ to~$8$ (option \opt{-i $i$}),
as well as in random and custom affinity mode
(options \opt{+ra} and~\opt{+ca}):
\begin{verbatim}
power7% for i in $(seq 0 8)
> do
> sh run.sh -i $i > Z.0$i
> done
power7% sh run.sh +ra > Z.0R
power7% sh run.sh +ca > Z.0C

\end{verbatim}
The following table summarises the results, with X meaning that the targeted
relaxed outcome is observed:
\begin{center}\def\tstdir{tst-ppc}\let\handletest\xhandletest
\begin{tabular}{|l||c|c|c|c|c|c|c|c|c|c|c|}
\hline
& Z.00& Z.01& Z.02& Z.03& Z.04& Z.05& Z.06& Z.07& Z.08& Z.0C& Z.0R \\
\hline
\hline
\handletest{IRIW}& X &     & X & X &     & X & X & X & X & X & X  \\
\hline
\handletest{IRIW+addr+po}& X &     & X & X &     &     &     &     & X  & X &      \\
\hline
\handletest{IRIW+addrs}&     &     & X &     &     &     &     &     &     & X & X   \\
\hline
\handletest{IRIW+eieio+addr}&     &     & X & X &     &     &     &     &     & X &      \\
\hline
\handletest{IRIW+eieio+po}&     &     & X & X &     &     &     &     &     & X &      \\
\hline
\handletest{IRIW+eieios}&     &     & X & X &     &     &     &     &     & X & X  \\
\hline
\handletest{IRIW+lwsync+addr}&     &     & X & X &     &     &     &     &     & X &      \\
\hline
\handletest{IRIW+lwsync+eieio}&     &     & X & X &     &     &     &     &     & X &      \\
\hline
\handletest{IRIW+lwsync+po}& X  &     & X & X &     &     &     & X &     & X &      \\
\hline
\handletest{IRIW+lwsyncs}&     &     & X &     &     &     &     &     &     &  X &      \\
\hline
\handletest{IRIW+sync+addr}&     &     & X &     &     &     &     &     &     & X &      \\
\hline
\handletest{IRIW+sync+eieio}&     &     & X &     &     &     &     &     &     & X &      \\
\hline
\handletest{IRIW+sync+lwsync}&     &     & X &     &     &     &     &     &     & X &      \\
\hline
\handletest{IRIW+sync+po}& X &     & X & X & X &     &     &     &     & X & X  \\
\hline
\handletest{IRIW+syncs}&     &     &     &     &     &     &     &     &     &     &      \\
\hline
\end{tabular}


\end{center}
On sees that all possible relaxed outcomes shows up with proper affinity
control. More precisely, setting the affinity increment to $2$ or resorting
to custom affinity result in the same effect:
the first two threads of the test run on one core, while the last two threads
of the test run on a different core.
As demonstrated by the experiment, this allocation of test threads to cores
suffices to favour relaxed outcomes for all tests except for
\ahref{tst-ppc/IRIW+syncs.litmus}{\ltest{IRIW+syncs}},
where the \texttt{sync} fences forbid them.


\section{Cross running, testing low-end devices}
Together \litmus{} options \ahrefloc{gcc}{\opt{-gcc}}
and~\ahrefloc{linkopt}{\opt{-linkopt}} permit using
a C~cross compiler. For instance, assume that \prog{litmus}
runs on machine~$A$ and that \opt{crossgcc}, a cross compiler for machine~$C$,
is available on machine~$B$. Then, the following sequence of
commands can be used to test machine~$C$:
\begin{verbatim}
A% litmus -gcc crossgcc -linkopt -static -o C-files.tar ...
A% scp C-files.tar B:

B% tar xf C-files.tar
B% make
B% tar cf /tmp/C-compiled.tar .
B% scp /tmp/C-compiled.tar C:

C% tar xf C-compiled.tar
C% sh run.sh
\end{verbatim}
Alternatively, using option \ahrefloc{crossrun}{\opt{-crossrun $C$}},
one can avoid copying the archive \verb+C-compiled.tar+ to machine~$C$:
\begin{verbatim}
A% litmus -crossrun C -gcc crossgcc -linkopt -static -o C-files.tar ...
A% scp C-files.tar B:

B% tar xf C-files.tar
B% make
B% sh run.sh
\end{verbatim}
More specifically, option \opt{-crossrun $C$} instructs the \file{run.sh}
script to upload executables individually to machine~$C$, just before running
them. Notice that executables are removed from~$C$ once run.

We illustrate the crossrun feature by testing \ltest{LB} variations on
an ARM-based Tegra3 ($4$ cores) tablet.
Test \ahref{tst-arm/LB.litmus}{\ltest{LB}} (load-buffering) exercises
the following ``causality'' loop:
\begin{center}\cycle{LB}\end{center}
That is, thread~0 reads the values stored to location~\texttt{x} by thread~1,
thread~1 reads the values stored to location~\texttt{y} by thread~0,
and both threads read ``before'' they write.

We shall consider tests with varying interpretations of ``before'':
the write may simply follow the read in program order
(\texttt{po} in test names),
may depend on the read (\texttt{data} and \texttt{addr}), or
they may be some fence in-betweeen
(\texttt{isb} and \texttt{dmb}).
We first generate tests \ahref{tst-arm}{\texttt{tst-arm}}%
\ifhevea{} (\ahref{tst-arm.tar}{archive}) \fi
with \prog{diycross}:
\begin{verbatim}
% mkdir tst-arm
% diycross -arch ARM -name LB -o tst-arm PodRW,DpDatadW,DpCtrldW,ISBdRW,DMBdRW Rfe PodRW,DpDatadW,DpCtrldW,ISBdRW,DMBdRW Rfe
Generator produced 15 tests
\end{verbatim}

We use the following, \afile{tegra3.cfg}, configuration file:
\verbatiminput{tegra3.cfg}
Notice the ``cross-compilation'' section:
the name of the gcc cross-compiler is \texttt{arm-linux-gnueabi-gcc},
while the adequate version of the target ARM variant
and static linking are specified.

We compile the tests from litmus source files to C~source files in
directory \texttt{TST} as follows:
\begin{verbatim}
% mkdir TST
% litmus -mach tegra3 -crossrun app_81@wifi-auth-188153:2222 tst-arm/@all -o TST
\end{verbatim}
The extra option \texttt{-crossrun app\_81@wifi-auth-188153:2222}
specifies the address to log onto the tablet by \texttt{ssh},
which is connected on a local WiFi network and runs a \texttt{ssh} daemon
that listens on port~$2222$.

We compile to executables and run them  as as follows:
\begin{verbatim}
% cd TST
% make
arm-linux-gnueabi-gcc -Wall -std=gnu99 -march=armv7-a -O2 -pthread -O2 -c outs.c
arm-linux-gnueabi-gcc -Wall -std=gnu99 -march=armv7-a -O2 -pthread -O2 -c utils.c
arm-linux-gnueabi-gcc -Wall -std=gnu99 -march=armv7-a -O2 -pthread -S LB.c
...
% sh run.sh > ARM-LB.log
\end{verbatim}
\ifhevea(Complete \ahref{ARM-LB.log}{run log}.) \fi
It is important to notice that the shell script \texttt{run.sh} runs
on the local machine, not on the remote tablet.
Each test executable is copied (by using \texttt{scp}) to the tablet, runs there
and is deleted (by using \texttt{ssh}), as can be seen with \texttt{sh}
``\texttt{-x}'' option:
\begin{verbatim}
% sh -x run.sh 2>&1 >ARM-LB.log | grep -e scp -e ssh
+ scp -P 2222 -q ./LB.exe app_81@wifi-auth-188153:
+ ssh -p 2222 -q -n app_81@wifi-auth-188153 ./LB.exe -q  && rm ./LB.exe
+ scp -P 2222 -q ./LB+data+po.exe app_81@wifi-auth-188153:
+ ssh -p 2222 -q -n app_81@wifi-auth-188153 ./LB+data+po.exe -q  && rm ./LB+data+po.exe
...
\end{verbatim}


Experiment results can be extracted from the log file quite easily,
by reading the ``Observation'' information from test output:
\begin{verbatim}
% grep Observation ARM-LB.log
Observation LB Sometimes 1395 1998605
Observation LB+data+po Sometimes 360 1999640
Observation LB+ctrl+po Sometimes 645 1999355
Observation LB+isb+po Sometimes 1676 1998324
Observation LB+dmb+po Sometimes 18 1999982
Observation LB+datas Never 0 2000000
Observation LB+ctrl+data Never 0 2000000
Observation LB+isb+data Sometimes 654 1999346
Observation LB+dmb+data Never 0 2000000
Observation LB+ctrls Never 0 2000000
Observation LB+isb+ctrl Sometimes 1143 1998857
Observation LB+dmb+ctrl Never 0 2000000
Observation LB+isbs Sometimes 2169 1997831
Observation LB+dmb+isb Sometimes 178 1999822
Observation LB+dmbs Never 0 2000000
\end{verbatim}
What is observed (\texttt{Sometimes}) or not (\texttt{Never}) is the occurence
of the non-SC behaviour of tests. All tests have the same structure
and the observation of the non-SC behaviour can be interpreted as
some read not being ``before'' the write by the same thread.
This situation occurs for plain program order (plain test \ltest{LB} and
\texttt{po} variations) and for the \texttt{isb} fence.

\ifhevea
The following graph summarises the observations and illustrates
that data dependencies, control dependencies and the \texttt{dmb} barrier
apparently suffice to restore SC in the case of the LB family.
\begin{center}\img{LB-kinds}\end{center}
In the graph above, a red node means an observation of the non-SC behaviour.
\fi

\endinput
%%Not yet....

\section{Finding\label{example:invalid} and showing invalid executions}
\ifhevea
We now describe a complete experiment that will use some of
the additional tools we distribute.
The experiment aims at comparing ARM machines with the uniproc model.
The uniproc model is a very relaxed memory model that only enforces
single-thread correctness --- See Sec.~\ref{defuniproc}.
Single thread correctness can be defined as accesses to same
location by the same thread do not contradict communication candidate
relaxations --- See Sec.~\ref{communication:cr}.

\subsection*{Test generation}
We first produce a few simple tests that access one memory location
only. We use the \prog{diy} test generator with configuration
file~\ahref{tst-co/CO.conf}{\texttt{CO.conf}}:
\verbatiminput{tst-co/src/CO.conf}
The first section above describes generated cycles: the vocabulary
of candidate relaxations (we also consider DMB barriers),
the size of cycles, the (maximal) number of threads, and a strong limitation
on the number of consecutive internal candidate relaxations
(the setting \opt{ins 2} by rejecting 2 or more consecutive internal candidate
relaxations in effect forbids sequences of internal candidate
relaxations).
We also specify \prog{diy} \opt{uni} mode (See Sec.~\ref{uni:def})
that will allow test generation from one-location cycles,
and replace final conditions by the observation of test outcomes
(\opt{cond observe}).
Here we go:
\begin{verbatim}
% diy -conf CO.conf
Generator produced 31 tests
% ls
2+2W+dmb+pos.litmus  MP+dmbs.litmus     SB+dmb+pos.litmus  W+RR+dmb.litmus
2+2W+dmbs.litmus     MP+pos+dmb.litmus  SB+dmbs.litmus     W+RR.litmus
2+2W+poss.litmus     MP+poss.litmus     SB+poss.litmus     W+RW+dmb.litmus
@all                 R+dmb+pos.litmus   S+dmb+pos.litmus   W+RW.litmus
CO.conf              R+dmbs.litmus      S+dmbs.litmus      WW+dmb.litmus
LB+dmb+pos.litmus    R+pos+dmb.litmus   S+pos+dmb.litmus   WW.litmus
LB+dmbs.litmus       R+poss.litmus      S+poss.litmus
LB+poss.litmus       RW+dmb.litmus      WR+dmb.litmus
MP+dmb+pos.litmus    RW.litmus          WR.litmus
\end{verbatim}
\def\csname images-MPPOSS-img-MP+pos+dmb\endcsname{
\begin{verbatim}
Executions for behaviour: "1:R0=0 ; 1:R1=0 ; x=1"
\end{verbatim}
 \imgsrc{MPPOSS-img/MP+pos+dmb+cond00-00.png}\begin{verbatim}
Executions for behaviour: "1:R0=1 ; 1:R1=0 ; x=1"
\end{verbatim}
 \imgsrc{MPPOSS-img/MP+pos+dmb+cond01-00.png}\begin{verbatim}
Executions for behaviour: "1:R0=2 ; 1:R1=0 ; x=1"
\end{verbatim}
 \imgsrc{MPPOSS-img/MP+pos+dmb+cond02-00.png}\begin{verbatim}
Executions for behaviour: "1:R0=0 ; 1:R1=1 ; x=1"
\end{verbatim}
 \imgsrc{MPPOSS-img/MP+pos+dmb+cond03-00.png}\begin{verbatim}
Executions for behaviour: "1:R0=1 ; 1:R1=1 ; x=1"
\end{verbatim}
 \imgsrc{MPPOSS-img/MP+pos+dmb+cond04-00.png}\begin{verbatim}
Executions for behaviour: "1:R0=2 ; 1:R1=1 ; x=1"
\end{verbatim}
 \imgsrc{MPPOSS-img/MP+pos+dmb+cond05-00.png}\begin{verbatim}
Executions for behaviour: "1:R0=0 ; 1:R1=2 ; x=1"
\end{verbatim}
 \imgsrc{MPPOSS-img/MP+pos+dmb+cond06-00.png}\begin{verbatim}
Executions for behaviour: "1:R0=1 ; 1:R1=2 ; x=1"
\end{verbatim}
 \imgsrc{MPPOSS-img/MP+pos+dmb+cond07-00.png}\begin{verbatim}
Executions for behaviour: "1:R0=2 ; 1:R1=2 ; x=1"
\end{verbatim}
 \imgsrc{MPPOSS-img/MP+pos+dmb+cond08-00.png}\begin{verbatim}
Executions for behaviour: "1:R0=1 ; 1:R1=0 ; x=2"
\end{verbatim}
 \imgsrc{MPPOSS-img/MP+pos+dmb+cond09-00.png}\begin{verbatim}
Executions for behaviour: "1:R0=2 ; 1:R1=0 ; x=2"
\end{verbatim}
 \imgsrc{MPPOSS-img/MP+pos+dmb+cond10-00.png}\begin{verbatim}
Executions for behaviour: "1:R0=2 ; 1:R1=1 ; x=2"
\end{verbatim}
 \imgsrc{MPPOSS-img/MP+pos+dmb+cond11-00.png}}%
\def\csname images-MPPOSS-img-MP+poss\endcsname{
\begin{verbatim}
Executions for behaviour: "1:R0=0 ; 1:R1=0 ; x=1"
\end{verbatim}
 \imgsrc{MPPOSS-img/MP+poss+cond00-00.png}\begin{verbatim}
Executions for behaviour: "1:R0=1 ; 1:R1=0 ; x=1"
\end{verbatim}
 \imgsrc{MPPOSS-img/MP+poss+cond01-00.png}\begin{verbatim}
Executions for behaviour: "1:R0=2 ; 1:R1=0 ; x=1"
\end{verbatim}
 \imgsrc{MPPOSS-img/MP+poss+cond02-00.png}\begin{verbatim}
Executions for behaviour: "1:R0=0 ; 1:R1=1 ; x=1"
\end{verbatim}
 \imgsrc{MPPOSS-img/MP+poss+cond03-00.png}\begin{verbatim}
Executions for behaviour: "1:R0=1 ; 1:R1=1 ; x=1"
\end{verbatim}
 \imgsrc{MPPOSS-img/MP+poss+cond04-00.png}\begin{verbatim}
Executions for behaviour: "1:R0=2 ; 1:R1=1 ; x=1"
\end{verbatim}
 \imgsrc{MPPOSS-img/MP+poss+cond05-00.png}\begin{verbatim}
Executions for behaviour: "1:R0=0 ; 1:R1=2 ; x=1"
\end{verbatim}
 \imgsrc{MPPOSS-img/MP+poss+cond06-00.png}\begin{verbatim}
Executions for behaviour: "1:R0=1 ; 1:R1=2 ; x=1"
\end{verbatim}
 \imgsrc{MPPOSS-img/MP+poss+cond07-00.png}\begin{verbatim}
Executions for behaviour: "1:R0=2 ; 1:R1=2 ; x=1"
\end{verbatim}
 \imgsrc{MPPOSS-img/MP+poss+cond08-00.png}\begin{verbatim}
Executions for behaviour: "1:R0=1 ; 1:R1=0 ; x=2"
\end{verbatim}
 \imgsrc{MPPOSS-img/MP+poss+cond09-00.png}\begin{verbatim}
Executions for behaviour: "1:R0=2 ; 1:R1=0 ; x=2"
\end{verbatim}
 \imgsrc{MPPOSS-img/MP+poss+cond10-00.png}\begin{verbatim}
Executions for behaviour: "1:R0=2 ; 1:R1=1 ; x=2"
\end{verbatim}
 \imgsrc{MPPOSS-img/MP+poss+cond11-00.png}}%
\def\csname images-MPPOSS-img-MP+dmbs\endcsname{
\begin{verbatim}
Executions for behaviour: "1:R0=0 ; 1:R1=0 ; x=1"
\end{verbatim}
 \imgsrc{MPPOSS-img/MP+dmbs+cond00-00.png}\begin{verbatim}
Executions for behaviour: "1:R0=1 ; 1:R1=0 ; x=1"
\end{verbatim}
 \imgsrc{MPPOSS-img/MP+dmbs+cond01-00.png}\begin{verbatim}
Executions for behaviour: "1:R0=2 ; 1:R1=0 ; x=1"
\end{verbatim}
 \imgsrc{MPPOSS-img/MP+dmbs+cond02-00.png}\begin{verbatim}
Executions for behaviour: "1:R0=0 ; 1:R1=1 ; x=1"
\end{verbatim}
 \imgsrc{MPPOSS-img/MP+dmbs+cond03-00.png}\begin{verbatim}
Executions for behaviour: "1:R0=1 ; 1:R1=1 ; x=1"
\end{verbatim}
 \imgsrc{MPPOSS-img/MP+dmbs+cond04-00.png}\begin{verbatim}
Executions for behaviour: "1:R0=2 ; 1:R1=1 ; x=1"
\end{verbatim}
 \imgsrc{MPPOSS-img/MP+dmbs+cond05-00.png}\begin{verbatim}
Executions for behaviour: "1:R0=0 ; 1:R1=2 ; x=1"
\end{verbatim}
 \imgsrc{MPPOSS-img/MP+dmbs+cond06-00.png}\begin{verbatim}
Executions for behaviour: "1:R0=1 ; 1:R1=2 ; x=1"
\end{verbatim}
 \imgsrc{MPPOSS-img/MP+dmbs+cond07-00.png}\begin{verbatim}
Executions for behaviour: "1:R0=2 ; 1:R1=2 ; x=1"
\end{verbatim}
 \imgsrc{MPPOSS-img/MP+dmbs+cond08-00.png}\begin{verbatim}
Executions for behaviour: "1:R0=1 ; 1:R1=0 ; x=2"
\end{verbatim}
 \imgsrc{MPPOSS-img/MP+dmbs+cond09-00.png}\begin{verbatim}
Executions for behaviour: "1:R0=2 ; 1:R1=0 ; x=2"
\end{verbatim}
 \imgsrc{MPPOSS-img/MP+dmbs+cond10-00.png}\begin{verbatim}
Executions for behaviour: "1:R0=2 ; 1:R1=1 ; x=2"
\end{verbatim}
 \imgsrc{MPPOSS-img/MP+dmbs+cond11-00.png}}%
\def\csname images-MPPOSS-img-MP+dmb+pos\endcsname{
\begin{verbatim}
Executions for behaviour: "1:R0=0 ; 1:R1=0 ; x=1"
\end{verbatim}
 \imgsrc{MPPOSS-img/MP+dmb+pos+cond00-00.png}\begin{verbatim}
Executions for behaviour: "1:R0=1 ; 1:R1=0 ; x=1"
\end{verbatim}
 \imgsrc{MPPOSS-img/MP+dmb+pos+cond01-00.png}\begin{verbatim}
Executions for behaviour: "1:R0=2 ; 1:R1=0 ; x=1"
\end{verbatim}
 \imgsrc{MPPOSS-img/MP+dmb+pos+cond02-00.png}\begin{verbatim}
Executions for behaviour: "1:R0=0 ; 1:R1=1 ; x=1"
\end{verbatim}
 \imgsrc{MPPOSS-img/MP+dmb+pos+cond03-00.png}\begin{verbatim}
Executions for behaviour: "1:R0=1 ; 1:R1=1 ; x=1"
\end{verbatim}
 \imgsrc{MPPOSS-img/MP+dmb+pos+cond04-00.png}\begin{verbatim}
Executions for behaviour: "1:R0=2 ; 1:R1=1 ; x=1"
\end{verbatim}
 \imgsrc{MPPOSS-img/MP+dmb+pos+cond05-00.png}\begin{verbatim}
Executions for behaviour: "1:R0=0 ; 1:R1=2 ; x=1"
\end{verbatim}
 \imgsrc{MPPOSS-img/MP+dmb+pos+cond06-00.png}\begin{verbatim}
Executions for behaviour: "1:R0=1 ; 1:R1=2 ; x=1"
\end{verbatim}
 \imgsrc{MPPOSS-img/MP+dmb+pos+cond07-00.png}\begin{verbatim}
Executions for behaviour: "1:R0=2 ; 1:R1=2 ; x=1"
\end{verbatim}
 \imgsrc{MPPOSS-img/MP+dmb+pos+cond08-00.png}\begin{verbatim}
Executions for behaviour: "1:R0=1 ; 1:R1=0 ; x=2"
\end{verbatim}
 \imgsrc{MPPOSS-img/MP+dmb+pos+cond09-00.png}\begin{verbatim}
Executions for behaviour: "1:R0=2 ; 1:R1=0 ; x=2"
\end{verbatim}
 \imgsrc{MPPOSS-img/MP+dmb+pos+cond10-00.png}\begin{verbatim}
Executions for behaviour: "1:R0=2 ; 1:R1=1 ; x=2"
\end{verbatim}
 \imgsrc{MPPOSS-img/MP+dmb+pos+cond11-00.png}}%
%
\def\csname RW@base\endcsname{BASIC-RW}%
\def\csname RW@img\endcsname{\csname images-BASIC-img-RW\endcsname}%
\def\csname RW@src\endcsname{BASIC-src/RW.litmus}%
\gentest{RW}%
\def\csname WR@base\endcsname{BASIC-WR}%
\def\csname WR@img\endcsname{\csname images-BASIC-img-WR\endcsname}%
\def\csname WR@src\endcsname{BASIC-src/WR.litmus}%
\gentest{WR}%
\def\csname WW@base\endcsname{BASIC-WW}%
\def\csname WW@img\endcsname{\csname images-BASIC-img-WW\endcsname}%
\def\csname WW@src\endcsname{BASIC-src/WW.litmus}%
\gentest{WW}%
\def\csname W+RR@base\endcsname{BASIC-W+RR}%
\def\csname W+RR@img\endcsname{\csname images-BASIC-img-W+RR\endcsname}%
\def\csname W+RR@src\endcsname{BASIC-src/W+RR.litmus}%
\gentest{W+RR}%
\def\csname W+RW@base\endcsname{BASIC-W+RW}%
\def\csname W+RW@img\endcsname{\csname images-BASIC-img-W+RW\endcsname}%
\def\csname W+RW@src\endcsname{BASIC-src/W+RW.litmus}%
\gentest{W+RW}%
\def\BASIC{%
\ahref{\base{RW}.html}{\textsf{RW}}%
, \ahref{\base{WR}.html}{\textsf{WR}}%
, \ahref{\base{WW}.html}{\textsf{WW}}%
, \ahref{\base{W+RR}.html}{\textsf{W+RR}}%
{} and \ahref{\base{W+RW}.html}{\textsf{W+RW}}%
}%
\def\applyBASIC#1{%
#1{RW}%
#1{WR}%
#1{WW}%
#1{W+RR}%
#1{W+RW}%
}%
\newcommand{\basic}[1]{\ahref{BASIC-#1.html}{\ltest{#1}}}%
We get 31~tests, 
available in directory \ahref{tst-co}{\texttt{tst-co}}%
\ifhevea{} (\ahref{tst-co.tar}{archive})\fi.
Amongst those $31$~tests, are five tests that may exhibit
the five ``basic'' uniproc violations.
Those five basic uniproc violations are three direct contraditions of
program order and communication candidate relaxations Ws,Rf and~Fr ---
which we show
as executions of the tests \basic{WW}, \basic{RW} and~\basic{WR}:
\begin{center}
\img{WW}\qquad\qquad
\img{RW}\qquad\qquad
\img{WR}
\end{center}
plus two contradictions of
program order and communication candidate relaxation
sequences Ws;Rf and~Fr;Rf,
which we show
as executions of the tests \basic{W+RW} and~\basic{W+RR}:
\begin{center}
\img{W+RW}\qquad
\img{W+RR}
\end{center}

Namely, due to the transitivity of Ws and to the definition of Fr
(that implies $\textrm{Fr;Ws} \subseteq \textrm{Fr}$) all sequences of
communications are covered by the above listed five cases.

\def\csname images-MPPOSS-img-MP+pos+dmb\endcsname{
\begin{verbatim}
Executions for behaviour: "1:R0=0 ; 1:R1=0 ; x=1"
\end{verbatim}
 \imgsrc{MPPOSS-img/MP+pos+dmb+cond00-00.png}\begin{verbatim}
Executions for behaviour: "1:R0=1 ; 1:R1=0 ; x=1"
\end{verbatim}
 \imgsrc{MPPOSS-img/MP+pos+dmb+cond01-00.png}\begin{verbatim}
Executions for behaviour: "1:R0=2 ; 1:R1=0 ; x=1"
\end{verbatim}
 \imgsrc{MPPOSS-img/MP+pos+dmb+cond02-00.png}\begin{verbatim}
Executions for behaviour: "1:R0=0 ; 1:R1=1 ; x=1"
\end{verbatim}
 \imgsrc{MPPOSS-img/MP+pos+dmb+cond03-00.png}\begin{verbatim}
Executions for behaviour: "1:R0=1 ; 1:R1=1 ; x=1"
\end{verbatim}
 \imgsrc{MPPOSS-img/MP+pos+dmb+cond04-00.png}\begin{verbatim}
Executions for behaviour: "1:R0=2 ; 1:R1=1 ; x=1"
\end{verbatim}
 \imgsrc{MPPOSS-img/MP+pos+dmb+cond05-00.png}\begin{verbatim}
Executions for behaviour: "1:R0=0 ; 1:R1=2 ; x=1"
\end{verbatim}
 \imgsrc{MPPOSS-img/MP+pos+dmb+cond06-00.png}\begin{verbatim}
Executions for behaviour: "1:R0=1 ; 1:R1=2 ; x=1"
\end{verbatim}
 \imgsrc{MPPOSS-img/MP+pos+dmb+cond07-00.png}\begin{verbatim}
Executions for behaviour: "1:R0=2 ; 1:R1=2 ; x=1"
\end{verbatim}
 \imgsrc{MPPOSS-img/MP+pos+dmb+cond08-00.png}\begin{verbatim}
Executions for behaviour: "1:R0=1 ; 1:R1=0 ; x=2"
\end{verbatim}
 \imgsrc{MPPOSS-img/MP+pos+dmb+cond09-00.png}\begin{verbatim}
Executions for behaviour: "1:R0=2 ; 1:R1=0 ; x=2"
\end{verbatim}
 \imgsrc{MPPOSS-img/MP+pos+dmb+cond10-00.png}\begin{verbatim}
Executions for behaviour: "1:R0=2 ; 1:R1=1 ; x=2"
\end{verbatim}
 \imgsrc{MPPOSS-img/MP+pos+dmb+cond11-00.png}}%
\def\csname images-MPPOSS-img-MP+poss\endcsname{
\begin{verbatim}
Executions for behaviour: "1:R0=0 ; 1:R1=0 ; x=1"
\end{verbatim}
 \imgsrc{MPPOSS-img/MP+poss+cond00-00.png}\begin{verbatim}
Executions for behaviour: "1:R0=1 ; 1:R1=0 ; x=1"
\end{verbatim}
 \imgsrc{MPPOSS-img/MP+poss+cond01-00.png}\begin{verbatim}
Executions for behaviour: "1:R0=2 ; 1:R1=0 ; x=1"
\end{verbatim}
 \imgsrc{MPPOSS-img/MP+poss+cond02-00.png}\begin{verbatim}
Executions for behaviour: "1:R0=0 ; 1:R1=1 ; x=1"
\end{verbatim}
 \imgsrc{MPPOSS-img/MP+poss+cond03-00.png}\begin{verbatim}
Executions for behaviour: "1:R0=1 ; 1:R1=1 ; x=1"
\end{verbatim}
 \imgsrc{MPPOSS-img/MP+poss+cond04-00.png}\begin{verbatim}
Executions for behaviour: "1:R0=2 ; 1:R1=1 ; x=1"
\end{verbatim}
 \imgsrc{MPPOSS-img/MP+poss+cond05-00.png}\begin{verbatim}
Executions for behaviour: "1:R0=0 ; 1:R1=2 ; x=1"
\end{verbatim}
 \imgsrc{MPPOSS-img/MP+poss+cond06-00.png}\begin{verbatim}
Executions for behaviour: "1:R0=1 ; 1:R1=2 ; x=1"
\end{verbatim}
 \imgsrc{MPPOSS-img/MP+poss+cond07-00.png}\begin{verbatim}
Executions for behaviour: "1:R0=2 ; 1:R1=2 ; x=1"
\end{verbatim}
 \imgsrc{MPPOSS-img/MP+poss+cond08-00.png}\begin{verbatim}
Executions for behaviour: "1:R0=1 ; 1:R1=0 ; x=2"
\end{verbatim}
 \imgsrc{MPPOSS-img/MP+poss+cond09-00.png}\begin{verbatim}
Executions for behaviour: "1:R0=2 ; 1:R1=0 ; x=2"
\end{verbatim}
 \imgsrc{MPPOSS-img/MP+poss+cond10-00.png}\begin{verbatim}
Executions for behaviour: "1:R0=2 ; 1:R1=1 ; x=2"
\end{verbatim}
 \imgsrc{MPPOSS-img/MP+poss+cond11-00.png}}%
\def\csname images-MPPOSS-img-MP+dmbs\endcsname{
\begin{verbatim}
Executions for behaviour: "1:R0=0 ; 1:R1=0 ; x=1"
\end{verbatim}
 \imgsrc{MPPOSS-img/MP+dmbs+cond00-00.png}\begin{verbatim}
Executions for behaviour: "1:R0=1 ; 1:R1=0 ; x=1"
\end{verbatim}
 \imgsrc{MPPOSS-img/MP+dmbs+cond01-00.png}\begin{verbatim}
Executions for behaviour: "1:R0=2 ; 1:R1=0 ; x=1"
\end{verbatim}
 \imgsrc{MPPOSS-img/MP+dmbs+cond02-00.png}\begin{verbatim}
Executions for behaviour: "1:R0=0 ; 1:R1=1 ; x=1"
\end{verbatim}
 \imgsrc{MPPOSS-img/MP+dmbs+cond03-00.png}\begin{verbatim}
Executions for behaviour: "1:R0=1 ; 1:R1=1 ; x=1"
\end{verbatim}
 \imgsrc{MPPOSS-img/MP+dmbs+cond04-00.png}\begin{verbatim}
Executions for behaviour: "1:R0=2 ; 1:R1=1 ; x=1"
\end{verbatim}
 \imgsrc{MPPOSS-img/MP+dmbs+cond05-00.png}\begin{verbatim}
Executions for behaviour: "1:R0=0 ; 1:R1=2 ; x=1"
\end{verbatim}
 \imgsrc{MPPOSS-img/MP+dmbs+cond06-00.png}\begin{verbatim}
Executions for behaviour: "1:R0=1 ; 1:R1=2 ; x=1"
\end{verbatim}
 \imgsrc{MPPOSS-img/MP+dmbs+cond07-00.png}\begin{verbatim}
Executions for behaviour: "1:R0=2 ; 1:R1=2 ; x=1"
\end{verbatim}
 \imgsrc{MPPOSS-img/MP+dmbs+cond08-00.png}\begin{verbatim}
Executions for behaviour: "1:R0=1 ; 1:R1=0 ; x=2"
\end{verbatim}
 \imgsrc{MPPOSS-img/MP+dmbs+cond09-00.png}\begin{verbatim}
Executions for behaviour: "1:R0=2 ; 1:R1=0 ; x=2"
\end{verbatim}
 \imgsrc{MPPOSS-img/MP+dmbs+cond10-00.png}\begin{verbatim}
Executions for behaviour: "1:R0=2 ; 1:R1=1 ; x=2"
\end{verbatim}
 \imgsrc{MPPOSS-img/MP+dmbs+cond11-00.png}}%
\def\csname images-MPPOSS-img-MP+dmb+pos\endcsname{
\begin{verbatim}
Executions for behaviour: "1:R0=0 ; 1:R1=0 ; x=1"
\end{verbatim}
 \imgsrc{MPPOSS-img/MP+dmb+pos+cond00-00.png}\begin{verbatim}
Executions for behaviour: "1:R0=1 ; 1:R1=0 ; x=1"
\end{verbatim}
 \imgsrc{MPPOSS-img/MP+dmb+pos+cond01-00.png}\begin{verbatim}
Executions for behaviour: "1:R0=2 ; 1:R1=0 ; x=1"
\end{verbatim}
 \imgsrc{MPPOSS-img/MP+dmb+pos+cond02-00.png}\begin{verbatim}
Executions for behaviour: "1:R0=0 ; 1:R1=1 ; x=1"
\end{verbatim}
 \imgsrc{MPPOSS-img/MP+dmb+pos+cond03-00.png}\begin{verbatim}
Executions for behaviour: "1:R0=1 ; 1:R1=1 ; x=1"
\end{verbatim}
 \imgsrc{MPPOSS-img/MP+dmb+pos+cond04-00.png}\begin{verbatim}
Executions for behaviour: "1:R0=2 ; 1:R1=1 ; x=1"
\end{verbatim}
 \imgsrc{MPPOSS-img/MP+dmb+pos+cond05-00.png}\begin{verbatim}
Executions for behaviour: "1:R0=0 ; 1:R1=2 ; x=1"
\end{verbatim}
 \imgsrc{MPPOSS-img/MP+dmb+pos+cond06-00.png}\begin{verbatim}
Executions for behaviour: "1:R0=1 ; 1:R1=2 ; x=1"
\end{verbatim}
 \imgsrc{MPPOSS-img/MP+dmb+pos+cond07-00.png}\begin{verbatim}
Executions for behaviour: "1:R0=2 ; 1:R1=2 ; x=1"
\end{verbatim}
 \imgsrc{MPPOSS-img/MP+dmb+pos+cond08-00.png}\begin{verbatim}
Executions for behaviour: "1:R0=1 ; 1:R1=0 ; x=2"
\end{verbatim}
 \imgsrc{MPPOSS-img/MP+dmb+pos+cond09-00.png}\begin{verbatim}
Executions for behaviour: "1:R0=2 ; 1:R1=0 ; x=2"
\end{verbatim}
 \imgsrc{MPPOSS-img/MP+dmb+pos+cond10-00.png}\begin{verbatim}
Executions for behaviour: "1:R0=2 ; 1:R1=1 ; x=2"
\end{verbatim}
 \imgsrc{MPPOSS-img/MP+dmb+pos+cond11-00.png}}%
%
\def\csname MP+dmb+pos@base\endcsname{MPPOSS-MP+dmb+pos}%
\def\csname MP+dmb+pos@img\endcsname{\csname images-MPPOSS-img-MP+dmb+pos\endcsname}%
\def\csname MP+dmb+pos@src\endcsname{MPPOSS-src/MP+dmb+pos.litmus}%
\gentest{MP+dmb+pos}%
\def\csname MP+dmbs@base\endcsname{MPPOSS-MP+dmbs}%
\def\csname MP+dmbs@img\endcsname{\csname images-MPPOSS-img-MP+dmbs\endcsname}%
\def\csname MP+dmbs@src\endcsname{MPPOSS-src/MP+dmbs.litmus}%
\gentest{MP+dmbs}%
\def\csname MP+pos+dmb@base\endcsname{MPPOSS-MP+pos+dmb}%
\def\csname MP+pos+dmb@img\endcsname{\csname images-MPPOSS-img-MP+pos+dmb\endcsname}%
\def\csname MP+pos+dmb@src\endcsname{MPPOSS-src/MP+pos+dmb.litmus}%
\gentest{MP+pos+dmb}%
\def\csname MP+poss@base\endcsname{MPPOSS-MP+poss}%
\def\csname MP+poss@img\endcsname{\csname images-MPPOSS-img-MP+poss\endcsname}%
\def\csname MP+poss@src\endcsname{MPPOSS-src/MP+poss.litmus}%
\gentest{MP+poss}%
\def\MPPOSS{%
\ahref{\base{MP+dmb+pos}.html}{\textsf{MP+dmb+pos}}%
, \ahref{\base{MP+dmbs}.html}{\textsf{MP+dmbs}}%
, \ahref{\base{MP+pos+dmb}.html}{\textsf{MP+pos+dmb}}%
{} and \ahref{\base{MP+poss}.html}{\textsf{MP+poss}}%
}%
\def\applyMPPOSS#1{%
#1{MP+dmb+pos}%
#1{MP+dmbs}%
#1{MP+pos+dmb}%
#1{MP+poss}%
}%
\newcommand{\mpposs}[1]{\ahref{MPPOSS-#1.html}{#1}}%
However, having more tests than the five basic ones is relevant to
hardware testing, as the \textsc{uniproc} check will be exercised in
more contexts. For instance the test \mpposs{MP+poss}
may reveal $12$~different violations of \textsc{uniproc}.
We shall also test similar violations
in the presence of one or two \texttt{dmb} fences
(tests \mpposs{MP+dmb+pos}, \mpposs{MP+pos+dmb} and~\mpposs{MP+dmbs}).

\subsection{Running tests on hardware}

\subsubsection{Trimslice computer (a machine that runs linux)}
We first run our test set on a
\ahref{http://utilite-computer.com/web/trim-slice}{Trimslice computer},
which is powered by a NVIDIA cortex-A9 based Tegra2 chipset.
As the Trimslice machine runs some Linux distribution, we proceed by ordinary
cross-compilation (\emph{i.e.} we compile litmus tests into C sources
on our local machine and compile C files on the remote machine):
\begin{verbatim}
% litmus -mach trimslice -mem direct -st 1 -o /tmp/A.tar tst-co/@all
% scp /tmp/A.tar trimslice:/tmp/A.tar
\end{verbatim}
We use the \afile{trimslice.cfg} (\prog{litmus}) configuration file present
in the \prog{litmus} distribution, additionnally specifying
\ahrefloc{defmemorymode}{memory direct mode}
and a \ahrefloc{defstride}{stride value} of~\opt{1}.
We then copy the \texttt{A.tar} archive to our Trimslice machine.

We unpack and compile the C~sources on the remote Trimslice computer:
\begin{verbatim}
$ mkdir TST
$ cd TST
$ tar xf /tmp/A.tar
$ make -j 4 all
gcc -D_GNU_SOURCE -Wall -std=gnu99 -mcpu=cortex-a9 -marm -O2 -pthread -O2 -c affinity.c
gcc -D_GNU_SOURCE -Wall -std=gnu99 -mcpu=cortex-a9 -marm -O2 -pthread -O2 -c outs.c
gcc -D_GNU_SOURCE -Wall -std=gnu99 -mcpu=cortex-a9 -marm -O2 -pthread -O2 -c utils.c
gcc -D_GNU_SOURCE -Wall -std=gnu99 -mcpu=cortex-a9 -marm -O2 -pthread -S R+dmbs.c
...
gcc -D_GNU_SOURCE -Wall -std=gnu99 -mcpu=cortex-a9 -marm -O2 -pthread  -o W+RW.exe affinity.o outs.o utils.o W+RW.s
gcc -D_GNU_SOURCE -Wall -std=gnu99 -mcpu=cortex-a9 -marm -O2 -pthread  -o W+RR.exe affinity.o outs.o utils.o W+RR.s
\end{verbatim}
We are now ready for running the tests, we perform $10$~runs of the tests,
with varying strides, using a shell script \afile{trimslice.sh}.
\verbatiminput{trimslice.sh}
Notice that we also use the
\ahrefloc{affinity:runopt}{affinity setting \opt{-i 1}}.
We do so in order to accelerate the tests, as very little
can ne expected from a dual-core system as regards thread placement on cores.
We run the tests:
\begin{verbatim}
$ sh ./trimslice.sh
\end{verbatim}
After a bit less then $20$~minutes we get ten files
\texttt{CO.00} to~\texttt{CO.09} that we transfer back to our local machine,
into some sub-directory \texttt{trimslice}.

\subsubsection{\label{driverc:example}APQ8060 (a development board that runs Android)}
We then run our test set on a development board
powered by a Qualcomm \ahref{http://en.wikipedia.org/wiki/Snapdragon_(system_on_chip)#Snapdragon_S3}{APQ8060 system-on-chip} (dual-core Scorpion).
This board runs Android and is connected to our local computer
by the \emph{Android Debug Bridge} (adb).
We perform complete cross-compilation, \emph{i.e.} we shall compile
both litmus sources into C~sources and C~sources into executables
on our local machine. We first run \prog{litmus} as follows:
\begin{verbatim}
% mkdir -p SRC/DRAGON
% litmus -mach dragon -mem direct -st 1 -o SRC/DRAGON -driver C tst-co/@all
\end{verbatim}
We use the \afile{dragon.cfg} (\prog{litmus}) configuration file present
in the \prog{litmus} distribution:
\verbatiminput{dragon.cfg}
The configuration file defines the number of available cores
(\verb+avail = 2+),
gives some defaults for standard tests parameters,
specifies a few \ahrefloc{affinity:control}{affinity settings}
(\verb+affinity = incr0+ and \verb+force_affinity = true+), and then
define the C~cross-compiler and its options.
Notice that the C~sources are dumped into the pre-existing
directory (\opt{-o SRC/DRAGON}) and that we build a single executable
(\opt{-driver C}).
We then compile C~sources, on the local machine.
\begin{verbatim}
% cd SRC/DRAGON
% make -j 8
arm-linux-gnueabi-gcc -D_GNU_SOURCE -Wall -std=gnu99 -march=armv7-a -mthumb -O2 -pthread -O2 -c affinity.c
arm-linux-gnueabi-gcc -D_GNU_SOURCE -Wall -std=gnu99 -march=armv7-a -mthumb -O2 -pthread -O2 -c outs.c
arm-linux-gnueabi-gcc -D_GNU_SOURCE -Wall -std=gnu99 -march=armv7-a -mthumb -O2 -pthread -O2 -c utils.c
arm-linux-gnueabi-gcc -DASS -D_GNU_SOURCE -Wall -std=gnu99 -march=armv7-a -mthumb -O2 -pthread -S R+dmbs.c
...
arm-linux-gnueabi-gcc  -D_GNU_SOURCE -Wall -std=gnu99 -march=armv7-a -mthumb -O2 -pthread -static -o run.exe affinity.o outs.o utils.o R+dmbs.o MP+dmbs.o R+dmb+pos.o MP+dmb+pos.o WW+dmb.o 2+2W+dmbs.o S+dmbs.o 2+2W+dmb+pos.o S+dmb+pos.o WR+dmb.o SB+dmbs.o SB+dmb+pos.o R+pos+dmb.o RW+dmb.o LB+dmbs.o S+pos+dmb.o LB+dmb+pos.o W+RW+dmb.o MP+pos+dmb.o W+RR+dmb.o R+poss.o MP+poss.o WW.o 2+2W+poss.o S+poss.o WR.o SB+poss.o RW.o LB+poss.o W+RW.o W+RR.o run.o
rm W+R
...
\end{verbatim}
Thanks to the setting \opt{-driver C} the above compilation produced one
executable file ``\file{run.exe}'', which we upload onto the Android device:
\begin{verbatim}
% adb push ./run.exe /data/tmp
6104 KB/s (892848 bytes in 0.142s)
\end{verbatim}
Then, we perform ten runs of the test set with stride variation,
by running the shell script \afile{dragon.sh} on our local machine:
\begin{verbatim}
% cat dragon.sh
adb shell /data/tmp/run.exe -s 5k -r 2k -i 1 -st 133 > CO.00
...
adb shell /data/tmp/run.exe -s 5k -r 2k -i 1 -st 37 > CO.09
% sh dragon.sh
\end{verbatim}
After a bit less then $20$~minutes we get ten files
\texttt{CO.00} to~\texttt{CO.09} in directory \file{SRC/DRAGON}
that we move into some sub-directory \texttt{dragon}.

\subsubsection{Other machines}
We also run our test set on more recent
Google nexus7 tablet (dual-core Cortex-A15 based Exynos5250, runing Android)
and Hardkernel ODROID-XU development board
(quad-core CortexA7-Cortex-A15 in tandem based Exynos5410, running Linux),
resulting in log files into sub-directories \file{nexus10} and~\file{odroid-xu}.

\fi


\clearpage
\part{Automating\label{part:auto} the testing process}
\cutname{auto.html}

The authors of~\dont{} are Jade Alglave and Luc Maranget
(INRIA Paris--Rocquencourt).

\section{Preamble}
Following Part~\ref{part:diy}, we describe our tests \emph{via} cycles, built
from the candidate relaxations they involve.  We consider a candidate
relaxation to be \emph{relaxed}, or \emph{non-global}, when it corresponds to
the weaknesses that can be observed on a system implementing $A$. We consider a
candidate relaxation to be \emph{safe}, or \emph{global}, when it is
guaranteed, $\eg$ by the documentation, never to be relaxed.

In the following, we consider an architecture $A$ to be a pair $(\relaxs_{A},
\safes_{A})$, where $\relaxs_{A}$ (resp.  $\safes_{A}$) are the candidate
relaxations relaxed (resp. safe) for $A$.
The automated front-end \dont{} mechanises the task of checking
that a machine or executable model conforms to such an architecture, and of
exploring architectures.
We provide some experiment reports
\footahref{http://diy.inria.fr/dont/dont/index.html}{elsewhere}.
This document is intended to be a gentle introduction to~\dont{} and
a partial reference.



\section{A tour of~\dont}

\subsection{Checking \aname{conform}{conformance}}
We want to check that a given machine $M$ is conform to an
architecture $A$. By conform, we mean that the machine $M$ does not exhibit
more behaviours than the architecture $A$ actually allows.

For example, let us consider an x86 machine with $2$~processors. Suppose that
we have been told that x86 machines are TSO, and that we want
to check that.
As the default values of~\dont{} options handle that very situation, we type:
\begin{verbatim}
$ dont -mode conform
** Step 0 **
Phase 2 in A (6 tests)
...
Phase 2 in A (6 tests)
** Step 5 **
Safe set {Rfe, Fre, Wse, PodWW, PodRW, PodRR, MFencedWR} is conform
\end{verbatim}
The automated front-end \dont,
assumed the TSO safe set (the default for x86),
called the \diy{} tool (see Part~\ref{part:diy}) to generate
all the tests that are forbidden by TSO --- up to $2$~processors;
ran them ($5$ times) with our companion \prog{litmus} tool, (see
Part~\ref{part:litmus}) against our x86 machine;
and observed that the machine does not exhibit any outcome
forbidden by TSO.
In effect, \dont{} in conformance check mode automates the safe
tests of Sec.~\ref{safe:test:sec}. 


\subsection{Checking  non-conformance}
Now, we wish to prove that an x86 machine is not
sequentially consistent.
To that end, we write the following configuration file~\afile{x86.sc}:
\verbatiminput{x86.sc}
Most of \dont{} controls are set, sometimes to
their default values:
\begin{itemize}
\item \opt{arch = X86} sets the targeted architecture,
\opt{mode = conform} sets conformance check mode,
and \opt{stablise = 1} commands performing the check round once
(the default is five times, cf. \emph{supra}).
\item \opt{safe = Rfe,Fre,Wse,Pod**,[Rfi,PodRR]} defines
the set of safe relaxation candidates used to generate litmus tests
(up to $2$ processors, by \opt{nprocs = 2}).
\item The front-end~\dont{} calls \litmus{} and runs the tests
with the specified options.
The setting \opt{litmus\_opt = -a 2 -i 0} specifies that two processors
are available and enables affinity control (see Sec.~\ref{litmus:option:sec}
for the description of \litmus{} options).
Tests will be run twice per check round, once
with options \opt{-s 100000 -r 10},
and once with options \opt{-s 5000 -r 200 -i 1}
(see Sec.~\ref{exec:control} for the description of test executable options).
Finally, the setting \opt{build = make -j 2 -s} specifies the command to use
to compile the C~source files that \litmus{} produces.
\end{itemize}
We run \dont{} configured by \afile{x86.sc} as follows:
\begin{verbatim}
$ dont x86.sc 
** Step 0 **
Phase 2 in A (9 tests)
...
** Step 1 **
Safe set {[Rfi,PodRR], Rfe, Fre, Wse, PodWW, PodWR, PodRW, PodRR} is not conform
++ Invalidating tests ++
A006: 'Fre PodWR Fre PodWR' {Fre, PodWR}
A007: 'Fre PodWW Wse PodWR' {Fre, Wse, PodWW, PodWR}
A001: 'Rfi PodRR Fre PodWR Fre' {[Rfi,PodRR], Fre, PodWR}
A002: 'Rfi PodRR Fre PodWW Wse' {[Rfi,PodRR], Fre, Wse, PodWW}
A000: 'Rfi PodRR Fre Rfi PodRR Fre' {[Rfi,PodRR], Fre}
++++++++
\end{verbatim}
The conformance check failed and the tests that invalidate the hypothesis
``x86 is sequentially consistent'' are listed.
The check took place in directory~\opt{A}.
Directory~\opt{A} contains the actual logs of \litmus{} runs
as files \opt{A.00}, \opt{A.01} etc.,
in addition to the sources of the litmus tests:
\begin{verbatim}
$cat A/A006.litmus
X86 A006
"Fre PodWR Fre PodWR"
Cycle=Fre PodWR Fre PodWR
Relax=
Safe=Fre PodWR
{ }
 P0          | P1          ;
 MOV [x],$1  | MOV [y],$1  ;
 MOV EAX,[y] | MOV EAX,[x] ;
exists (0:EAX=0 /\ 1:EAX=0)
\end{verbatim}
Notice that, since tests are described by their cycles,
the source of tests can also be reconstructed with~\ahrefloc{diyone}{\diyone}:
\begin{verbatim}
% diyone -arch X86 Fre PodWR Fre PodWR
X86 a
"Fre PodWR Fre PodWR"
{ }
 P0          | P1          ;
 MOV [y],$1  | MOV [x],$1  ;
 MOV EAX,[x] | MOV EAX,[y] ;
exists (0:EAX=0 /\ 1:EAX=0)
\end{verbatim}

\subsection{Automatically exploring the memory model exhibited by a  machine}
Now suppose that we have no idea of the memory model of
our $2$ processors x86 machine.
Another mode of our \dont{} tool automatically explores a given
machine, and outputs an architecture ($\ie$ a pair $(\relaxs_{A},
\safes_{A})$) to which the machine conforms.
The following configuration file~\afile{x86.explo}
instructs~\dont{} to perform such an exploration.
\verbatiminput{x86.explo}
With respect to conformance check, new or changed settings are the selection
of exploration mode by \opt{mode = explo},
the definition of the initial safe set by \opt{safe = Fre,Wse},
and and the definition of the candidate relaxations to be tested
(\opt{testing = Rfe,Pod**,MFenced**,[Rfi,PodR*]}).


We launch the exploration as:
\begin{verbatim}
$ dont x86.explo
\end{verbatim}
The whole process only takes a few minutes, mostly
due to the limited number of tests induced by the setting \opt{nprocs = 2}.

We now detail \dont{} output
\ifhevea (complete log of the exploration: \afile{x86-log.txt})
\else (the \footahref{\url{http://diy.inria.fr/doc/auto.html}}{html version}
of this document includes the complete log of the experience)\fi.
We start by a first exploration round:
\begin{verbatim}
** Step 0 **
Testing: {[Rfi,PodRW], [Rfi,PodRR], Rfe, PodWW, PodWR, PodRW, PodRR, MFencedWW, 
MFencedWR, MFencedRW, MFencedRR}
Relaxed: {}
Safe   : {Fre, Wse}
Phase 1 in A (6 tests)
Actually tested: {[Rfi,PodRW], [Rfi,PodRR], PodWW, PodWR, MFencedWW, MFencedWR}
Added relax: {[Rfi,PodRR], PodWR}
Added safe: {[Rfi,PodRW], PodWW, MFencedWW, MFencedWR}
Phase 2 in B (6 tests)
\end{verbatim}
The log above first indicates the current status of exploration
as three sets: \emph{testing}, \emph{relaxed} and~\emph{safe}.
Initially, no candidate relaxation has yet been observed to be relaxed,
while the testing and safe sets are as assumed.
Each exploration round is divided in two phases.
The aim of Phase~1 (performed in directory~\opt{A})
is to classify some candidate relaxations as either relaxed or safe.
It here
succeeds for 6 candidate relaxations, whose observed status is indicated.
Phase~2 (performed in directory~\opt{B})
basically is a conformance check of the current safe set.
The conformance check succeeds and all safe candidate relaxations
found at phase~1 make it to the next round:
\begin{verbatim}
** Step 1 **
Testing: {Rfe, PodRW, PodRR, MFencedRW, MFencedRR}
Relaxed: {[Rfi,PodRR], PodWR}
Safe   : {[Rfi,PodRW], Fre, Wse, PodWW, MFencedWW, MFencedWR}
Phase 1 in C (10 tests)
Actually tested: {Rfe, PodRW, PodRR, MFencedRW, MFencedRR}
Added safe: {Rfe, PodRW, PodRR, MFencedRW, MFencedRR}
Phase 2 in D (17 tests)
\end{verbatim}
Phase~1 (performed in directory~\opt{C})
can now target new candidate relaxations, because of the increased
safe set. All of targeted candidate relaxations
are observed to be safe, which is confirmed
by phase~2. As a  consequence, there does not remain
any candidate relaxation to be tested and the next round reduces
to a conformance check:
\begin{verbatim}
** Step 2 **
Testing: {}
Relaxed: {[Rfi,PodRR], PodWR}
Safe   : {[Rfi,PodRW], Rfe, Fre, Wse, PodWW, PodRW, PodRR, MFencedWW, MFencedWR, MFencedRW, MFencedRR}
Phase 1 in E (0 tests)
Phase 2 in D (17 tests)
\end{verbatim}
The same check is performed
for $4$~additional rounds as governed by the default
value of $5$ for the setting of~\opt{stabilise}.
Round number~$6$ then shows the result of exploration,
(\emph{i.e.} the pair $(\relaxs_{A}, \safes_{A})$), prefixed
by the list of tests that justify observed relaxations:
\begin{verbatim}
** Step 6 **
...
++ Witness(es) for relaxed [Rfi,PodRR] ++
A001: 'Rfi PodRR Fre Rfi PodRR Fre' {[Rfi,PodRR], Fre}
++++++++
++ Witness(es) for relaxed PodWR ++
A003: 'Fre PodWR Fre PodWR' {Fre, PodWR}
++++++++
Observed relaxed: {Rfi, PodWR}
Observed safe: {Rfe, Fre, Wse, PodWW, PodRW, PodRR, MFencedWW, MFencedWR, MFencedRW, MFencedRR}
\end{verbatim}
And we go again for $5$~additional rounds of pure conformance check:
\begin{verbatim}
** Now checking safe set conformance **
** Step 7 **
Phase 2 in F (17 tests)
...
* Step 12 **
Observed relaxed: {Rfi, PodWR}
Observed safe: {Rfe, Fre, Wse, PodWW, PodRW, PodRR, MFencedWW, MFencedWR, MFencedRW, MFencedRR}
\end{verbatim}


Once exploration is complete, all litmus tests and logs of \litmus{} runs are
still present in their directories \opt{A}, \opt{B}, etc.
For instance, the directory~\opt{F} contain the $10$ logs of the final
conformance check, as the files \opt{F.01}, \ldots, \opt{F.09}:
\begin{verbatim}
$ ls F/F.??
F/F.00  F/F.01  F/F.02  F/F.03  F/F.04  F/F.05  F/F.06  F/F.07  F/F.08  F/F.09
\end{verbatim}

The tool \dont{} offers a convenient replay feature:
\begin{verbatim}
$ dont -restart
** Step 0 *
...
* Step 12 **
Observed relaxed: {Rfi, PodWR}
Observed safe: {Rfe, Fre, Wse, PodWW, PodRW, PodRR, MFencedWW, MFencedWR, MFencedRW, MFencedRR}
\end{verbatim}
The command above takes a few seconds of time, since experiments
are not run again. Instead, the logs of \litmus{} runs are read and
their interpretation is re-performed.
Notice that the restart feature also permits to pursue interrupted
experiments.

\section{Usage of~\dont}
In effect, the tool~\dont{} automates the complete testing
procedure described in the documentation of \diy{} proper
(Sec.~\ref{diy:intro}).
It is to be noticed that \dont{} requires a fully functional installation
of the \diy~tool suite.
In particular, the commands~\diy{} and~\litmus{} must be installed
and runnable as ``\diy'' and~``\litmus'' (\emph{i.e.} installed in path).



\subsection{Command-line options}
The automated front-end \dont{} is configured mostly by the means
of a configuration file, which \dont{} takes as a command-line argument.
Nevertheless, \dont{} accepts the following, limited, set of options:
\begin{description}
\item[{\tt -v}] Be verbose, repeat to increase verbosity.
\item[{\tt -version}]  Show version number and exit.
\item[{\tt -arch (X86|PPC|ARM)}] Set architecture. Default is~\opt{X86}.
ARM is untested.
\item[{\tt -mode (conform|explo)}] Set main mode,
either conformance check or exploration. Default is~\opt{explo}.
\item[{\tt -nprocs <n>}]
Generate tests up to <n> processors (defaults: X86=2, PPC=4)
\item[{\tt -restart}] Restart the experiment in hand in current directory.
\end{description}
Except for \opt{-restart} command lines options are not intended
for normal use.
In particular, command-line options do not override values defined
in configuration files.

Namely, there are many parameters to set and appropriate values
for them will depend on the tested machine.
In particular, \litmus{} parameters need to be chosen carefully, by
the means of preliminary experiments.
For instructions on configuring~\litmus{}, refer to
Sec.~\ref{litmus:control} of \litmus{} documentation.


\subsection{Configuration files}
The general syntax of configurations files is a sequence of
lines \textit{key}\texttt{ = }\textit{value}.
Comment lines are introduced by \verb+#+.
The tool \dont{} recognises the following keys:

\paragraph*{General behaviour}
\begin{description}
\item[{\tt mode = (conform|explo)}]
Main operating mode. Default is~\opt{explo}
\item[{\tt arch = (X86|PPC|ARM)}]
Target architecture. Default is~\opt{X86}.
\item[{\tt run = (local|ssh <addr>|cross <addr1> <addr2>)}]
Give access to the tested machine,
which can be either the machine where \dont{} runs,
or remote machine \opt{<addr>},
or compile C~files on remote machine \opt{addr1} and execute on
tests on remote machine \opt{addr2}.
Machine addresses are \opt{[user@]machine[:port]} expressing
connection elements for both \texttt{ssh} and~\texttt{scp}.
Default is~\opt{local}.

\item[{\tt work\_dir = \textit{dir}}]
Directory for temporary files, default is \opt{/var/tmp}.
\item[{\tt stabilise = <\textit{n}>}]
In conformance check mode, \dont{} performs $n$ rounds of conformance testing.
In exploration mode, \dont{} ends the exploration after $n$~rounds
without state change. Default is~$5$.
\item[{\tt interactive = <bool>}]
In exploration mode and after $n$~rounds
without state change, \dont{} will either assume that the whole current
testing set is safe (\opt{false}), or ask the user (\opt{true}) to
decide for some of the elements of this set to be safe.
Default is~\opt{true}, \emph{i.e.} ask user.
\end{description}

\paragraph{Controlling Cycle Generation}
\begin{description}
\item[{\tt nprocs = <\textit{n}>}]
Generate cycles up to $n$ processors. Default is $2$ for~x86 and~$4$ for Power.
\item[{\tt diy\_sz = <\textit{m}>}]
Upper limit on the size of cycles of candidate relaxations.
Default is $2 \times n$, where $n$ is the number of processors.
With decent values of the initial candidate relaxations sets
(see below), this default commands the generation of all
(critical, see~Sec.~\ref{critical:def}) cycles that involve
up to $n$~processors.
\item[{\tt safe = <\textit{relax-list}>}]
Define the safe set~$S$. In exploration mode, $S$ is the initial value
of the safe set (default \opt{Fre, Wse}). In conformance
mode, $S$ is the safe set checked.
Ddefault is \opt{Rfe, Fre, Wse, PodR*, PodWW, MFencedWR} for x86,
and unspecified for other architectures.
\item[{\tt testing = <\textit{relax-list}>}]
Define the tested set of candidate relaxations.
The tested set is relevant only in exploration mode.
Default values are
\opt{Rfe,Pod**,MFenced**,[Rfi,MFencedR*],[Rfi,PodR*]} for x86
and unspecified for other architectures.
\end{description}
The syntax for \textit{relax-list} above is a comma (or space) separated
list of candidate relaxations.
Candidate relaxations are introduced by the documentation of~\diy{}
(see Part~\ref{part:diy})

\paragraph{Control of external tools}
\begin{description}
\item[{\tt litmus\_opts = <\textit{opts}>}]
Define options used by~\dont{} when it calls \litmus.
Default is the empty string, \emph{i.e.} use \litmus{} defaults.
\item[{\tt run\_opts = <\textit{opts$_1$},\ldots,\textit{opts$_n$}>}]
Define options used for running litmus tests.
Any set of litmus tests generated and compiled by~\dont,
will be run $n$ times, with specified options.
More concretely, \dont{} will run the litmus tests with commands
\texttt{sh run.sh \textit{opts$_1$}}, \ldots,
\texttt{sh run.sh \textit{opts$_n$}}.
The default is the empty string, \emph{i.e.} run tests once with no option.
\item[{\tt build = <\textit{command}>}]
Defines the command issued by~\dont{} to compile the C~source files
produced by \litmus.
The default is \opt{sh comp.sh}, \emph{i.e.} runs the compilation script
produced by \litmus. An interesting alternative is
\texttt{make -s -j \textit{n}} for concurrent compilation,
with up to~$n$ concurrent tasks.
\end{description}

\endinput
We now perform a TSO-conformance check of our machine \opt{saumur}.
The machine \opt{saumur} is an  $8$~(logical) processors Intel Xeon.
However,
we shall limit conformance tests to those that involve up to $4$~processors
--- generating tests up to $8$~processors is doable, but the conformance
check would then last several days.
We write the following configuration file~\afile{saumur.tso}:
\verbatiminput{saumur.tso}
The configuration file above for~\dont{}
first specifies conformance check mode,
instructs \diy{} to generate cycles up to $4$ processors,
then instructs \dont{} to call \litmus{} with the given options.
Again, the there will be $5$~runs of tests.
However (\opt{run\_opts = -i 1,-i 4}), each of these five runs will consist
in running the tests twice, first with option \opt{-i 1} and then
with option \opt{-i 4}. Finally tests are compiled with the specified build
command.
We run \dont{} as follows:
\begin{verbatim}
$ dont saumur.tso
** Step 0 **
Phase 2 in A (68 tests)
...
** Step 5 **
Safe set {Rfe, Fre, Wse, PodWW, PodRW, PodRR, MFencedWR} is conform
\end{verbatim}
The conformance check lasts for about one hour and succeeds.


\bibliographystyle{plain}
\bibliography{jade}
\end{document}
